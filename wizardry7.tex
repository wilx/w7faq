\documentclass[10pt,twoside,openright]{report}
\usepackage[a4paper,outer=1.5cm,outer=1.5cm,bindingoffset=0.75cm,twoside]{geometry}
\usepackage{amssymb,amsmath} \usepackage{ifxetex,ifluatex}
%\usepackage{fixltx2e} % provides \textsubscript
\usepackage{fontspec}
\usepackage{microtype}
\usepackage[font={slshape},leftmargin=0.5\leftmargin,rightmargin=0.5\rightmargin]{quoting}
\usepackage{multicol}
\usepackage[defaultlines=2,all]{nowidow}
\usepackage{ragged2e}
\usepackage{hyphenat}
\usepackage{xstring}
\usepackage[super]{nth}

\defaultfontfeatures{Ligatures=TeX,Scale=MatchLowercase}
%\setmainfont[]{Charis SIL}
%\setmainfont[]{TeX Gyre Pagella}
\setmainfont[]{Linux Libertine O}
%\setmainfont[]{Noto Serif}

\newcommand{\fractionsOn}[1]{{\addfontfeature{Fractions=On}#1}}

%\usepackage{bigfoot}
%\DeclareNewFootnote{default}

% This must be loaded before hyperref or bookmarks packages so that the
% generated Index is clickable.
\usepackage[xindy]{imakeidx}

% Bookmark package loads hyperref internally.
\usepackage{bookmark}
\hypersetup{unicode=true,
            pdftitle={A random Crusaders of the Dark Savant FAQ},
            pdfauthor={Written by: Ravashack},
            pdfborder={0 0 0},
            breaklinks=true}
\urlstyle{same}  % don't use monospace font for urls

\usepackage{enumitem}
\usepackage{longtable,booktabs}
\usepackage{array}
\usepackage{calc}
\usepackage{parskip}

\newcolumntype{P}[1]{>{\raggedright\arraybackslash}p{#1}}

\setlength{\emergencystretch}{3em}  % prevent overfull lines
\tolerance=1000
\providecommand{\tightlist}{%
  \setlength{\itemsep}{0pt}\setlength{\parskip}{0pt}}
\makeatletter
\newcommand*\nobreaklist{\@beginparpenalty=\@M}
\makeatother
\setcounter{secnumdepth}{0}

% Redefines (sub)paragraphs to behave more like sections
\ifx\paragraph\undefined\else
\let\oldparagraph\paragraph
\renewcommand{\paragraph}[1]{\oldparagraph{#1}\mbox{}\nopagebreak}
\fi
\ifx\subparagraph\undefined\else
\let\oldsubparagraph\subparagraph
\renewcommand{\subparagraph}[1]{\oldsubparagraph{#1}\mbox{}}
\fi

\usepackage{background}

\backgroundsetup{%
scale=1,
angle=0,
contents={%
  \IfFileExists{parchemin_6.jpg}{%
    \includegraphics[width=\paperwidth,height=\paperheight]{parchemin_6.jpg}}{}
}}


\usepackage{tikz}
\usetikzlibrary{decorations.pathreplacing,calc}
\usetikzlibrary{matrix}
\makeatletter
\newcommand{\tikzmark}[1]{\protect\tikz[overlay,remember picture] \node (#1) {};}
\makeatother

\title{A random Crusaders of the Dark Savant FAQ}
\author{Written by: Ravashack}
\date{v.\ 1.95b}

\usepackage{needspace}
\makeatletter
% The following macro makes sure there is enough space on page for following
% text. It is used to glue text before lists to the lists.
\newcommand{\WviiNeedSpace}{\Needspace*{4\baselineskip+\parskip}}
\makeatother

\makeindex[name=monsters,title=Monsters index]
\makeindex[name=npcs,title=NPCs index]
\makeindex[name=spells,title=Spells index]
\makeindex[name=places,title=Places index]

\makeatletter
\DeclareRobustCommand{\indexSpell}[1]{\index[spells]{#1}}
\DeclareRobustCommand{\spell}[1]{\texorpdfstring{#1\indexSpell{#1}}{#1}}
\DeclareRobustCommand{\indexPlace}[1]{\index[places]{#1}}
\DeclareRobustCommand{\place}[1]{\texorpdfstring{#1\indexPlace{#1}}{#1}}
\DeclareRobustCommand{\indexRace}[1]{\index{#1}}
\DeclareRobustCommand{\race}[1]{\texorpdfstring{#1\indexRace{#1}}{#1}}
\DeclareRobustCommand{\indexClass}[1]{\index{#1}}
\DeclareRobustCommand{\class}[1]{\texorpdfstring{#1\indexClass{#1}}{#1}}
\DeclareRobustCommand{\indexMonster}[1]{%
  \IfEndWith{#1}{s}{\StrGobbleRight{#1}{1}[\WviiResult]%
  \index[monsters]{\WviiResult}}{%
  \index[monsters]{#1}}}
\DeclareRobustCommand{\monster}[1]{\texorpdfstring{#1\indexMonster{#1}}{#1}}
\DeclareRobustCommand{\monsterB}[2]{\texorpdfstring{#1\indexMonster{#2}}{#1}}
\DeclareRobustCommand{\indexNpc}[1]{\index[npcs]{#1}}
\DeclareRobustCommand{\npc}[1]{\texorpdfstring{#1\indexNpc{#1}}{#1}}
\makeatother

\newcommand\textlcsc[1]{\texorpdfstring{\textsc{\MakeLowercase{#1}}}{#1}}

\newcommand{\faqentry}[1]{%
  \pagebreak[1]\paragraph{Q\@: #1}\phantomsection\addcontentsline{toc}{subsubsection}{#1}\nopagebreak\par}

\newcommand{\spellentry}[1]{%
  \mbox{\textbf{\spell{#1}}\quad}\phantomsection\addcontentsline{toc}{subsubsection}{#1}\nopagebreak}

\newcommand{\WviiTwoColumnSetup}{\raggedcolumns\RaggedRight}

\newcommand{\qmwordqm}[1]{\mbox{?#1?}}
\DeclareRobustCommand{\WviiSPOT}{\mbox{\monster{*S~P~O~T*}}}
\DeclareRobustCommand{\WviiDOOM}{\mbox{\monster{*D~O~O~M*}}}
\DeclareRobustCommand{\WviiSPAWN}{\mbox{\monster{*S~P~A~W~N*}}}

%before=\nopagebreak\setlength{\multicolsep}{\topsep}\begin{multicols}{2}\WviiTwoColumnSetup,
\SetEnumitemKey{WviiTwoColumn}{%
  before=\begin{multicols}{2}\WviiTwoColumnSetup,
  after=\end{multicols}}

\newenvironment{twocolumnitemize}[1]
 {%
  \begin{multicols}{2}[#1]
  \raggedcolumns\RaggedRight
  \begin{itemize}
 }
 {%
  \end{itemize}
  \end{multicols}
 }

\makeindex

\raggedbottom

%%
%% Document
%%

\begin{document}
\begin{titlepage}
  \centering
  {\addfontfeature{StylisticSet=2}\Huge A random\\\uppercase{Crusaders}\\{\fontspec{EB Garamond}\textit{of the}}\\\uppercase{Dark Savant}\\FAQ\par}
  \vspace{1.5cm}
  {\Large Written by: \textit{Ravashack}\protect\footnote{\LaTeX{} conversion done by Václav Haisman.
See also GitHub repository \href{https://github.com/wilx/w7faq}{\texttt{wilx/w7faq}}.}\par}
  \vspace{1.5cm}
  {\large v.\ 1.95b\par}
  \vfill

  % Bottom of the page

\end{titlepage}
\mbox{}
\thispagestyle{empty}
\newpage

\chapter*{Introductions}%
%
\section{Mandatory Copyright
information}\label{mandatory-copyright-information}%
%
This file is copyright 2000 by Ravashack, a.k.a. Leo Wang in the non-internet
world. If you want to use any of the information here in any form, feel free
to contact me at
\href{mailto:asianfro@hotmail.com}{\nolinkurl{asianfro@hotmail.com}} to get
my permission---that way I can keep better track of who's using it and who
isn't. If I don't reply back, then I'm busy at the moment and that the answer
to whatever your request is is no---never assume that I will automatically
answer yes. Usage of this document without my permission violates copyright
laws, no matter what you say. For an interesting essay on copyright myths by
Brad Templeton, take a look at
\url{http://www.templetons.com/brad/copymyths.html} and read to your heart's
content.

\section{\texorpdfstring{A little ``getting to know you'' stuff and
Contact
Info}{A little getting to know you stuff and Contact Info}}\label{a-little-getting-to-know-you-stuff-and-contact-info}%
%
This here is a FAQ for the game Crusaders of the Dark Savant. It can also be
used for people playing Wizardry Gold, which is the same thing with more
bugs. \verb|^_^| It is the first FAQ I have ever made, so feel free
to point out the mistakes, screw-ups, and suggestions that I'm sure will be
in here to me at asianfro@hotmail.com. This email address will
\textlcsc{ALWAYS} be in use unless I update this FAQ stating that I have
closed this email. I have problems answering email when it is not a normal
working weekday however. (You do the math\ldots{})

If you wish to chat with me, I'm on ICQ as Orlandu. Of course, I'm not just
going to \emph{give} out my UIN just like that---I want to know who you are
before I let you add me, which means I'll want an email or two from you to
check that you aren't the malicious type; I'm getting too much spam to be
trusting anymore. If you wish to avoid that hassle, I am also (occasionally)
on the Galaxynet IRC server as ``Rava'' or ``Ravashack'' on weekdays---just
login to \url{http://www.galaxynet.org} with a browser or use mIRC to connect
to a Galaxynet server.

Minor updates such as spelling errors, typos, site additions, etc.\ will
be habitually done at \url{http://www.gamefaqs.com}. Major updates (such
as adding a new section or changing the format of the document) will be
done to all the sites that have this FAQ\ldots{}or at least the
administrator of the site or whoever it is that manages the FAQs will be
notified of the update.

Emails asking about cheat codes, editors or about games I have not
written a FAQ about will go under the Block Sender list and be ignored
for the rest of time. If you need to look up something like that, just
use a search engine like www.google.com and type in your query.

Just so you know, an actual walkthrough of this game \emph{requires}
maps. The game is just too big to do it effectively without them.
Because of that, I cannot enclose a full length walkthrough until I
solve my map problem---otherwise I would. It is currently in the works
however.

\begin{center}\rule{0.5\linewidth}{\linethickness}\end{center}

The big appreciative `Thank You' goes to

Llevram, who decided to help me out on this little venture by looking over my
pre-born FAQ and suggesting a lot of things---too many to thank him
individually on each section he suggested. His site is at:
\url{http://www.softwarespecialties.com/}. If you really have to cheat and
aren't stingy about it, at least take a look at the programs he's written for
Wizardry~7, even though Wiz7Edit must be bought Excellent interface, and the
Item Viewer Wiz7Item (a separate program) is more accurate than the in-game
methods! Llevram's always there to help you with problems if you need him.

The folks on the Software Specialties board for the surprises that pop
up about the game every few days and so. You know who you are. Thanks
for putting up with me ever since I joined

Sir-Tech Canada, for trying to get Wizardry~8 done even though they are
having difficulties. I know you guys will find a way\ldots{}

D.W. Bradley, for being the brains behind the game. He's working with a
different company on a game similar in view to Dark Savant. If you liked
Crusaders of the Dark Savant, you'll probably like his upcoming game as
it seems similar. Try visiting \url{http://www.heuristicpark.com}.
(Unfortunately it is quite buggy\ldots{})

DSimpson a.k.a.\ ManyMoose, a GameFAQs FAQ writer, for unconsciously
planting the idea in my head that I should try writing a FAQ a year or
so ago. The lengths of his FAQs should only show that there is a wealth
of information inside.

Some of my friends at Panumbra, who reminded me of the existence of this
game with a few off-topic posts on the webboard.

Anyone who I've actually helped at the game for listening to me, and
anyone who I've missed. \verb|:)|

Anyone who sent in a question not covered here. I appreciate the
feedback guys. \verb|:)|

The folks at various game sites for being interested in my
FAQ---especially the ones that put it up.

GameFAQs and CJayC, just for being there (and posting this up).

Dallas on the GameFAQs message boards, for taking the time to review
this FAQ and give pointers about it.

The Ironworks Wizardry~8 board at
\url{http://www.ironworksforum.com/ubb/cgi-bin/ultimatebb.cgi} for just
being a Wizardry~8 board.

Bernice Carter, a nice 40-something year old lady who wishes to point
out that not all avid gamers are young male ``whippersnappers.''

Baz, for reminding me about an oversight.

\begin{center}\rule{0.5\linewidth}{\linethickness}\end{center}

\WviiNeedSpace{}This FAQ is currently displayed at:%
\begin{itemize}
\tightlist
\item
  \url{http://www.gamefaqs.com/}
\item
  \url{http://www.the-spoiler.com/}
\end{itemize}

If you want to doublecheck where this FAQ should be, go to
\url{http://www.gamefaqs.com} and look. If you want to alert me to
someone who may be ripping off my FAQ, feel free to drop me a line---I
will appreciate it.

\clearpage
\chapter*{Contents}
\begin{multicols}{2}\RaggedRight
%\tableofcontents
\makeatletter
\@starttoc{toc}
\makeatother
\end{multicols}

\chapter{Character Creation}\label{character-creation}%
%
\section{Male vs.~Female}\label{male-vs.female}%
%
What's the difference? Females, when they start out, get a −2 in
Strength, +1 in Personality, and a +1 in Karma compared to the base
statistics given in the manual. Males start out with the base statistics
given in the manual.

\subsection{Males}\label{males}%
%
\begin{multicols}{2}\WviiTwoColumnSetup
\WviiNeedSpace{}Perks:%
\begin{itemize}
\item
  Classes based on strength are easier to get for males.
\item
  Males can wear the Cameo Locket.
\end{itemize}
\columnbreak

\WviiNeedSpace{}Flaws:%
\begin{itemize}
\item
  See benefits of Females.
\item
  Cameo Locket is only attainable by importing a party from Bane of the
  Cosmic Forge or saved game editing.
\end{itemize}
\end{multicols}

\subsection{Females}\label{females}%
%
\begin{multicols}{2}\WviiTwoColumnSetup
\WviiNeedSpace{}Perks:%
\begin{itemize}
\item
  Classes based on Personality will require 1 less point to get for
  females, unless it has a strength requirement as well.
\item
  Females can use certain items like Stud-Cuir Bras and Jazeraint
  Skirts. (Sorry, no cross-dressers in the game.) Some of these items
  are the best for the particular class the female may end up in.
\item
  Females can become \class{Valkyrie}s, which are \class{Lord}s with less
  experience point requirements and no guaranteed starting Diplomacy.
\end{itemize}
\columnbreak

\WviiNeedSpace{}Flaws:%
\begin{itemize}
\item
  See benefits of Males.
\item
  Diamond Ring, one of the better female-only items, must be imported
  from Bane or edited into the saved game.
\end{itemize}
\end{multicols}

\subsection{Overall}\label{overall}%
%
Use Male characters only if you are importing a game from Bane of the
Cosmic Forge, if you are rolling up a \class{Fighter}-type character or a
\class{Psionic}, or you cheat

Use Female characters whenever you are rolling up a spellcaster and all
other situations that don't apply to male characters---you don't lose
anything out of it, and it gives you the option to class change into a
\class{Valkyrie} later.

\section{Races}\label{races}%
%
\WviiNeedSpace{}Basic Layout:%
\begin{itemize}[topsep=0pt]
\tightlist
\item Name of Race
\item Race's Stats
\item List of Perks
\item List of Flaws
\item A chart of how many points the race needs to achieve a class.
\end{itemize}

\begin{longtable}[]{@{}lccccccccr@{}}
\caption{Sample Chart:}\tabularnewline
\toprule
Class Name & STR & INT & PIE & VIT & DEX & SPE & PER & F? &
Total\tabularnewline
\midrule
\endfirsthead
\toprule
Class Name & STR & INT & PIE & VIT & DEX & SPE & PER & F? &
Total\tabularnewline
\midrule
\endhead
Fighter & 3 & & & & & & & N & 3\tabularnewline
\bottomrule
\end{longtable}

\WviiNeedSpace{}Explanations:%
\begin{description}[style=nextline, labelwidth=4.5em, leftmargin=!, labelindent=0em]\RaggedRight
\tightlist
\item[STR] Points towards Strength needed.
\item[INT] Points towards Intelligence needed.
\item[PIE] Points towards Piety needed.
\item[VIT] Points towards Vitality needed.
\item[DEX] Points towards Dexterity needed.
\item[SPE] Points towards Speed needed.
\item[PER] Points towards Personality needed.
\item[F?] Should I use a Female for the class?
\item[Y] Yes, it will lower the amount of points needed.
\item[--] Doesn't matter.
\item[N] No, it won't help you in the points.
\item[Total] Total amount of points needed for the class.
\end{description}

\subparagraph{Note:} Any class with Y will have point distribution altered to
reflect stat changes for female characters.

\WviiNeedSpace{}Convenient starting professions:%
\begin{itemize}
\tightlist
\item
  A profession(s) which the race excels at naturally OR A non-basic (not
  \class{Fighter}, \class{Mage}, \class{Priest}, or \class{Thief}) profession
  that can be achieved with a below average amount of points out of all the
  races OR A basic profession achievable within 2 points or less.
\end{itemize}

\pagebreak[1]\subsection{\race{Human}}\label{human}%
%
Base \race{Human} stats: 9 8 8 9 9 8 8

\begin{multicols}{2}\WviiTwoColumnSetup
\WviiNeedSpace{}Perks:%
\begin{itemize}
\tightlist
\item
  No inherent weaknesses to an attack type.
\end{itemize}
\columnbreak

\WviiNeedSpace{}Flaws:%
\begin{itemize}
\tightlist
\item
  No inherent resistances to an attack type.
\item
  Too average.
\item
  Someone else can fit a class better than they do in most cases.
\end{itemize}
\end{multicols}

\begin{longtable}[]{@{}lccccccccr@{}}
\caption{Points needed to meet Class Requirements for
\race{Human}:}\tabularnewline
\toprule
Class Name & STR & INT & PIE & VIT & DEX & SPE & PER & F? &
Total\tabularnewline
\midrule
\endfirsthead
\toprule
Class Name & STR & INT & PIE & VIT & DEX & SPE & PER & F? &
Total\tabularnewline
\midrule
\endhead
Fighter & 3 & & & & & & & N & 3\tabularnewline
Mage & & 4 & & & & & & - & 4\tabularnewline
Priest & & & 4 & & & & & - & 4\tabularnewline
Thief & & & & & 3 & & & - & 3\tabularnewline
Ranger & 1 & & & 2 & 1 & & & N & 4\tabularnewline
Bard & & 2 & & & 3 & & 3 & Y & 8\tabularnewline
Psionic & 1 & 6 & & 5 & & & 2 & N & 14\tabularnewline
Alchemist & & 5 & & & 4 & & & - & 9\tabularnewline
Valkyrie & 3 & & 3 & 2 & 1 & 3 & & Y & 12\tabularnewline
Bishop & & 7 & 7 & & & & & - & 14\tabularnewline
Lord & 3 & 1 & 4 & 3 & & 1 & 6 & N & 18\tabularnewline
Samurai & 3 & 3 & & & 3 & 6 & & N & 15\tabularnewline
Monk & 4 & & 5 & & 1 & 5 & & N & 15\tabularnewline
Ninja & 3 & 2 & 2 & 3 & 3 & 4 & & N & 17\tabularnewline
\bottomrule
\end{longtable}

\begin{twocolumnitemize}{Convenient starting professions:}
\item
  \class{Fighter}
\item
  \class{Thief}
\item
  \class{Ranger} (don't expect your spell points to come back quickly though)
\end{twocolumnitemize}

\pagebreak[1]\subsection{\race{Elf}}\label{elf}%
%
Base \race{Elf} stats 7 10 10 7 9 9 8

\begin{multicols}{2}\WviiTwoColumnSetup
\WviiNeedSpace{}Perks:%
\begin{itemize}
\tightlist
\item
  Higher resistance to sleep spells.
\item
  Combat-types can take advantage of a Race-specific weapon (Elven Bow).
\item
  Decent spell point recharge rate.
\end{itemize}
\columnbreak

\WviiNeedSpace{}Flaws:%
\begin{itemize}
\tightlist
\item
  Strength- and Vitality-based classes harder to get in the beginning.
\item
  Race has slightly lower hit points than normal.
\end{itemize}
\end{multicols}

\WviiNeedSpace{}Suggestions:
\begin{itemize}
\item
  When picking an \race{Elf}, pick female unless you fully intend to make it a
  fighting character.
\item
  Go for a spellcasting class.
\end{itemize}

\begin{longtable}[]{@{}lccccccccr@{}}
\caption{Points needed to meet Class Requirements for
\race{Elf}:}\tabularnewline
\toprule
Class Name & STR & INT & PIE & VIT & DEX & SPE & PER & F? &
Total\tabularnewline
\midrule
\endfirsthead
\toprule
Class Name & STR & INT & PIE & VIT & DEX & SPE & PER & F? &
Total\tabularnewline
\midrule
\endhead
Fighter & 5 & & & & & & & N & 5\tabularnewline
Mage & & 2 & & & & & & - & 2\tabularnewline
Priest & & & 2 & & & & & - & 2\tabularnewline
Thief & & & & & 3 & & & - & 3\tabularnewline
Ranger & 3 & & & 4 & 1 & & & N & 8\tabularnewline
Bard & & & & & 3 & & 3 & Y & 6\tabularnewline
Psionic & 3 & 4 & & 7 & & & 2 & N & 16\tabularnewline
Alchemist & & 3 & & & 4 & & & - & 7\tabularnewline
Valkyrie & 5 & & 1 & 4 & 1 & 2 & & Y & 13\tabularnewline
Bishop & & 5 & 5 & & & & & - & 10\tabularnewline
Lord & 5 & & 2 & 5 & & & 6 & N & 18\tabularnewline
Samurai & 5 & 1 & & 2 & 3 & 5 & & N & 16\tabularnewline
Monk & 6 & & 3 & & 1 & 4 & & N & 14\tabularnewline
Ninja & 5 & & & 5 & 3 & 3 & & N & 16\tabularnewline
\bottomrule
\end{longtable}

\begin{twocolumnitemize}{Convenient starting professions:}
\item
  \class{Mage}
\item
  \class{Priest}
\item
  \class{Bard}
\item
  \class{Alchemist}
\item
  \class{Bishop}
\end{twocolumnitemize}

\pagebreak[1]\subsection{\race{Dwarf}}\label{dwarf}%
%
Base \race{Dwarf} stats 11 6 10 12 7 7 7

\begin{multicols}{2}\WviiTwoColumnSetup
\WviiNeedSpace{}Perks:%
\begin{itemize}
\item
  Poison resistance
\item
    Magic resistance (i.e.\ to spells like \spell{Lifesteal} or \spell{Make
    Wounds})
\end{itemize}
\columnbreak

\WviiNeedSpace{}Flaws:%
\begin{itemize}
\tightlist
\item
  Slow. Don't expect them to go first anytime soon unless you sacrifice
  all your bonus points towards their Dexterity and Speed.
\end{itemize}
\end{multicols}

\begin{longtable}[]{@{}lccccccccr@{}}
\caption{Points needed to meet Class Requirements for
\race{Dwarf}:}\tabularnewline
\toprule
Class Name & STR & INT & PIE & VIT & DEX & SPE & PER & F? &
Total\tabularnewline
\midrule
\endfirsthead
\toprule
Class Name & STR & INT & PIE & VIT & DEX & SPE & PER & F? &
Total\tabularnewline
\midrule
\endhead
Fighter & 1 & & & & & & & N & 1\tabularnewline
Mage & & 6 & & & & & & - & 6\tabularnewline
Priest & & & 2 & & & & & - & 2\tabularnewline
Thief & & & & & 5 & 1 & & - & 6\tabularnewline
Ranger & & 2 & & & 3 & 1 & 1 & N & 7\tabularnewline
Bard & & 4 & & & 5 & 1 & 4 & Y & 14\tabularnewline
Psionic & & 8 & & 2 & & & 3 & N & 13\tabularnewline
Alchemist & & 7 & & & 6 & & & - & 13\tabularnewline
Valkyrie & & & 1 & & 3 & 4 & & Y & 8\tabularnewline
Bishop & & 9 & 5 & & & & & Y & 14\tabularnewline
Lord & 1 & 3 & 2 & & 2 & 2 & 7 & N & 17\tabularnewline
Samurai & 1 & 5 & & & 5 & 7 & 1 & N & 19\tabularnewline
Monk & 2 & 2 & 3 & & 3 & 6 & 1 & N & 17\tabularnewline
Ninja & 1 & 4 & & & 5 & 5 & & N & 15\tabularnewline
\bottomrule
\end{longtable}

\begin{twocolumnitemize}{Convenient starting professions:}
\item
  \class{Fighter}
\item
  \class{Priest}
\item
  \class{Ranger}
\item
  \class{Valkyrie}
\end{twocolumnitemize}

\pagebreak[1]\subsection{\race{Gnome}}\label{gnome}%
%
Base \race{Gnome} Stats 10 7 13 10 8 6 6

\begin{multicols}{2}\WviiTwoColumnSetup
\WviiNeedSpace{}Perks:%
\begin{itemize}
\tightlist
\item
  Magic resistance (i.e.~to spells like \spell{Lifesteal} or Make Wounds)
\end{itemize}
\columnbreak

\WviiNeedSpace{}Flaws:%
\begin{itemize}
\tightlist
\item
  Slow. Don't expect them to go first anytime soon unless you sacrifice
  all your bonus points towards their Speed.
\end{itemize}
\end{multicols}

\begin{longtable}[]{@{}lccccccccr@{}}
\caption{Points needed to meet Class Requirements for
\race{Gnome}:}\tabularnewline
\toprule
Class Name & STR & INT & PIE & VIT & DEX & SPE & PER & F? &
Total\tabularnewline
\midrule
\endfirsthead
\toprule
Class Name & STR & INT & PIE & VIT & DEX & SPE & PER & F? &
Total\tabularnewline
\midrule
\endhead
Fighter & 2 & & & & & & & N & 2\tabularnewline
Mage & & 5 & & & & & & - & 5\tabularnewline
Priest & & & & & & & & - & 0\tabularnewline
Thief & & & & 4 & 2 & & & - & 6\tabularnewline
Ranger & & 1 & & 1 & 2 & 2 & 2 & N & 8\tabularnewline
Bard & & 3 & & & 4 & 2 & 7 & Y & 16\tabularnewline
Psionic & & 7 & & 4 & & & 4 & N & 15\tabularnewline
Alchemist & & 6 & & & 5 & & & - & 11\tabularnewline
Valkyrie & 2 & & & 2 & 1 & 5 & 1 & Y & 11\tabularnewline
Bishop & & 8 & 2 & & & & 1 & Y & 11\tabularnewline
Lord & 2 & 2 & & 2 & 1 & 3 & 8 & N & 18\tabularnewline
Samurai & 2 & 4 & & 1 & 4 & 8 & 2 & N & 21\tabularnewline
Monk & 3 & 1 & & & 2 & 7 & 2 & N & 15\tabularnewline
Ninja & 2 & 3 & & 2 & 4 & 6 & & N & 17\tabularnewline
\bottomrule
\end{longtable}

\begin{twocolumnitemize}{Convenient starting professions:}
\item
  \class{Fighter}
\item
  \class{Priest}
\item
  \class{Ranger}
\item
  \class{Valkyrie}
\end{twocolumnitemize}

\pagebreak[1]\subsection{\race{Hobbit}}\label{hobbit}%
%
Base \race{Hobbit} stats 8 7 6 9 10 7 13

\begin{multicols}{2}\WviiTwoColumnSetup
\WviiNeedSpace{}Perks:%
\begin{itemize}
\tightlist
\item
  Magic Resistance (i.e.\ to spells like \spell{Lifesteal} or \spell{Make
  Wounds})
\item
  Natural thieving characters
\end{itemize}
\columnbreak

\WviiNeedSpace{}Flaws:%
\begin{itemize}
\tightlist
\item
  Poor Intelligence and Piety makes them poor spellcasters. Don't rely
  on any spells they may use on coming back quickly.
\end{itemize}
\end{multicols}

\begin{longtable}[]{@{}lccccccccr@{}}
\caption{Points needed to meet Class Requirements for
\race{Hobbit}:}\tabularnewline
\toprule
Class Name & STR & INT & PIE & VIT & DEX & SPE & PER & F? &
Total\tabularnewline
\midrule
\endfirsthead
\toprule
Class Name & STR & INT & PIE & VIT & DEX & SPE & PER & F? &
Total\tabularnewline
\midrule
\endhead
Fighter & 4 & & & & & & & N & 4\tabularnewline
Mage & & 5 & & & & & & - & 5\tabularnewline
Priest & & & 6 & & & & & - & 6\tabularnewline
Thief & & & & & 2 & 1 & & - & 3\tabularnewline
Ranger & 2 & 1 & 2 & 2 & & 1 & & N & 8\tabularnewline
Bard & & 3 & & & 2 & 1 & & Y & 6\tabularnewline
Psionic & 2 & 7 & & 5 & & & 1 & N & 15\tabularnewline
Alchemist & & 6 & & & 3 & & & - & 9\tabularnewline
Valkyrie & 4 & & 5 & 2 & & 4 & & Y & 15\tabularnewline
Bishop & & 8 & 9 & & & & & - & 17\tabularnewline
Lord & 4 & 2 & 6 & 3 & & 2 & 1 & N & 18\tabularnewline
Samurai & 4 & 4 & & & 2 & 7 & & N & 17\tabularnewline
Monk & 5 & 1 & 7 & & & 6 & & N & 19\tabularnewline
Ninja & 4 & 3 & 4 & 3 & 2 & 5 & & N & 21\tabularnewline
\bottomrule
\end{longtable}

\begin{twocolumnitemize}{Convenient starting professions:}
\item
  \class{Thief}
\item
  \class{Bard}
\end{twocolumnitemize}

\pagebreak[1]\subsection{\race{Faerie}}\label{faerie}%
%
Base \race{Faerie} Stats 5 11 6 6 10 14 12

\begin{multicols}{2}\WviiTwoColumnSetup
\WviiNeedSpace{}Perks:%
\begin{itemize}
\item
  Character always start with Gossamer Gown (U), Gossamer Gown (L),
  Faerie Stick, Faerie Dust(5), and Lt. Heal(3) regardless of class.
\item
  Lower AC than other races
\item
  Magic Resistance (i.e.\ to spells like \spell{Lifesteal} or \spell{Make Wounds})
\item
  Spell points return at a very quick rate.
\item
  Natural spellcaster
\item
  Race specific helmet (Faerie Cap)
\item
  Race \& class specific weapon (Cane of Corpus for the \race{Faerie} \class{Ninja})
\end{itemize}
\columnbreak

\WviiNeedSpace{}Flaws:%
\begin{itemize}
\item
  Cannot use most weapons and armor.
\item
  Consistency of starting equipment prevents \race{Faerie} characters from
  using sometimes irreplaceable equipment such as the Poet's Lute.
\item
  Low carrying capacity
\item
  Tends to have low amounts of hit points.
\end{itemize}
\end{multicols}

\begin{longtable}[]{@{}lccccccccr@{}}
\caption{Points needed to meet Class Requirements for
\race{Faerie}:}\tabularnewline
\toprule
Class Name & STR & INT & PIE & VIT & DEX & SPE & PER & F? &
Total\tabularnewline
\midrule
\endfirsthead
\toprule
Class Name & STR & INT & PIE & VIT & DEX & SPE & PER & F? &
Total\tabularnewline
\midrule
\endhead
Fighter & 7 & & & & & & & N & 7\tabularnewline
Mage & & 1 & & & & & & - & 1\tabularnewline
Priest & & & 6 & & & & & - & 6\tabularnewline
Thief & & & & & 2 & & & - & 2\tabularnewline
Ranger & 5 & & 2 & 5 & & & & N & 12\tabularnewline
Bard & & & & & 2 & & & - & 2\tabularnewline
Psionic & 5 & 3 & & 8 & & & & N & 16\tabularnewline
Alchemist & & 2 & & & 3 & & & - & 5\tabularnewline
Valkyrie & 7 & & 5 & 5 & & & & Y & 17\tabularnewline
Bishop & & 4 & 9 & & & & & - & 13\tabularnewline
Lord & 7 & & 6 & 6 & & & 2 & N & 21\tabularnewline
Samurai & 7 & & & 3 & 2 & & & N & 12\tabularnewline
Monk & 8 & & 7 & & & & & N & 15\tabularnewline
Ninja & 7 & & 4 & 6 & 2 & & & N & 19\tabularnewline
\bottomrule
\end{longtable}

\begin{twocolumnitemize}{Convenient starting professions:}
\item
  \class{Mage}
\item
  \class{Priest} -- Most extended weapons available for \race{Faerie}s are
  \class{Priest}\fshyp{}\class{Bishop}.
\item
  \class{Thief}
\item
  \class{Alchemist}
\item
  \class{Bard} -- A pointless choice after playing as you do not get the Poet's
  Lute.
\item
  \class{Samurai} -- A semi-useless choice as you cannot wear most decent armor.
\end{twocolumnitemize}

\pagebreak[1]\subsection{\race{Lizardman}}\label{lizardman}%
%
Base \race{Lizardman} Stats 12 5 5 14 8 10 3

\begin{multicols}{2}\WviiTwoColumnSetup
\WviiNeedSpace{}Perks:%
\begin{itemize}
\item
  Mental Resistance
\item
  Acid Resistance
\item
  Greater than normal amount of hit points
\end{itemize}
\columnbreak

\WviiNeedSpace{}Flaws:%
\begin{itemize}
\item
  Very slow mana recharge rate
\item
  Academia skills tend to not mature
\item
  Intelligence and Piety go \textlcsc{DOWN} on level ups if too high.
\end{itemize}
\end{multicols}

\begin{longtable}[]{@{}lccccccccr@{}}
\caption{Points needed to meet Class Requirements for
\race{Lizardman}:}\tabularnewline
\toprule
Class Name & STR & INT & PIE & VIT & DEX & SPE & PER & F? &
Total\tabularnewline
\midrule
\endfirsthead
\toprule
Class Name & STR & INT & PIE & VIT & DEX & SPE & PER & F? &
Total\tabularnewline
\midrule
\endhead
Fighter & & & & & & & & N & 0\tabularnewline
Mage & & 7 & & & & & & - & 7\tabularnewline
Priest & & & 7 & & & & & - & 7\tabularnewline
Thief & & & & & 4 & & & - & 4\tabularnewline
Ranger & & 3 & 3 & & 2 & & 4 & Y & 12\tabularnewline
Bard & & 5 & & & 4 & & 8 & Y & 17\tabularnewline
Psionic & & 9 & & & & & 6 & Y & 15\tabularnewline
Alchemist & & 7 & & & 5 & & & - & 12\tabularnewline
Valkyrie & & & 6 & & 2 & 1 & 4 & Y & 13\tabularnewline
Bishop & & 10 & 10 & & & & 4 & Y & 24\tabularnewline
Lord & & 4 & 7 & & 1 & & 9 & N & 21\tabularnewline
Samurai & & 6 & & & 4 & 4 & 5 & N & 19\tabularnewline
Monk & 1 & 3 & 8 & & 2 & 3 & 5 & N & 22\tabularnewline
Ninja & & 5 & 5 & & 4 & 2 & & N & 16\tabularnewline
\bottomrule
\end{longtable}

\begin{twocolumnitemize}{Convenient starting professions:}
\item
  \class{Fighter}
\item
  \class{Ninja}
\end{twocolumnitemize}

\pagebreak[1]\subsection{\race{Dracon}}\label{dracon}%
%
Base \race{Dracon} Stats 10 7 6 12 10 8 6

\begin{multicols}{2}\WviiTwoColumnSetup
\WviiNeedSpace{}Perks:%
\begin{itemize}
\item
  Breathe Acid
\item
  Acid resistance
\item
  Mental Resistance
\end{itemize}
\columnbreak

\WviiNeedSpace{}Flaws:%
\begin{itemize}
\item
  Breathing Acid takes out a large percentage of stamina, regardless of
  your level.
\item
  Damage of Breath is based on amount of stamina used
\end{itemize}
\end{multicols}

\begin{longtable}[]{@{}lccccccccr@{}}
\caption{Points needed to meet Class Requirements for
\race{Dracon}:}\tabularnewline
\toprule
Class Name & STR & INT & PIE & VIT & DEX & SPE & PER & F? &
Total\tabularnewline
\midrule
\endfirsthead
\toprule
Class Name & STR & INT & PIE & VIT & DEX & SPE & PER & F? &
Total\tabularnewline
\midrule
\endhead
Fighter & 2 & & & & & & & N & 2\tabularnewline
Mage & & 5 & & & & & & - & 5\tabularnewline
Priest & & & 6 & & & & & - & 6\tabularnewline
Thief & & & & & 2 & & & - & 2\tabularnewline
Ranger & 1 & & 2 & & & & 2 & N & 5\tabularnewline
Bard & & 3 & & & 2 & & 5 & Y & 10\tabularnewline
Psionic & & 7 & & 2 & & & 4 & N & 13\tabularnewline
Alchemist & & 6 & & & 3 & & & - & 9\tabularnewline
Valkyrie & 2 & & 5 & & & 3 & 1 & Y & 11\tabularnewline
Bishop & & 8 & 9 & & & & 1 & Y & 11\tabularnewline
Lord & 2 & 2 & 6 & & & 1 & 8 & N & 19\tabularnewline
Samurai & 2 & 4 & & & 2 & 6 & 2 & N & 16\tabularnewline
Monk & 3 & 1 & 7 & & & 5 & 2 & N & 18\tabularnewline
Ninja & 2 & 3 & 4 & 2 & & 4 & & N & 15\tabularnewline
\bottomrule
\end{longtable}

\begin{twocolumnitemize}{Convenient starting professions:}%
\item
  \class{Fighter}
\item
  \class{Thief}
\item
  \class{Ranger}
\item
  \class{Bishop}
\item
  \class{Ninja}
\end{twocolumnitemize}

\pagebreak[1]\subsection{\race{Rawulf}}\label{rawulf}%
%
Base \race{Rawulf} stats: 8 6 12 10 8 8 10

\begin{multicols}{2}\WviiTwoColumnSetup
\WviiNeedSpace{}Perks:%
\begin{itemize}
\item
  Cold resistance
\end{itemize}
\columnbreak

\WviiNeedSpace{}Flaws:%
\begin{itemize}

\item
  None I know of
\end{itemize}
\end{multicols}

\begin{longtable}[]{@{}lccccccccr@{}}
\caption{Points needed to meet Class Requirements for
\race{Rawulf}:}\tabularnewline
\toprule
Class Name & STR & INT & PIE & VIT & DEX & SPE & PER & F? &
Total\tabularnewline
\midrule
\endfirsthead
\toprule
Class Name & STR & INT & PIE & VIT & DEX & SPE & PER & F? &
Total\tabularnewline
\midrule
\endhead
Fighter & 4 & & & & & & & N & 4\tabularnewline
Mage & & 6 & & & & & & - & 6\tabularnewline
Priest & & & & & & & & - & 0\tabularnewline
Thief & & & & & 4 & & & - & 4\tabularnewline
Ranger & 2 & 2 & & 1 & 2 & & & N & 7\tabularnewline
Bard & & 4 & & & 4 & & 1 & Y & 9\tabularnewline
Psionic & 2 & 8 & & 4 & & & & N & 14\tabularnewline
Alchemist & & 7 & & & 5 & & & - & 12\tabularnewline
Valkyrie & 4 & & & 1 & 2 & 3 & & Y & 10\tabularnewline
Bishop & & 9 & 3 & & & & & - & 12\tabularnewline
Lord & 4 & 3 & & 2 & 1 & 1 & 4 & N & 15\tabularnewline
Samurai & 4 & 5 & & & 4 & 6 & & N & 19\tabularnewline
Monk & 5 & 2 & 1 & & 2 & 5 & & N & 15\tabularnewline
Ninja & 4 & 4 & & 2 & 4 & 4 & & N & 18\tabularnewline
\bottomrule
\end{longtable}

\begin{twocolumnitemize}{Convenient starting professions:}
\item
  \class{Priest}
\item
  \class{Bard}
\item
  \class{Valkyrie}
\item
  \class{Lord}
\item
  \class{Monk}
\end{twocolumnitemize}

\pagebreak[1]\subsection{\race{Felpurr}}\label{felpurr}%
%
Base \race{Felpurr} Stats: 7 10 7 7 10 12 10

\begin{multicols}{2}\WviiTwoColumnSetup
\WviiNeedSpace{}Perks:%
\begin{itemize}
\item
  Dodge missile attacks
\item
  Dodge spells
\end{itemize}
\columnbreak

\WviiNeedSpace{}Flaws:%
\begin{itemize}
\tightlist
\item
  None that I can tell
\end{itemize}
\end{multicols}

\begin{longtable}[]{@{}lccccccccr@{}}
\caption{Points needed to meet Class Requirements for
\race{Felpurr}:}\tabularnewline
\toprule
Class Name & STR & INT & PIE & VIT & DEX & SPE & PER & F? &
Total\tabularnewline
\midrule
\endfirsthead
\toprule
Class Name & STR & INT & PIE & VIT & DEX & SPE & PER & F? &
Total\tabularnewline
\midrule
\endhead
Fighter & 5 & & & & & & & N & 5\tabularnewline
Mage & & 2 & & & & & & - & 2\tabularnewline
Priest & & & 5 & & & & & - & 5\tabularnewline
Thief & & & & & 2 & & & - & 2\tabularnewline
Ranger & 3 & & 1 & 4 & & & & N & 8\tabularnewline
Bard & & & & & 2 & & 1 & Y & 3\tabularnewline
Psionic & 3 & 4 & & 7 & & & & N & 14\tabularnewline
Alchemist & & 6 & & & 5 & & & - & 11\tabularnewline
Valkyrie & 5 & & 4 & 4 & & & & Y & 13\tabularnewline
Bishop & & 5 & 8 & & & & & - & 13\tabularnewline
Lord & 5 & & 5 & 5 & & & 4 & N & 19\tabularnewline
Samurai & 5 & 1 & & 2 & 2 & 2 & & N & 12\tabularnewline
Monk & 6 & & 6 & & & 1 & & N & 13\tabularnewline
Ninja & 5 & & 3 & 5 & 2 & & & N & 15\tabularnewline
\bottomrule
\end{longtable}

\begin{twocolumnitemize}{Convenient starting professions:}
\item
  \class{Mage}
\item
  \class{Thief}
\item
  \class{Bard}
\item
  \class{Samurai}
\item
  \class{Monk}
\item
  \class{Ninja}
\end{twocolumnitemize}

\pagebreak[1]\subsection{\race{Mook}}\label{mook}%
%
Base \race{Mook} Stats: 10 10 6 10 7 7 9

\begin{multicols}{2}\WviiTwoColumnSetup
\WviiNeedSpace{}Perks:%
\begin{itemize}
\item
  Cold Resistance
\item
  Magic Resistance (i.e.\ to spells like \spell{Lifesteal} or \spell{Make Wounds})
\end{itemize}
\columnbreak

\WviiNeedSpace{}Flaws:%
\begin{itemize}
\tightlist
\item
  None that I can see
\end{itemize}
\end{multicols}

\begin{longtable}[]{@{}lccccccccr@{}}
\caption{Points needed to meet Class Requirements for
\race{Mook}:}\tabularnewline
\toprule
Class Name & STR & INT & PIE & VIT & DEX & SPE & PER & F? &
Total\tabularnewline
\midrule
\endfirsthead
\toprule
Class Name & STR & INT & PIE & VIT & DEX & SPE & PER & F? &
Total\tabularnewline
\midrule
\endhead
Fighter & 2 & & & & & & & N & 2\tabularnewline
Mage & & 2 & & & & & & - & 2\tabularnewline
Priest & & & 6 & & & & & - & 6\tabularnewline
Thief & & & & & 5 & 1 & & - & 6\tabularnewline
Ranger & & & 2 & 1 & 3 & 1 & & N & 7\tabularnewline
Bard & & & & & 5 & 1 & 2 & Y & 8\tabularnewline
Psionic & & 4 & & 5 & & & 1 & N & 10\tabularnewline
Alchemist & & 6 & & & 5 & & & N & 11\tabularnewline
Valkyrie & 2 & & 5 & 1 & 3 & 4 & & Y & 15\tabularnewline
Bishop & & 5 & 9 & & & & & - & 14\tabularnewline
Lord & 2 & & 6 & 2 & 2 & 2 & 5 & N & 19\tabularnewline
Samurai & 2 & 1 & & & 5 & 7 & & N & 15\tabularnewline
Monk & 3 & 7 & & & 3 & 6 & & N & 19\tabularnewline
Ninja & 2 & & 4 & 2 & 5 & 5 & & N & 18\tabularnewline
\bottomrule
\end{longtable}

\begin{twocolumnitemize}{Convenient starting professions:}
\item
  \class{Fighter}
\item
  \class{Mage}
\item
  \class{Psionic}
\end{twocolumnitemize}

\pagebreak[1]\section{Classes}\label{classes}%
%
\nopagebreak\subsection{\class{Fighter}}\label{fighter}%
%
requirements: 12 0 0 0 0 0 0

\begin{multicols}{2}\WviiTwoColumnSetup
\WviiNeedSpace{}Perks:%
\begin{itemize}
\item
  Starting kit is very advanced compared to other characters
\item
  Gain levels quickly compared to most classes
\item
  May use a majority of weapons and the strongest armor available
\item
  Greater than normal amount of hit points
\end{itemize}
\columnbreak

\WviiNeedSpace{}Flaws:%
\begin{itemize}
\item
  No spells
\item
  Heavy armor weighs a \textlcsc{LOT}.
\item
  Lessened skill points for Academia
\end{itemize}
\end{multicols}

\begin{twocolumnitemize}{Profession changing suggestions:}
\item
  \class{Priest}→\class{Valkyrie}
\item
  \class{Mage}→\class{Samurai}
\item
  \class{Thief}→\class{Alchemist}→\class{Ninja}
\item
  \class{Ranger}→\class{Ninja}
\item
  \class{Valkyrie} (females only)
\item
  \class{Lord}
\item
  \class{Samurai}
\end{twocolumnitemize}

\pagebreak[1]\subsection{\class{Mage}}\label{mage}%
%
requirements: 0 12 0 0 0 0 0

\WviiNeedSpace{}%
\begin{multicols}{2}\WviiTwoColumnSetup
\WviiNeedSpace{}Perks:%
\begin{itemize}
\item
  Learns the Thaumaturgic spells the quickest.
\item
  Increased skill points for Academia
\end{itemize}
\columnbreak

\WviiNeedSpace{}Flaws:%
\begin{itemize}
\item
  Low hit points
\item
  No healing
\item
  Doesn't start with a ranged weapon
\item
  Lessened skill points for Weapon
\end{itemize}
\end{multicols}

\begin{twocolumnitemize}{Profession changing suggestions:}
\item
  \class{Fighter}→\class{Samurai}
\item
  \class{Priest}→\class{Bishop}
\item
  \class{Thief}→\class{Alchemist}
\item
  \class{Psionic}→\class{Monk}
\item
  \class{Bishop}→\class{Psionic}
\item
  \class{Bard}
\item
  \class{Samurai}
\end{twocolumnitemize}

\pagebreak[1]\subsection{\class{Priest}}\label{priest}%
%
requirements: 0 0 12 0 0 0 0

\begin{multicols}{2}\WviiTwoColumnSetup
\WviiNeedSpace{}Perks:%
\begin{itemize}
\item
  Learns the Theological spells the fastest
\item
  Healer
\item
  Starts with healing potions
\item
  Starts with a ranged weapon if not \race{Faerie}
\item
  Automatically has the Diplomat skill
\item
  Increased skill points for Academia
\end{itemize}
\columnbreak

\WviiNeedSpace{}Flaws:%
\begin{itemize}
\item
  Few attacking spells
\item
  Poor armor selection
\item
  Lessened skill points for weapon
\end{itemize}
\end{multicols}

\begin{twocolumnitemize}{Profession changing suggestions:}%
\item
  \class{Fighter}→\class{Valkyrie} (females only)\fshyp{}\class{Lord}
\item
  \class{Mage}→\class{Bishop}
\item
  \class{Bishop}→\class{Valkyrie} (females
  only)\fshyp{}\class{Lord}\fshyp{}\class{Psionic}
\item
  \class{Psionic}→\class{Monk}
\item
  \class{Valkyrie} (females only)
\item
  \class{Lord}
\end{twocolumnitemize}

\pagebreak[1]\subsection{\class{Thief}}\label{thief}%
%
requirements: 0 0 0 0 12 8 0

\begin{multicols}{2}\WviiTwoColumnSetup
\WviiNeedSpace{}Perks:%
\begin{itemize}
\item
  Gets a \textlcsc{LARGE} boost in Skulduggery every level for a good while.
\item
  Class specific weapon (Thieves Dagger) that is cursed, allowing use as
  a secondary weapon for thieves that change professions.
\item
  Learns Ninjutsu
\item
  Learns Legerdemain
\item
  Learns Skulduggery
\item
  Increased skill points to Physical
\end{itemize}
\columnbreak

\WviiNeedSpace{}Flaws:%
\begin{itemize}
\item
  No spells
\item
  Class specific weapon is weak compared to other class specific weapons.
\item
  Becomes outdated after mid-game as other classes get better kit.
\item
  Poor armor selection
\item
  Limited weapon selection
\end{itemize}
\end{multicols}

\begin{twocolumnitemize}{Profession changing suggestions:}
\item
  \class{Ranger}→\class{Ninja}
\item
  \class{Bard}→\class{Samurai}
\item
  \class{Alchemist}→\class{Ninja}
\item
  \class{Samurai}
\item
  \class{Monk}
\item
  \class{Ninja}
\end{twocolumnitemize}

\pagebreak[1]\subsection{\class{Ranger}}\label{ranger}%
%
requirements: 10 8 8 11 10 8 8

\begin{multicols}{2}\WviiTwoColumnSetup
\WviiNeedSpace{}Perks:%
\begin{itemize}
\item
  Kirijutsu effect with missile weapons
\item
  Class specific weapon and accessory (Estoc de~Olivia, Forest Cape)
\item
  Learns Alchemical spells
\item
  Cannot be silenced
\item
  Learns Skulduggery
\item
  Learns Legerdemain
\item
  Learns Ninjutsu
\item
  Limited Healing capabilities (\spell{Heal Wounds})
\end{itemize}
\columnbreak

\WviiNeedSpace{}Flaws:%
\begin{itemize}
\item
  Second rate protection
\item
  Bows run out of ammo
\item
  Scouting is better suited for \class{Fighter}s
\item
  Limited weapon selection
\end{itemize}
\end{multicols}

\begin{twocolumnitemize}{Profession changing suggestions:}
\item
  \class{Bard}→\class{Samurai}
\item
  \class{Alchemist}→\class{Ninja}
\item
  \class{Psionic}→\class{Monk}
\item
  \class{Ninja}
\end{twocolumnitemize}

\pagebreak[1]\subsection{\class{Bard}}\label{bard}%
%
requirements: 0 10 0 0 12 8 12

\begin{multicols}{2}\WviiTwoColumnSetup
\WviiNeedSpace{}Perks:%
\begin{itemize}
\item
  Learns Thamaturgical spells
\item
  Learns Ninjutsu
\item
  Can play musical instruments
\item
  Learns Skulduggery
\item
  Learns Legerdemain
\item
  Starts with a Poet's Lute (unless character is a \race{Faerie})
\item
  Increased skill points to Physical
\end{itemize}
\columnbreak

\WviiNeedSpace{}Flaws:%
\begin{itemize}
\item
  Must be a female character to use the best protection available to the
  class
\item
  Limited weapon selection
\end{itemize}
\end{multicols}

\begin{twocolumnitemize}{Profession changing suggestions:}
\item
  \class{Mage}
\item
  \class{Thief}
\item
  \class{Samurai}
\end{twocolumnitemize}

\pagebreak[1]\subsection{\class{Psionic}}\label{psionic}%
%
requirements: 10 14 0 14 0 0 10

\begin{multicols}{2}\WviiTwoColumnSetup
\WviiNeedSpace{}Perks:%
\begin{itemize}
\item
  Learns Theosophical spells the fastest
\item
  Class-specific accessory (PK Crystal)
\item
  Starts with Shadow Cloak (unless character is a \race{Faerie})
\item
  Limited Healing capabilities (\spell{Heal Wounds}, \spell{Lifesteal})
\item
  Increased skill points to Academia
\end{itemize}
\columnbreak

\WviiNeedSpace{}Flaws:%
\begin{itemize}
\item
  Many monsters resist Mental attacks more often than other attacks,
  making most of the spells available to the \class{Psionic} useless.
\item
  Shadow Cloak useless if no \class{Thief}, \class{Ninja}, or
  \class{Bard} in party
\end{itemize}
\end{multicols}

\begin{twocolumnitemize}{Profession changing suggestions:}
\item
  \class{Valkyrie} (females only)
\item
  \class{Bishop}
\item
  \class{Lord}
\end{twocolumnitemize}

\pagebreak[1]\subsection{\class{Alchemist}}\label{alchemist}%
%
requirements: 0 13 0 0 13 0 0

\begin{multicols}{2}\WviiTwoColumnSetup
Perks:%
\begin{itemize}
\item
  Learns Alchemical spells the fastest
\item
  Cannot be silenced
\item
  Starts with Cherry Bomb (unless starting character is a \race{Faerie})
\item
  Class specific accessory (Medicine Bag)
\item
  Increased skill points to Academia
\item
  Limited Healing capabilities (\spell{Heal Wounds})
\end{itemize}
\columnbreak

\WviiNeedSpace{}Flaws:%
\begin{itemize}
\tightlist
\item
  Class specific accessory\ldots{}stinks and is only found near the end
  of the game.
\end{itemize}
\end{multicols}

\begin{twocolumnitemize}{Profession changing suggestions:}
\item
  \class{Thief}
\item
  \class{Ranger}→\class{Ninja}
\item
  \class{Bard}→\class{Samurai}
\item
  \class{Ninja}
\end{twocolumnitemize}


\pagebreak[1]\subsection{\class{Valkyrie}}\label{valkyrie}%
%
requirements: 10 0 11 11 10 11 8

\begin{multicols}{2}\WviiTwoColumnSetup
\WviiNeedSpace{}Perks:%
\begin{itemize}
\item
  Learns Theological spells
\item
  Class-specific weapon (Maenad's Lance)
\item
  May use a majority of weapons and the strongest armor available
\item
  Anything a \class{Lord} can do, a \class{Valkyrie} can do better
\item
  Healer
\item
  Greater than normal amount of hit points
\item
  \class{Valkyrie}s require less experience to gain a level
\end{itemize}
\columnbreak

\WviiNeedSpace{}Flaws:%
\begin{itemize}
\item
  Available to only female characters
\item
  Heavy armor weighs a \textlcsc{LOT}.
\end{itemize}
\end{multicols}

\begin{twocolumnitemize}{Profession changing suggestions:}
\item
  \class{Samurai}
\item
  \class{Monk}
\item
  \class{Ninja}
\end{twocolumnitemize}

\pagebreak[1]\subsection{\class{Bishop}}\label{bishop}%
%
requirements: 0 15 15 0 0 0 8

\begin{multicols}{2}\WviiTwoColumnSetup
\WviiNeedSpace{}Perks:%
\begin{itemize}
\item
  Learns both Theological and Thaumaturgical spells
\item
  Learns Diplomacy
\item
  Can wear better armor than the \class{Priest}
\item
  Healer
\item
  Greatly increased skill points to Academia
\end{itemize}
\columnbreak

\WviiNeedSpace{}Flaws:%
\begin{itemize}
\tightlist
\item
  Spell progression is \textlcsc{SLOW} as research must be divided in half.
\end{itemize}
\end{multicols}

\begin{twocolumnitemize}{Profession changing suggestions:}
\item
  \class{Psionic}
\item
  \class{Alchemist}
\end{twocolumnitemize}

\pagebreak[1]\subsection{\class{Lord}}\label{lord}%
%
requirements: 12 9 12 12 9 9 14

\begin{multicols}{2}\WviiTwoColumnSetup
\WviiNeedSpace{}Perks:%
\begin{itemize}
\item
  Learns Theological spells
\item
  Learns Diplomacy
\item
  May use a majority of weapons and the strongest armor available
\item
  Starts with very good equipment
\item
  Greater than normal amount of hit points
\item
  Increased skill points to Weapon
\end{itemize}
\columnbreak

\WviiNeedSpace{}Flaws:%
\begin{itemize}
\item
  Very expensive class to attain
\item
  Very expensive experience requirements
\item
  Most things a \class{Lord} can do, a \class{Valkyrie} can do better
\item
  Heavy armor weighs a \textlcsc{LOT}.
\item
  \class{Valkyrie}s have smaller experience requirements.
\end{itemize}
\end{multicols}

\begin{twocolumnitemize}{Profession changing suggestions:}
\item
  \class{Priest}
\item
  \class{Ranger}
\item
  \class{Bard}
\item
  \class{Samurai}
\item
  \class{Monk}
\item
  \class{Ninja}
\end{twocolumnitemize}

\pagebreak[1]\subsection{\class{Samurai}}\label{samurai}%
%
Requirements: 12 11 0 9 12 14 8

\begin{multicols}{2}\WviiTwoColumnSetup
\WviiNeedSpace{}Perks:%
\begin{itemize}
\item
  Learns Kirijutsu
\item
  Learns Thaumaturgical spells
\item
  Class specific weapons and armor (Muramasa Blade, Do-Maru~(U) and (L),
  Tosei-Do~(U) \& (L), Hi-Kane-Do~(U) \& (L), Kabuto)
\item
  Increased skill points to Weapon
\end{itemize}
\columnbreak

\WviiNeedSpace{}Flaws:%
\begin{itemize}
\item
  Semi-limited selection of weapons and armor
\item
  Expensive class to attain and maintain
\end{itemize}
\end{multicols}

\begin{twocolumnitemize}{Profession changing suggestions:}
\item
  \class{Valkyrie} (female)
\item
  \class{Monk}
\end{twocolumnitemize}

\pagebreak[1]\subsection{\class{Monk}}\label{monk}%
%
Requirements: 13 8 13 0 10 13 8

\begin{multicols}{2}\WviiTwoColumnSetup
\WviiNeedSpace{}Perks:%
\begin{itemize}
\item
  Learns Kirijutsu
\item
  Learns Theosophical spells
\item
  Learns Ninjutsu
\item
  Learns Hands and Feet
\item
  AC benefits from Ninjutsu
\item
  Limited Healing capabilities (\spell{Heal Wounds}, \spell{Lifesteal})
\end{itemize}
\columnbreak

\WviiNeedSpace{}Flaws:%
\begin{itemize}
\item
  Theosophical spells resisted more often
\item
  Expensive class to attain
\item
  Limited weapon selection
\item
  Armor restricted to robes
\end{itemize}
\end{multicols}

\begin{twocolumnitemize}{Profession changing suggestions:}
\item
  \class{Ninja}
\end{twocolumnitemize}

\pagebreak[1]\subsection{\class{Ninja}}\label{ninja}%
%
Requirements: 12 10 10 12 12 12 0

\begin{multicols}{2}\WviiTwoColumnSetup
\WviiNeedSpace{}Perks:%
\begin{itemize}
\item
  Learns Kirijutsu
\item
  Learns Alchemical spells
\item
  Cannot be silenced
\item
  Learns Ninjutsu
\item
  Learns Hands and Feet
\item
  AC benefits from Ninjutsu
\item
  Learns Skulduggery
\item
  Learns Legerdemain
\item
  Limited Healing capabilities (\spell{Heal Wounds})
\item
  Weapons \class{Ninja}s can use are useful.
\item
  Class specific weapons, armor, and accessory (Sai, Nunchuku, Ninjato,
  Ninja Cowl, Ninja Garb~(U), Ninja Garb~(L), Tabi Boots, Blackbelt of 5
  Flowers)
\end{itemize}
\columnbreak

\WviiNeedSpace{}Flaws:%
\begin{itemize}
\item
  Expensive class to attain
\item
  Very expensive experience requirements
\item
  Slightly decreased skill points to Academia
\item
  Usage as a `Jack of all Trades' creates difficulties mid-game.
\item
  Semi-limited weapon selection
\item
  May only equip Ninja Cowl, Ninja Garb~(U) \& (L), Tabi Boots, and race
  specific items for armor.
\end{itemize}
\end{multicols}

\begin{twocolumnitemize}{Profession changing suggestions:}
\item
  None
\end{twocolumnitemize}

\section{Skill selection}\label{skill-selection}%
%
\subsection{Essential Skills}\label{essential-skills}%
%
\paragraph{Swimming} Everyone needs to have at least a 10 in swimming before
attempting to enter the water to practice. There are three places to practice
without wasting too much time: \place{New City} (using the fountain in the
\place{Starter Dungeon} or the one at \npc{Father Rulae}), the
\monster{Ra-sep-re-tep} pool in the \place{Starter Dungeon} (using the
fountain in the same area) and \place{Munkharama} (at the Polar Munk
Society\ldots{} bring healing though\ldots{}). Get your swimming to at least
a 20 using these three places---the \place{Lost Temple} has a fountain that
is in the water, but you need to have at least a 20 swimming to reach
it. Once you do, you can practice until you have a 100 in it.

\paragraph{Climbing} In the beginning of the game, you won't need
this. However, once you get to the mountains in Guardia you're going to need
at least a 30 in climbing skill. Don't say I didn't warn
you\ldots{}\spell{Levitate} can temporarily boost your climbing skill when
attempting to climb by the way.

\paragraph{Scouting} If this is your first time through, you need
scouting. Without it, you won't be able to pass several areas, as you won't
know to search an area On second and subsequent times through the game
however, it is safe to skip this if you remember where everything is.

\paragraph{Skulduggery} When starting, don't settle for less than at least a
5 or 6 in Skulduggery on a character that is going to be your lockpicker. At
least, start with that amount before you try to disarm a chest. Chests are
\textlcsc{HARD} to disarm. You will be Quit/No Save-Restoring very
often---having an actual \class{Thief} in the party instead of a \class{Bard},
\class{Ninja}, or \class{Ranger} substitute can eliminate this problem after
a couple level ups, as they gain a nice bonus to Skulduggery every level up
to a certain point.

\paragraph{Mapping} You need a minimum of 10 in Mapping to map walls, a
minimum of 30 to map 1$\times$1 rooms, and a minimum of 60 to map stairs and
pits. Some teleporters aren't mapped, as far as I can tell; only the ones
that use pits are. In some areas of the game, the inability to map will leave
you stuck or moving at a snail's crawl if you do not know the area,
especially in the darkened areas where the only thing you \textlcsc{CAN} map
are the walls.

\paragraph{Theosophy\fshyp{}Theology\fshyp{}Thaumaturgy\fshyp{}Alchemy}
Essential for anyone that learns them. Until it reaches 95--100, if you don't
put points into this, you're potentially missing out on a spell that might
help you in game.

\subsection{\texorpdfstring{``Nice to have''
skills}{Nice to have skills}}\label{nice-to-have-skills}%
%
\paragraph{Artifacts} Having a high Artifacts skill lets you assay items to
find out what they are without casting \spell{Identify}. Don't worry about
putting points into it however; you automatically gain points every time you
successfully use (not equip and throw) a wand in battle as an item.  This,
however, doesn't eliminate the need for \spell{Identify} unless you already
know what an item does.

\paragraph{Ninjutsu} \class{Monk}s and \class{Ninja}s gain an AC bonus from
this, in addition to anything else that other Ninjutsu practitioners
get. They should build this up all the time by practicing hiding instead of
putting precious skill points into it. Anyone else with this simply gains a
nice ability to hide and do a backstab or surprise attack.

\paragraph{Legerdemain} The 5 Fingered Discount is a nice thing to have, but
you won't pilfer much. The chance of getting caught is quite high as well,
even with a level~100 Legerdemain, so save before trying. If the character's
inventory slots are full only gold will be taken, which keeps this skill from
being totally useless; most shopkeepers have horrible items in their
inventory---you wouldn't want to steal them if your life depended on
it. Legerdemain is best when the pickpocket using it is rather high level, is
decent at it (has at least a 30 so in Legerdemain), and pickpocketing someone
worthwhile like \npc{Belcanzor} for his items or \npc{Dame Ke-Li} for her Ankhs. Be
especially careful pickpocketing wandering NPCs. Some like \npc{Capt.~Boerigard}
have very nice weapons and such, but they may attack you if they catch you
pickpocketing them, which is a bad thing.

\paragraph{Kirijutsu} This skill gives you a chance to get a Critical Hit
that instantly kills a monster. Very nice to have, but try not to get the
person with it confused!

\paragraph{Mind Control} A personal skill that you can get in
\place{Dionysceus}; it reduces the effectiveness of psionics and other mental
spells. A minor side effect is that you will never fall asleep in the poppy
field with even 1 point in this skill.

\paragraph{Power Strike} I don't know about you, but I \textlcsc{like} hitting
for more damage. \verb|:)|

\paragraph{Diplomacy} Diplomacy is kind of a unused skill in the game---most
of the time. However, when used it can be really useful, when an NPC is very
angry with you and you need to calm them down (and maybe get friendly
again). The fact that characters with a Personality of 14 can automatically
get it, \textlcsc{AND} the fact that anyone who successfully ``pleases'' an
NPC automatically gets a one point raise in the skill regardless of whether
they have it or not however makes the skill kind of a ``I'll get it when I
need it'' skill. Warning: Diplomacy is a useless skill in the Wizardry Gold
version of Wizardry~7. No matter how pleased you make an NPC, they will never
trade with you until you please the NPC by doing things for them (i.e.\ doing
\npc{Barlone} a favor will please \npc{Mick the Pick} and \npc{Ratsputin}
enough to trade with you).

\subsection{\texorpdfstring{``Really depends on your playing style''
skills}{Really depends on your playing style skills}}\label{really-depends-on-your-playing-style-skills}%
%
\paragraph{Mythology} This lets you successfully ID the monster you're
fighting. In some cases it will be very useful, if you're conserving mana and
don't want to mistake \monster{Night Rooks} for \monster{Vampire
  Vultures}. In most cases however, it is easy to figure out what the monster
is after a round, or even based on where it is. Putting points into this is
really up to your strategy, but it goes up naturally over time, albeit
slowly.

\paragraph{Scribe} If you like to save on mana, you may find a use for this
skill. Personally, I've gone through the game with only one point put into
this skill for \spell{Knock-Knock} scrolls to be slightly more
effective---other than \spell{Magic Screen}, \spell{Armorplate}, \spell{Locate
  Person}, and \spell{Enchanted Blade} scrolls, \spell{Knock-Knock} scrolls
are the only other ones I actually use and all of them can (or are supposed
to) be used outside of battle. Some nice spells are on scrolls, but chances
are, you're going to sell them for the cash they give you---by the point you
don't need cash anymore, scrolls are already quite useless.

\paragraph{Firearms} I personally don't like 2 of the 5 weapons that use this
skill. Still, you might as well get it. Raising it is up to you however.

\chapter{Choosing a Party}\label{choosing-a-party}%
%
Depending on what difficulty you are playing, there are five types of
parties that will be used throughout the game: the `Super' Party, the
`Quick and Easy' Party, the `Dual Class' Party, the `Blender' Party, and
the `Vanilla' Party.

\section{\texorpdfstring{The `Super'
Party}{The Super Party}}\label{the-super-party}%
%
The Super Party is what everyone wants to end up with at the end of the game,
but it is very hard to use in the beginning. The reason is because the Super
Party is a party with hard to get characters such as a \race{Faerie}
\class{Samurai}\fshyp{}\class{Monk}\fshyp{}\class{Ninja}, \race{Elf}
\class{Lord}s, or other such hard to get/play classes for example. Because
these guys are hard to get, you don't have a lot of excess points to
distribute for other stats. To top it off, the class usually has almost
double the experience requirements of the easier-to-achieve classes such as
the \class{Fighter}. With all these factors thrown together, this party is
not recommended for a purist new to the game---various maps that hint at what
to do are your only clues as to what should be done if you do not wish to
consult outside sources, and with a Super party it will take too long to
level them up; the NPCs will have snatched them away before you get to them,
and the only ways to get your hands on them are buying it for 10,000
gold---which you won't have for a long time---or killing them, which is very
difficult with this kind of party until much much later.

\WviiNeedSpace{}Sample Super Party:%
\begin{itemize}[topsep=0pt]
\tightlist
\item
  \race{Lizardman} \class{Samurai}
\item
  \race{Faerie} \class{Ninja}
\item
  \race{Elf} \class{Lord}
\item
  \race{Mook} \class{Monk}
\item
  \race{Faerie} \class{Bishop}
\item
  \race{Elf} \class{Valkyrie}
\end{itemize}

\section{\texorpdfstring{The `Quick and Easy'
Party}{The Quick and Easy Party}}\label{the-quick-and-easy-party}%
%
The Quick and Easy Party is the exact opposite of the Super Party. This party
is usually made up of characters quickly rolled up, and is usually something
like 2 \class{Fighter}s, an offensive spellcaster (\class{Mage} or a
\class{Psionic}), a healing spellcaster like a \class{Priest}, a thief, and
maybe a miscellaneous character thrown in for fun. Races are chosen by
whether the race can get a class easier than other races, thereby giving you
more points to distribute elsewhere. Sometimes, the party will be rolled up
with higher scores so that you can distribute the excess points as you see
fit. The game will be easy in the beginning, but after a while you will get
the urge to change profession as experience requirements get higher and
higher and your diversity in what you can do gets lower and lower. Which
brings us to\ldots{}

\WviiNeedSpace{}Sample Quick and Easy Party:%
\begin{itemize}[topsep=0pt]
\tightlist
\item
  \race{Lizardman} \class{Fighter}
\item
  \race{Human} \class{Ranger}
\item
  \race{Faerie} \class{Thief}
\item
  \race{Rawulf} \class{Priest}
\item
  \race{Faerie}\fshyp{}\race{Elf} \class{Mage}
\item
  \race{Mook} \class{Psionic}
\end{itemize}

\section{\texorpdfstring{The `Dual class'
Party}{The Dual class Party}}\label{the-dual-class-party}%
%
This party is similar to the Quick and Easy Party except for one major
difference: The characters rolled up are customized to be able to switch
professions at will. For example, say you have an \race{Elf} who can be a
\class{Ranger} You choose \class{Fighter} instead, add points to meet the
minimum requirements for a \class{Ranger}, and put any excess points where
you want it (usually Vitality in this case, as you will be hurting for hit
points from all the profession changing). The reasoning for this party is to
mainly take advantage of the relatively low amount of experience needed in
the first 5--10 levels to accumulate skills and spells quickly---a Faery
\class{Thief} that I converted into a \class{Ninja} after hitting level~7
already had 100 Skulduggery at level~3!

\WviiNeedSpace{}Sample Dual Class Party (Characters rolled with 15 points to distribute):%
\begin{itemize}
\tightlist
\item
  \race{Elf} \class{Fighter} → Converts to \class{Ranger} at level~10
\item
  \race{Felpurr} \class{Fighter} → Converts to \class{Samurai} at level~10.
\item
  \race{Faerie} \class{Thief} → Converts to \class{Ninja} when stats meet
  minimum \class{Ninja} requirements and Skulduggery \textgreater{} 80.
\item
  \race{Elf} \class{Priest} (F)\protect\footnote{(F) means female} →
  Converts to \class{Valkyrie} when Theology \textgreater{} 80.
\item
  \race{Elf} \class{Mage} (F) → Converts to \class{Bard} when Thaumaturgy
  \textgreater{} 80.
\item
  \race{Elf} \class{Bard} (F) → Converts to \class{Bishop} when
  \class{Mage} converts to \class{Bard}.
\end{itemize}

Assuming that each character gets 15 points to distribute, this is what
would happen normally.

Stats are given in the order of: Strength, Intelligence, Piety,
Vitality, Dexterity, Speed, Personality.

\pagebreak[1]\subsection{\race{Elf} \class{Fighter}}\label{sec:elf-fighter}%
%
\race{Elf} Base stats:%
\begin{itemize}[topsep=0pt]
\tightlist
\item
  7: 5 points added (meet \class{Fighter} requirements)
\item
  10:
\item
  10:
\item
  7: 4 points added (meet \class{Ranger} requirements)
\item
  9: 1 point added (meet \class{Ranger} requirements)
\item
  9:
\item
  8:
\end{itemize}

That leaves you with 5 more points to redistribute; in this case we
distribute the excess to Vitality as all the ability scores will be
reset down to the bare minimum after the profession change; might as
well get some extra hit points and stamina out of it.

\pagebreak[1]\subsection{\race{Felpurr} \class{Fighter}}\label{sec:felpurr-fighter}%
%
\race{Felpurr} Base stats:%
\begin{itemize}[topsep=0pt]
\tightlist
\item
  7: 5 points added (meet \class{Fighter} requirements)
\item
  10: 1 points added (meet \class{Samurai} requirements)
\item
  7:
\item
  7: 2 points added (meet \class{Samurai} requirements)
\item
  10: 2 points added (meet \class{Samurai} requirements)
\item
  12: 2 points added (meet \class{Samurai} requirements)
\item
  10:
\end{itemize}

There are three points to distribute left as you wish. Again, I would
put it into Vitality.

\pagebreak[1]\subsection{\race{Faerie} \class{Thief}}\label{sec:faerie-thief}%
%
\race{Faerie} Base stats:%
\begin{itemize}[topsep=0pt]
\tightlist
\item
  5: 5 points added (almost meet \class{Ninja} requirements)
\item
  11:
\item
  6: 3 points added (almost meet \class{Ninja} requirements)
\item
  6: 5 points added (almost meet \class{Ninja} requirements)
\item
  10: 2 points added (meet \class{Thief} requirements)
\item
  14:
\item
  12:
\end{itemize}

In this case, there are no excess points to use. You will be relying on
level ups to give you the rest of the stat boost you need. This could
easily be met by level~7 or 8--less if you reload your level ups to get
the bonuses you want.

\pagebreak[1]\subsection{\race{Elf} \class{Priest} (F)}\label{sec:elf-priest-f}%
%
\race{Elf} Base stats:%
\begin{itemize}[topsep=0pt]
\tightlist
\item
  7: −2 for being female, 5 points added (meet \class{Valkyrie} requirements)
\item
  10:
\item
  10: 2 points added (meet \class{Priest} requirements)
\item
  7: 4 points added (meet \class{Valkyrie} requirements)
\item
  9: 1 point added (meet \class{Valkyrie} requirements)
\item
  9: 2 points added (meet \class{Valkyrie} requirements)
\item
  8: +1 for being female
\end{itemize}

This leaves you with 1 extra point to distribute. Either Vitality or
Intelligence is a good choice; getting more Academia points to speed up
getting the higher level spells is nice, but so is more hit points.

\pagebreak[1]\subsection{\race{Elf} \class{Mage} (F)}\label{sec:elf-mage-f}%
%
\race{Elf} Base stats:%
\begin{itemize}[topsep=0pt]
\tightlist
\item
  7: −2 for being female
\item
  10: 2 points added (meet \class{Mage} requirements)
\item
  10:
\item
  7:
\item
  9: 3 points added (meet \class{Bard} requirements)
\item
  9:
\item
  8: +1 for being female, 3 points added (meet \class{Bard} requirements)
\end{itemize}

You've got 7 points to redistribute. As \class{Mage}s don't have a hope to gain
a lot of hit points a level, dump them either in Intelligence for
increased amount of Academia points or in Piety for better spell point
gain\fshyp{}level and spell point recovery.

\pagebreak[1]\subsection{\race{Elf} \class{Bard} (F)}\label{sec:elf-bard-f}%
%
\race{Elf} Base stats:%
\begin{itemize}[topsep=0pt]
\tightlist
\item
  7: −2 for being female
\item
  10: 5 points added (meet \class{Bishop} requirements)
\item
  10: 4 points added (almost meet \class{Bishop} requirements)
\item
  7:
\item
  9: 3 points added (meet \class{Bard} requirements)
\item
  9:
\item
  8: +1 for being female, 3 points added (meet \class{Bard} requirements)
\end{itemize}

Again, you have no points to redistribute after that, but like the
\race{Faerie} \class{Thief}, level ups will take care of the rest of the stat
requirements---only quicker in this case as you only need Piety to go up
one.

I think you can see the pattern: If you can afford it, fill out the
requirements of the \nth{2} class first before distributing the points to
other statistics. If you can't, spread out the requirements somewhat so
that it only requires as few level gains as possible to get there. 15 is
the number I use for a minimum when creating profession changing
characters; you might use a lesser amount or a greater amount, depending
on your patience. Whatever you do, if you want to take the greatest
advantage out of changing professions do so by level~10. This is because
it takes the same amount of experience to get from level~1 to level~10
as it does to get from level~10 to level~11.

\section{\texorpdfstring{The `Blender'
Party}{The Blender Party}}\label{the-blender-party}%
%
I call this the Blender party mainly because it takes a little bit of
the Super Party and a little bit of the Quick and Easy Party and/or the
Dual class party. Basically, there will be a couple characters that may
have been chosen because of the items they start with, and others which
were chosen because you like the class or some other reason. Most people
new to the game will probably end up using something like this, where
they have a prize character that they do not wish to change profession
with---especially when the character gets a 19 or 20 in Intelligence or
Personality, which are scores that cannot be improved by certain items
that can be bought, or a 19 or 20 in Vitality, which cannot go over 18
with items.

\WviiNeedSpace{}Sample Blender Party:%
\begin{itemize}
\tightlist
\item
  \race{Felpurr} \class{Samurai}
\item
  \race{Lizardman} \class{Ninja}
\item
  \race{Faerie} \class{Monk}
\item
  \race{Faerie} \class{Thief} → Converts to \class{Ninja} when stats meet
  minimum \class{Ninja} requirements and Skulduggery \textgreater{} 80.
\item
  \race{Elf} \class{Priest} (F) → Converts to \class{Valkyrie} when
  Theology \textgreater{} 80.
\item
  \race{Hobbit} \class{Bard}
\end{itemize}

\section{\texorpdfstring{The `Vanilla'
Party}{The Vanilla Party}}\label{the-vanilla-party}%
%
Called the Vanilla Party for its simpleness, this party \textlcsc{never}
changes classes, so pick carefully. You'll need at least 1~healer,
1~offensive spellcaster, and 1~Skulduggery character unless you're looking
for a challenge. After you're familiar with the game, try going through with
only a single character! It \emph{can} be done. This party is usually done as
a Quick and Easy party, but Super parties that don't class change are done
too.

Sample Vanilla Party: (See Sample Quick and Easy Party or Sample Super
Party)

\chapter{Picking Spells}\label{picking-spells}%
%
Thank you Llevram for much of the information in this section.

\section{General spellcasting}\label{general-spellcasting}%
%
All spells that you can cast are cumulative. That means you can cast them
over and over to make them last longer (if they have a duration), reduce the
duration of all status ailments except disease, stoning and death (in the
case of the ailment cures), or increase the effectiveness of the spell (which
only happens with battle-only spells if at all). The six spells you can cast
at any time (\spell{Enchanted Blade}, \spell{Armorplate}, \spell{Magic
  Screen}, \spell{Detect Secret}, \spell{Direction}, \spell{Levitate}) are
reflected in the globes on top of the game screen and are cumulative in
duration only.

Casting any of the other battle-helping spells (such as \spell{Bless},
\spell{Haste}, etc) outside of battle will have \textlcsc{NO EFFECT}, even if
cast before a fight or some action that causes damage of some sort, so don't
waste your spell points and stamina. Keep in mind that when casting the six
out-of-battle spells repeatedly that it is best done in front of a fountain
so you can restore yourself. Also keep in mind that the longer the spell has
been active, the weaker it gets, so don't rely on multiple castings casted 3
game days ago to get you through certain tougher fights.

\section{Learning spells from
books}\label{learning-spells-from-books}%
%
The following spells may be learned from books:

\begin{itemize}[WviiTwoColumn]
\tightlist
\item
  \spell{Air Pocket}
\item
  \spell{Anti-Magic}
\item
  \spell{Astral Gate}
\item
  \spell{Fire Shield}
\item
  \spell{Ice Shield}
\item
  \spell{Missile Shield}
\item
  \spell{Bless}
\item
  \spell{Charm}
\item
  \spell{Conjuration}
\item
  \spell{Detect Secret}
\item
  \spell{Direction}
\item
  \spell{Haste}
\item
  \spell{Knock-Knock}
\item
  \spell{Levitate}
\item
  \spell{Identify}\protect\footnote{\emph{Anyone} can learn this from a book!}
\item
  \spell{Stamina}
\item
  \spell{Wizard's Eye}
\item
  \spell{Armor Shield}
\item
  \spell{Remove Curse}
\item
  \spell{Watchbells}
\end{itemize}

Be warned however; if you do not have a spell already of the same type as the
one you wish to scribe in your book (for example, having \spell{Shrill Sound}
when you scribe \spell{Missile Shield} in the case of a mage), you will
\textlcsc{NOT} gain any spell points in that category until a spell from the
same category is ``learned'' from a level up. For example, an
\class{Alchemist} scribes \spell{Stamina} instead of learning it the normal
way. The \class{Alchemist} will \textlcsc{NOT} gain any water spell points
until he or she learns another water spell by a level up---in this case
\spell{Cure Paralysis}, \spell{Draining Cloud}, or \spell{Cure Disease}.

\section{Commonly used spells}\label{commonly-used-spells}%
%
\begin{itemize}[WviiTwoColumn]
\tightlist
\item
  \spell{Heal Wounds}
\item
  \spell{Stamina}
\item
  \spell{Dispel Undead}
\item
  \spell{Enchanted Blade}
\item
  \spell{Cure Lesser Condition}
\item
  \spell{Air Pocket}
\item
  \spell{Purify Air}
\item
  \spell{Cure Poison}
\item
  \spell{Cure Paralysis}
\item
  \spell{Silence}
\item
  \spell{Armorplate}
\item
  \spell{Magic Screen}
\item
  \spell{Haste}
\item
  \spell{Mindread}\protect\footnote{Only on your first time through---it
    tells you what to ask the NPC sometimes; otherwise it gives you a bit of
    information you could have gotten with Lore or just gives Nothing.}
\item
  \spell{Fire Shield}\fshyp{}\spell{Ice Shield}
\item
  \spell{Lifesteal}
\item
  \spell{Poison Gas}\fshyp{}\spell{Acid Bomb}\fshyp{}\spell{Firestorm}
\item
  \spell{Conjuration}\fshyp{}\spell{Illusion}\fshyp{}\spell{Create Life}
\item
  \spell{Nuclear Blast}\fshyp{}\spell{Deadly Air}\fshyp{}\spell{Word of
    Death}\fshyp{}\spell{Mind Flay}
\item
  \spell{Asphyxiation}\protect\footnote{Expert mode only for crowd control and
    convenience.}
\end{itemize}

\section{Spell-picking Choices}\label{spell-picking-choices}%
%
\begin{multicols}{2}[\subparagraph{Note:} The required amount of skill points
  for a group of spells may be off by one or two points.]
%
\subsection{Thaumaturgy (\class{Mage} spellbook)}\label{sec:thaumaturgy-mage-spellbook}%
%
\subsubsection{Level 1 spells}%

\spellentry{Energy Blast} Your (possible) first damaging spell. Nothing
special.

\spellentry{Chilling Touch} Your other (possible) first damaging
spell. Nothing special.

\spellentry{Terror} Give the opponent the ``Afraid'' ailment for a short
time. I do not advise using this spell if you are looking for
experience. This becomes useless mid-game if it wasn't useless before
already.

\spellentry{Sleep} Gives the opponent the ``Asleep'' status for a short
time. One of the most useful spells in the game, even if your party already
has a \class{Bard} with the Poet's Lute. This becomes useless after mid-game
however, as you'll have ways of dealing with multiple targets.

\spellentry{Armor Shield} Grants an Armor Class bonus to a
character. Initially, the meager protection offered by the spell will not be
too helpful. Later on however, it is a very useful spell to cast while hiding
(or after you drank an invisibility potion) due to its cheap costs and
noticeable drop in Armor Class when cast at level~7.

\spellentry{Direction} Shows the direction your party is facing. A useless
spell after you get the Journey Map Kit, as you can easily check which way
you are facing by using the map.

\subsubsection{Level 2 spells (18 Thaumaturgy)}%

\spellentry{Magic Missile} Damages a limited number of opponents. Useful when
you need to deal medium amounts of damage to a group.

\spellentry{Shrill Sound} Damages an entire group, regardless of the size of
the group. Very useful when you don't want to waste extra spell points to hit
everything in a group.

\spellentry{Missile Shield} Deflects any attacks that are projectile attacks
(i.e.\ uses ``Throw'' or ``Shoot'' as a form of attack). Chance of being
deflected is related to how high of a spell level you use to cast the spell,
with around a 99\% deflection rate for a level~7 \spell{Missile Shield}.

\spellentry{Knock-Knock} Opens locked chests and doors. Success rate is
related to the power of the spell and the complexity of the lock. Don't
expect to use this spell successfully midgame and onwards.

\spellentry{Detect Secret} Acts like you have a high amount of scouting
skill. Very useful the first time through the game if you don't want to build
up scouting. Very useless subsequent times. When you need to search, the eye
will twinkle; however if you have a weak \spell{Detect Secret}, the eye may
not pick up on some things.

\spellentry{Watchbells} Wakes up everyone who is asleep. The success rate of
the spell depends on how tired the characters are and how long they've been
sleeping. Not very useful after you get the Mind Control Personal
skill---even before then, the spell doesn't really work too well if you're
not already at full stamina.

\spellentry{Weaken} Makes the opponent do less damage when they
attack. Useful I suppose, but I prefer killing them as quickly as
possible. Supposedly this spell also lowers resistances, which can be very
useful, but there's no way to tell for certain.

\subsubsection{Level 3 spells (37 Thaumaturgy)}%

\spellentry{Fireball} Once you get this, you'll be avoiding usage of Magic
Missile except on Undead and fire-resistant monsters.

\spellentry{Fire Shield} Reduces or eliminates the effectiveness of all Fire
Realm spells and Fire based attacks. Don't expect it to eliminate spell
damage to nothing when it is coming from something strong, although Fire Crow
breath will be much less painful.

\spellentry{Ice Shield} Reduces or eliminates the effectiveness of all Water
Realm spells and Ice-based attacks. You'll be happy to have this up when
something decides to cast \spell{Deep Freeze} on you.

\spellentry{Web} Paralyzes one target. Yuck. There are better things out
there to learn; the only monsters you would want to cast this on are
incredibly resistant to it because they are NPCs or the monster equivalent of
one If you can paralyze an NPC with this, you're \textlcsc{WAAAY} above the
NPC's level.

\spellentry{Whipping Rocks} For a group hitting spell, this isn't that
great. A lot of monsters have enough Earth resistance to make this spell
relatively useless---however, when the damage gets through it is all right.

\spellentry{Stink Bomb} Not a very useful spell. It will hit from 1--3 targets
only, which explains the relatively inexpensive casting cost. However,
\spell{Shrill Sound} and \spell{Magic Missile} are better, \textlcsc{AND} you
get them earlier. If you think the added chance of Nausea is worth it though,
go right ahead and get it.

\spellentry{Air Pocket} Deflects \spell{Asphyxiation} spells and reduces the
effectiveness of incoming Air-based spells that stick around as clouds. It
also reduces the effectiveness of breath weapons like Dragonlizard
breath. Don't expect this spell to have any effects on \spell{Whirlwind} or
\spell{Firestorm} however.

\spellentry{Blink} Randomly makes you invisible\fshyp{}uninvisible during the
round. You will always become uninvisible to attack. Lasts quite a while
too. Don't bother with this spell if the character knows Ninjitsu though.

\subsubsection{Level 4 spells (54 Thaumaturgy)}%

\spellentry{Iceball} The Ice equivalent of \spell{Fireball}, only more
expensive. But then, it hurts more, so that's a fair trade off. With
\spell{Fireball} and \spell{Iceball} in your belt, you're probably finished
with mass-target spells for a long while through the game---at least for your
mage.

\spellentry{Magic Screen} Reduces the effectiveness of all incoming spells
and spell-like attacks. That includes breath weapons and fireball-like
blasts, but not little laser beams like the ones the \monster{T'Rang Tecniks}
use to attack. Remember: this can be casted outside of battle.

\spellentry{Conjuration} Summons a monster to help you. They range from the
pathetic Vulture at a level~1 casting to perhaps \monster{Fieros} and
\monster{Myxlmynx} at level 7 (which are pretty darn useful). The best out of
the three summoning spells if you want magical damage.

\spellentry{Armormelt} Makes the opponent easier to hit. Think of it as the
opposite of \spell{Armor Shield}, except it targets everyone.

\spellentry{Crush} Does up to 200 damage to one target. Of course, it misses
so much that it just isn't worth it. That's right, \textlcsc{MISSES}, not
resisted.

\spellentry{Wizard Eye} Gives an overhead view of the area---just like you had
a Journey Map kit. (Surprise\ldots{}) Skip this spell if I were you, unless
you've been skimping on your Mapping skill or don't know if there is a secret
passage behind a wall.

\spellentry{Spooks} A Terror spell that affects all opponents. I don't
recommend it.

\subsubsection{Level 5 spells (72 Thaumaturgy)}%

\spellentry{Prismic Missile} Causes random status ailments or damage. It is
more likely to cause damage on weaker opponents and more likely to cause
status ailments (which rarely includes ``withers and dies'') on more powerful
ones. Keep in mind that this won't work on something with 100\% or greater
Light Resistance, like \monster{Man~O'~Groves}.

\spellentry{Anti-Magic} Tries to prevent a group from casting spells
successfully.  Not too useful when you're fighting multiple groups of
spellcasters as it is better to just off them all right off the bat, but
insanely useful when fighting one group of spellcasters if you want to
conserve on your mass-target spells.

\spellentry{Levitate} Prevents damage if you fall into a pit, and augments
your Climbing skill when you climb. \spell{Levitate} won't help you if you
jump off a cliff however; you'll still die. The power level in this case is
for duration, not damage prevented as a level~1 \spell{Levitate} will protect
you from falling into pits.

\spellentry{Deep Freeze} Learn to love this spell. It is the most
consistently damaging spell in the game---which is why you need \spell{Ice
  Shield} when monsters cast it on you!

\subsubsection{Level 6 spells (90 Thaumaturgy)}%

\spellentry{Firestorm} Fun fun fun. Burns the enemy for a small amount for
several turns. Very useful up to mid-game to kill things---after that it's
useful to add a bit of damage here and there to cut battle time down.

\spellentry{Astral Gate} Instantly kills demons. Since it's specialized,
getting this is up to you. You \emph{will} run into demons however.

\spellentry{Zap Undead} Useless. Don't waste your time with this
spell---anything undead you find is either going to be real easy or real
resistant to this expensive spell.

\spellentry{Recharge} Very useful when used on items that have the effect of
Magicfood when used and can be used multiple times. Otherwise mildly useful
to get back charges on items like the Amulet of Asphyxiation.

\spellentry{Noxious Fumes} Not very useful. Just a better version of
\spell{Stink Bomb} that hits a group. However, there is a much better chance
of nausea than \spell{Stink Bomb}.

\spellentry{Asphyxiation} Wipes out lower level critters with ease. You can
get this decently fast with a mage, so try to grab this spell as fast as
possible if you're on hard mode and the 5 groups of critters that ambush you
each fight start to get on your nerves. Becomes useless near the end of the
game though.

\subsubsection{Level 7 spells (98 Thaumaturgy)}%

\spellentry{Nuclear Blast} Can't go wrong with this spell. Just watch your
Fire spell points.

\spellentry{Ressurection} You can always quit-reload, and unlike
Ressurection, you don't have to wait forever to get it. However, since it can
be used in a fight, it isn't totally useless. However, if you're the
perfectionist type you're probably going to skip this spell. If you ressurect
a character, the character loses 1 point of Vitality and their life counter
increases by~1.

\subsection{Theology (\class{Priest} spellbook)}\label{sec:theology-priest-spellbook}%
%
\subsubsection{Level 1 spells}%

\spellentry{Heal} Wounds Get it if you like it, get it if you don't.

\spellentry{Make Wounds} An attacking spell which uses spell points better
suited for healing? I sure don't like that. You might, but saving the spell
points for healing will save you some nap time.

\spellentry{Stamina} Get it if you like it, get it if you don't.

\spellentry{Bless} Grants you bonuses to hit. Hey, I'd take that any
day. Even later in the game, this spell is useful.

\spellentry{Charm} Paralyzes an opponent. Out of battle, it makes it easier
to make friends with NPCs. If you have diplomatic problems, this may be
useful.  Otherwise, skip for a better spell unless you have no choice.

\subsubsection{Level 2 spells (18 Theology)}%

\spellentry{Enchanted Blade} Helps with hitting and dealing damage,
\textlcsc{AND} you can cast it out of battle. Love this spell.

\spellentry{Dispel Undead} Only works on undead, but there's a lot of undead
in the game. It works more often on the skeleton types of undead than the
ghost types of undead. Keep in mind that any of the ``Spectral'' monsters
like Spectral Ravens or \monster{Spectral Moths} are considered undead as well. Don't
expect it to work too well on some unique undead monsters, with a very
notable exception of the ones in \place{Witch Mountains}.

\spellentry{Cure Lesser Condition} Cures or reduces the duration of the
``Irritation,'' ``Sleep,'' ``Afraid,'' ``Blind,'' and ``Nausea''
ailments. This spell is very useful to cut down on the time you spend
sleeping, and if you run out of Mental spell points it is no big deal, so
there is no need to ration uses of this spell.

\spellentry{Divine Trap} Helps to identify which trap a chest is booby
trapped with.  You will only need it if your Skulduggery character has a low
amount of Skulduggery, or if you are unfamiliar with guessing the traps of
the game.

\spellentry{Identify} Tells you how much damage (if applicable) the item does
and what special effects it may have. If it has a special power, it won't
tell you what it is, but it will mention the that it has one. Special powers
are where when you equip the item, it asks you if you want to invoke its
special power, and generally are very limited in usage. This is best used as
a ``what does this item do'' check, where you cast and reload---however, for
some things (like weapon damage for some weapons) it will give the wrong
numbers.

\spellentry{Slow} Slows down an enemy group. Useful when fighting enemies
stronger than you so that you can try to go first in subsequent rounds. Much
more useful when combined with \spell{Haste} to make sure you go first.

\subsubsection{Level 3 spells (37 Theology)}%

\spellentry{Hold Monsters} Attempts to paralyze a group. Unlike \spell{Web}
or \spell{Paralyze}, the ``Paralyze'' ailment caused by this spell is easily
removed by an attack.

\spellentry{Sane Mind} Cures or reduces the duration of the ``Insanity''
ailment.

\spellentry{Silence} Silences a group of monsters. Don't try it on the
Umpani, most Munks, \monster{Gorn Rangers}, and other Alchemical spellcasters.

\spellentry{Armorplate} Improves the armor class of each character (although
you won't see any changes in the review screen). The bonus is that it can be
casted outside of battle.

\spellentry{Blades} Does decent damage, especially considering that it's a
damaging spell the \class{Priest} has access to. After multiple castings of
\spell{Armorplate}, you'll want to do \emph{something} with all the spare
Earth spell points you have.

\spellentry{Haste} Speeds up your party. Very useful in fights against
monsters faster than you (or higher level).

\spellentry{Cure Paralysis} Cures or reduces the duration of the ``Paralyze''
ailment. Useful. Get it. \verb|:)|

\spellentry{Restfull} Restores everyone's stamina. The amount restored is
less than what you would get back if you casted Stamina on each character,
but the convenience makes this spell useful. Still, if you can, there are
better spells to get than this.

\subsubsection{Level 4 spells (54 Theology)}%

\spellentry{Conjuration} Summons a monster to help you.  They range from the
pathetic Vulture at a level~1 casting to perhaps \monster{Fieros} and \monster{Myxlmynx} at level
7 (which are pretty darn useful). The best out of the three summoning spells
if you want magical damage.

\spellentry{Paralyze} Paralyzes an enemy. Is about as useful as Web is for
mages and \class{Alchemist}s.

\spellentry{Superman} \textlcsc{VERY} useful. It reduces the amount of
stamina you use when you take a swing at a monster, or whatever form of
attack you use.

\spellentry{Cure Poison} Cures or reduces the ``Poison'' ailment, or reduces
the ``Badly Poisoned'' ailment to ``Poison''. Grab this quickly---you won't
always have that \spell{Cure Poison} potion handy.

\spellentry{Whirlwind} Another attacking spell for the
\class{Priest}. Unfortunately, most monsters don't take a whole lot of damage
from this, if they get hurt at all.

\subsubsection{Level 5 spells (72 Theology)}%

\spellentry{Lightning} The only Fire spell for \class{Priest}s, and it hits in a
group to boot. Grab it as early as possible so that you can have as many
spell points for it as possible unless you already have Fire spell points.

\spellentry{Death} Instantly kills one target. Not too handy as it doesn't
work often---the things it works on are easy to kill anyway in most cases.

\spellentry{Remove Curse} Makes it possible to unequip a cursed item. Quite
useful; however, a weak enough \spell{Remove Curse} will only allow you to
unequip some items and not others, so make sure you use a level~6 or 7
\spell{Remove Curse}.

\spellentry{Healthfull} Heals everyone for a small amount of hit points. This
spell would be nicer if it actually healed more damage---a level~1 Healthfull
seems to heal around 1--4 damage; you do the math. Still, it's group healing,
and that's always nice.

\spellentry{Purify Air} Eliminates or reduces the duration of any cloud
spells casted on your party. Handy when you get pounded with \spell{Poison
  Gas} or \spell{Firestorm}.

\spellentry{Cure Disease} Cures the ``Disease'' ailment. If it doesn't go
away when you cast it, cast a stronger \spell{Cure Disease}. The higher your
vitality, the lower the spell level you will need to cure the disease.

\subsubsection{Level 6 spells (90 Theology)}%

\spellentry{Cure Stone} Cures the ``Stone'' ailment. If it doesn't go away
when you cast it, cast a stronger \spell{Cure Stone}. The higher your
vitality, the lower the spell level you will need to cure the
petrification. When cured, the character who was stoned loses 1 point in
Vitality.

\spellentry{Locate Object} Acts like \spell{Wizard Eye}, only it shows the
locations of any chests in the area (not dropped items).

\spellentry{Lifesteal} Attempts to do a large amount of damage to one enemy
and heal you with that damage. Your spellcaster will never need to cast Heal
Wounds on him or herself again. It works on a \emph{lot} of enemies, but
don't expect it to work on the real nasties like \monster{Rexx} or Godzylli very
often---even at level~7.

\spellentry{Astral Gate} Instantly kills demons. Since it's specialized,
getting this is up to you. You \emph{will} run into demons however.

\spellentry{Recharge} Very useful when used on items that have the effect of
Magicfood when used and can be used multiple times. Otherwise mildly useful
to get back charges on items like the Amulet of Asphyxiation.

\subsubsection{Level 7 spells (98 Theology)}%

\spellentry{Locate Person} Attempts to find all the NPCs. Level~7 \spell{Locate
Person} tends to find all of them.

\spellentry{Word of Death} The \class{Priest} equivalent of \spell{Nuclear
  Blast}. Get it if you need another source of mass damage.

\spellentry{Ressurection} You can always quit-reload, and unlike
\spell{Ressurection} you don't have to wait forever to get it. However, since
it can be used in a fight, it isn't totally useless. However, if you're the
perfectionist type you're probably going to skip this spell. If you ressurect
a character, the character loses 1 point of Vitality and their life counter
increases by~1.

\spellentry{Death Wish} Tries to instantly kill everything. Better than
\spell{Death}, but still horrible. Get it if you really like instant-death;
otherwise don't bother with it if you have a choice, unless you really need
another 20 Divine spell points or you're getting this early in the game.

\WviiNeedSpace{}%
\subsection{Theosophy (\class{Psionic} spellbook)}\label{sec:theosophy-psionic-spellbook}%
%
\subsubsection{Level 1 spells}%

\spellentry{Mental Attack} Your typical level~1 damage spell. It has the
bonus of a large possibility of inflicting ``Insanity'' but this bonus is
rather useless after mid-game.

\spellentry{Bless} Grants you bonuses to hit. Hey, I'd take that any
day. Even later in the game, this spell is useful.

\spellentry{Charm} Paralyzes an opponent. Out of battle, it makes it easier
to make friends with NPCs. If you have diplomatic problems, this may be
useful.  Otherwise, skip for a better spell unless you have no choice.

\spellentry{Sleep} Gives the opponent the ``Asleep'' status for a short
time. One of the most useful spells in the game, even if your party already
has a \class{Bard} with the Poet's Lute. This becomes useless after mid-game
however.

\spellentry{Heal Wounds} There really shouldn't be a reason why you didn't
pick this as your first spell. Any healing is nice.

\spellentry{Stamina} Unlike the \class{Priest}, get this at your
leisure. Still, get it quick, as you'll need stamina recovery spells and
items in the game.

\spellentry{Terror} Attempts to give the ``Afraid'' ailment to a group. Not
very helpful if you're trying to get experience.

\subsubsection{Level 2 spells (18 Theosophy)}%

\spellentry{Psionic Fires} The \class{Psionic} equivalent of
\spell{Fireball}, only earlier. Quite useful at this point in the game.

\spellentry{Knock-Knock} Opens locked chests and doors. Success rate is
related to the power of the spell and the complexity of the lock. Don't
expect to use this spell successfully midgame and onwards.

\spellentry{Cure Lesser Condition} Cures or reduces the duration of the
ailments ``Irritation,'' ``Sleep,'' ``Afraid,'' ``Blind,'' and ``Nausea.''
This spell is not that crucial for the \class{Psionic}--everything can be
cured with a little rest or by the \class{Priest} if you have one in your
party. In the case of ``Nausea,'' make \emph{sure} it is ``Nausea.''
``Disease'' looks a lot like ``Nausea'' if you weren't paying attention
when you got it.  The reason why this spell isn't too useful for the
\class{Psionic} is because the \class{Psionic} actually needs his/her Mental
spell points for spells like Mental attack.

\spellentry{Divine Trap} Helps to identify which trap a chest is booby
trapped with.  You will only need it if your Skulduggery character has a low
amount of Skulduggery, or if you are unfamiliar with guessing the traps of
the game.

\spellentry{Detect Secret} Acts like you have a high amount of scouting
skill. Very useful the first time through the game if you don't want to build
up scouting. Very useless subsequent times.

\spellentry{Identify} Tells you how much damage (if applicable) the item does
and what special effects it may have. If it has a special power, it won't
tell you what it is, but it will mention the that it has one. Special powers
are where when you equip the item, it asks you if you want to invoke its
special power, and generally are very limited in usage. This is best used as
a ``what does this item do'' check, where you cast and reload---however, for
some things (like weapon damage for some weapons) it will give the wrong
numbers.

\spellentry{Confusion} Attempts to inflicts the ``Insanity'' ailment on an
enemy group. Insanity is a nice ailment to have inflicted, but after a while
most of the monsters you want to nail with this are pretty darn resistant to
it.

\spellentry{Watchbells} Wakes up everyone who is asleep. The success rate of
the spell depends on how tired the characters are and how long they've been
sleeping. Not very useful after you get the Mind Control Personal skill---even
before then, the spell doesn't really work too well if you're not already at
full stamina.

\spellentry{Shrill Sound} Hurts an entire enemy group, regardless of
number. For the \class{Psionic} however, it takes a backseat for \spell{Psionic
  Fires} in priority unless you have mass enemy group problems (i.e.\ hordes
of \monster{Rattkin Rogues}\fshyp{}Bandits).

\spellentry{Weaken} Makes the opponent do less damage when they
attack. Useful I suppose, but I prefer killing them as quickly as
possible. Supposedly this spell also lowers resistances, which can be very
useful, but there's no way to really tell for certain.

\spellentry{Slow} Slows down an enemy group. Useful when fighting enemies
stronger than you so that you can try to go first in subsequent rounds. Much
more useful when combined with \spell{Haste} to make sure you go first.

\subsubsection{Level 3 spells (36 Theosophy)}%

\spellentry{Dazzling Lights} The \class{Psionic} version of \spell{Prismic
  Missile}, it dispenses status ailments or just sheer damage. For some odd
reason, this spell in my experience tends to dispense status ailments more
than just pure damage---however, the ``withers and dies'' effect seems to pop
up more often with this. Note: This spell will not work on enemies with 100\%
or greater Light Resistance.

\spellentry{Blades} Unlike the \class{Priest}, the \class{Psionic} already
has decent damage dealing spells. Still, it's useful when you run out of
spell points.

\spellentry{Hold Monsters} Attempts to paralyze an enemy group. Unlike the
other paralyzing spells, the paralyze effect from this is very likely to go
away from getting attacked.

\spellentry{Mind Read} Attempts to read an NPC's mind. You will either see a
phrase that will trigger some tidbit of information from the NPC, something
you could have found out by Loring with the NPC, or ``Nothing'' will flash on
the text part of the screen. Pointless after the first time through the game.

\spellentry{Sane Mind} Cures or reduces the duration of the ``Insanity''
ailment.  Comes in handy at times, as unlike the monsters your characters
seem to be susceptible to insanity quite often.

\spellentry{Blink} Randomly makes the character invisible\fshyp{}visible
throughout the round. Lasts for quite a while, but is not a suggested spell
if you are using a \class{Monk} to get these spells.

\spellentry{Silence} Attempts to silence a group of monsters. Try to avoid
casting this on Alchemical-casting monsters.

\spellentry{Haste} Speeds up your party. Quite useful in trying to make
everyone go first.

\spellentry{Cure Paralysis} Cures or reduces the duration of the
``Paralysis'' ailment. Especially useful when the trap goes off in your face
and leaves you paralyzed for days.

\subsubsection{Level 4 spells (54 Theosophy)}%

\spellentry{Armormelt} Makes it easier to hit and damage monsters. Think of
it as the opposite of \spell{Armor Shield}.

\spellentry{Psionic Blast} A group version of \spell{Mental Attack}, and just about
as useful.

\spellentry{Illusion} Summons monsters to fight for you. Fantasmogoras are at
the high end of things that are summoned by this. The best out of the three
summon spells if you like instant death spellcasters.

\spellentry{Wizard Eye} Gives an overhead view of the area---just like you had
a Journey Map kit. (Surprise\ldots{}) Skip this spell if I were you, unless
you've been skimping on your Mapping skill or don't know if there is a secret
passage behind a wall.

\spellentry{Spooks} The mass, hit-everything version of
\spell{Terror}. Yay. You're better off picking \spell{Psionic Blast} than
this.

\spellentry{Paralyze} Paralyzes one enemy. Same thing as \spell{Web}. You're
not going to be using this successfully when you want to.

\subsubsection{Level 5 spells (72 Theosophy)}%

\spellentry{Death} Useless, if it weren't for the fact that with the huge
amount of Mental spell points the \class{Psionic} has, it actually might work on
something after the umpteenth time. So near useless.

\subsubsection{Level 6 spells (90 Theosophy)}%

\spellentry{Locate Object} Is a \spell{Wizard Eye} spell that also shows the
locations of chests (but not dropped items).

\spellentry{Lifesteal} The \emph{only} true spell the \class{Psionic} will get at
high levels of Theosophy that is actually quite useful. Take it as soon as
you can learn it. The massive amounts of healing is really nice, and you kill
something in the process (sometimes something nasty like a Vampire Vulture or
a Dragorra).

\subsubsection{Level 7 spells (98 Theosophy)}%

\spellentry{Mind Flay} It's a mass target spell, and it's really
nice. Unfortunately it is the worst out of all the mass target spells because
a lot of monsters tend to resist the type of damage it deals, just like
\spell{Psionic Blast} and \spell{Mental Attack}.

\spellentry{Locate Person} Tries to find all the NPCs in the world. A level~7
\spell{Locate Person} tends to find all the NPCs.

\spellentry{Ressurection} You can always quit-reload, and unlike
\spell{Ressurection}, you don't have to wait forever to get it. However,
since it can be used in a fight, it isn't totally useless. However, if you're
the perfectionist type you're probably going to skip this spell. If you
ressurect a character, the character loses 1 point of Vitality and their life
counter increases by~1.

\subsection{Alchemy (\class{Alchemist} spellbook)}\label{sec:alchemy-alchemists-spellbook}%
%
\subsubsection{Level 1 spells}%

\spellentry{Acid Splash} Your basic level~1 damage spell. A couple groups of
monsters (the vapors in particular) are pretty much immune to acid, but not a
lot are.

\spellentry{Itching Skin} Attempts to inflict the ``Irritated'' ailment on a
group of monsters. Tends to work on a lot of them (even some boss-like
monsters like \WviiSPOT{}).

\spellentry{Heal Wounds} Heals one guy. There shouldn't be a reason why you
aren't picking this spell first.

\spellentry{Stamina} Restores stamina. Very useful, so get it in the early
stages of the game.

\spellentry{Sleep} Gives the opponent the ``Asleep'' status for a short
time. One of the most useful spells in the game, even if your party already
has a \class{Bard} with the Poet's Lute. This becomes useless after mid-game
however.

\spellentry{Charm} Paralyzes an opponent. Out of battle, it makes it easier
to make friends with NPCs. If you have diplomatic problems, this may be
useful.  Otherwise, skip for a better spell unless you have no choice.

\spellentry{Poison} Deals damage and attempts to poison the damaged
enemy. Much more useful than \spell{Acid Splash} in some cases, and less so
in others.

\subsubsection{Level 2 spells (18 Alchemy)}%

\spellentry{Blinding Flash} Attempts to inflict the ``Blind'' ailment on a
group of monsters. Don't expect this to work on monsters with light
resistance like \monster{Man~O'~Groves}. If \spell{Dazzling Lights} or
\spell{Prismic Missile} don't work on that monster, this won't either.

\spellentry{Cure Lesser Condition} Extremely useful for the \class{Alchemist}
to have---unlike the other spellcasters, \class{Alchemist}s got a lot of
ailment-happy spells; what do you think happens when they backfire on their
spells?  \verb|:)|

\spellentry{Confusion} Yet another ailment spell---it attempts to inflict the
``Insanity'' ailment on a group of monsters. You won't be using this too much
after mid-game, as most monsters seem to shrug off this spell.

\subsubsection{Level 3 spells (36 Alchemy)}%

\spellentry{Stink Bomb} Yuck. You have better ways of making people
sick. Still, you might like the spell. I don't.

\spellentry{Air Pocket} Deflects \spell{Asphyxiation} spells and reduces the
effectiveness of incoming Air-based spells that stick around as clouds. It
also reduces the effectiveness of breath weapons like Dragonlizard
breath. Don't expect this spell to have any effects on \spell{Whirlwind} or
\spell{Firestorm} however.

\spellentry{Web} You really don't need this spell. What do you prefer,
paralyzing one guy or Blinding the entire group?

\spellentry{Whipping Rocks} Your first group-target damage spell as an
\class{Alchemist}. The fact that it doesn't seem to effectively hit most of
the group at times can be overlooked in this case, as you have nothing better
as a group damage spell at this point.

\spellentry{Cure Paralysis} Cures or reduces the duration of the ``Paralyze''
ailment. Keep it handy.

\subsubsection{Level 4 spells (54 Alchemy)}%

\spellentry{Fire Bomb} Once you get this, you can stop using \spell{Whipping
  Rocks} if you got it. It's the \class{Alchemist}'s equivalent of
\spell{Fireball}, and as a Fire spell it hits quite often. Still, you may
pass this up for the Cloud-type spells you get in this batch of spells.

\spellentry{Acid Bomb} A lasting cloud of Acid damage that lasts around 3--4
turns.  Does decent damage, like \spell{Firestorm}, only not as much.

\spellentry{Crush} Tries to do up to 200 points of damage to one enemy. I say
``tries'' because it misses quite often. That's right, \textlcsc{MISSES}, not
resisted.

\spellentry{Poison Gas} Like \spell{Acid Bomb}, it is a lasting cloud that
sticks around for around 3--4 turns. However, unlike \spell{Acid Bomb}, it has
a chance of poisoning anyone in the cloud, and if they are hit multiple
times, the poison lasts longer. Enjoy.

\spellentry{Cure Poison} Cures or reduces the duration of the ``Poison''
ailment. You will need this as you get hit with nasty things like Vorpal
Swords and attacks that poison you on each hit.

\subsubsection{Level 5 spells (72 Alchemy)}%

\spellentry{Draining Cloud} Drains the stamina out of the group it is casted
on, and lasts for a short while. Useful when you're trying to tire out the
enemy---especially if they use breath attacks or spells.

\spellentry{Cure Disease} Cures the ``Disease'' ailment. If it doesn't go
away when you cast it, cast a stronger \spell{Cure Disease}. The higher your
vitality, the lower the spell level you will need to cure the disease.

\spellentry{Purify Air} Eliminates or reduces the duration of any cloud
spells casted on your party. Handy when you get pounded with \spell{Poison
  Gas} or \spell{Firestorm}.

\spellentry{Deadly Poison} Attempts to make the opponent wither and die;
failing that, it attempts to damage and inflict the ``Badly Poisoned'' status
on the opponent. When this spell works, it \textlcsc{REALLY} works. You'll either love
this spell or hate it. Either way, it's much better than \spell{Death} is.

\spellentry{Create Life} Creates monsters out of nothing. Monsters range from
\monster{Spectral Moths} to Godzylli and \monster{Bloodwyrms}. The best summoning spell out of
the three if you want sheer muscle; heck, it's the best period since the
Godzylli can last much longer than the other summoned creatures.

\subsubsection{Level 6 Spells (90 Alchemy)}%

\spellentry{Toxic Vapors} The cloud version of \spell{Stink Bomb}. Very
useful, and makes sure that at least even powerful monsters can be
incapacitated at times.

\spellentry{Noxious Fumes} Not very useful. Just a better version of
\spell{Stink Bomb} that hits a group. However, there is a much better chance
of nausea than \spell{Stink Bomb}.

\spellentry{Asphyxiation} Wipes out lower level critters with ease. You can
get this decently fast with a mage, so try to grab this spell as fast as
possible if you're on hard mode and the 5 groups of critters that ambush you
each fight start to get on your nerves. Becomes useless near the end of the
game though.

\spellentry{Cure Stone} Cures the ``Stone'' ailment. If it doesn't go away
when you cast it, cast a stronger \spell{Cure Stone}. The higher your
vitality, the lower the spell level you will need to cure the
petrification. When cured, the character who was stoned loses 1 point in
Vitality.

\subsubsection{Level 7 Spells (98 Alchemy)}%

\spellentry{Deadly Air} The \class{Alchemist} mass target spell.  Quite
useful. For an air spell, it isn't resisted very much at all.

\spellentry{Death Cloud} The \class{Alchemist} instant-death spell. Not very
great though, especially with that expensive amount of spell points needed.
\end{multicols}

\chapter{Survival 101}\label{survival-101}%
%
\section{Status Ailments}\label{status-ailments}%
%
\WviiNeedSpace{}Ailments and their cures:%
\begin{description}[style=nextline, labelwidth=4.5em, leftmargin=!, labelindent=0em]
\item[Sleep] \spell{Watchbells}, \spell{Cure Lesser Condition}, (automatic
rousing)

\item[Afraid] \spell{Cure Lesser Condition}, (resting--quick)

\item[Irritated] \spell{Cure Lesser Condition}, (resting--quick)

\item[Blindness] \spell{Cure Lesser Condition}, (resting--quick)

\item[Nausea] \spell{Cure Lesser Condition}, (resting--medium)

\item[Insanity] \spell{Sane Mind}, (resting--medium)

\item[Paralysis] \spell{Cure Paralysis}, (resting--long), successful hit in
combat (small chance)

\item[Poison] \spell{Cure Poison}, (resting--short to long and dangerous!)

\item[Badly Poisoned (getting hit with Deadly Poison)] \spell{Cure
Poison}--reduces to Poison

\item[Disease] \spell{Cure Disease}

\item[Stone] \spell{Cure Stone}

\item[Death\protect\footnotemark{}] \spell{Ressurection}\footnotetext{“Ailment”}
\end{description}

For ailments that can be cured by resting, the appropriate spell reduces the
duration of the ailment until it is down to 0 (i.e.\ the ailment is
cured). For ailments that \textlcsc{CANNOT} be cured by resting, you must use
a set level of the curative spell to restore the character or the ailment
will not go away. This level is based roughly on the subject's Vitality stat:
the higher it is, the lower your casting level can be.

\section{Killing NPCs}\label{killing-npcs}%
%
Generally, it is a good idea not to kill NPCs unless you know you don't need
them---and can deal with the consequences. For example, some people kill
\npc{Father Rulae} in \place{New City} and \npc{Brother TShober} at \place{Eryn River}
for experience, which is fine (although pointless)\ldots{}until they run into
\npc{Xen Xheng} and get slaughtered. If you want to kill NPCs, make sure there are
no more mobile NPCs of the same race still alive. Making them angry enough to
attack you can become awkward at the worst times possible until your party is
a killing machine, as most of the mobile NPCs tend to stalk you.

\section{Inventory Management}\label{inventory-management}%
%
\WviiNeedSpace{}Some pointers:%
\begin{itemize}[leftmargin=*]
\item
  Try to keep your inventories organized.
\item
  Move heavy non-essentials to characters with room, not characters with
  the highest strength. If you move them to a character with high
  strength, chances are that character has the high strength because it
  is wearing heavy armor and can't carry anymore. If the character isn't
  wearing heavy armor though, go for it.
\item
  Try not to carry more than what you need, for two reasons:

  \begin{enumerate}
  \def\labelenumi{\arabic{enumi}.}
  \item
    You will almost always pick up loot. If you're in front of a battle
    generator killing monsters, they will end up dropping loot
    eventually unless you're killing something with no loot at all, like
    the Savant Androids.
  \item
    When you get an item from dialog, it automatically goes to the first
    person with room in their inventory. If no one has room, it
    \textlcsc{REPLACES} an item. You could do this to try to get rid of a key
    item, but chances are that you'll lose something valuable instead. So
    don't have a full inventory whenever possible.
  \end{enumerate}
\item
  Merge items together to save space. When you need to spread the items
  around (i.e.\ you want everyone to have an invisibility potion before a
  fixed fight) merge the said item onto an empty inventory space to
  separate them.
\item
  If you're not wearing it, chances are you don't need to be lugging it
  around if it isn't a key item or a restorative item. Sell what you don't
  want, or drop it if you want it around just in case (Thesminster Abbey in
  \place{New City} is a good place to drop items as it is often used and
  rather large). Try not to drop \textlcsc{TOO} many items however---there is
  a set amount of items that can be dropped in each map area.
\item
  Avoid overloading your characters. When their carrying capacity score
  changes color from grey, their equipment is heavy enough to affect their
  AC, which is bad. Hey, what do you expect after carrying 300~lbs.\ of
  stuff?
\end{itemize}

\chapter{Playing the Game}\label{playing-the-game}%
%
The order of the following locations are a recommended, but not required,
order to get through the game for a brand new party starting just outside of
\place{New City}, based on when maps disappear.

\subsection{Basic Format}\label{basic-format}%
%
\begin{itemize}
\item Name of Area
\item List of monsters encountered (fixed) -- can only be fought in a fixed
encounter (you can't avoid it)
\item Potential Questions
\end{itemize}


\section{Questionaire Walkthrough
(Semi-Spoilers)}\label{questionaire-walkthrough-semi-spoilers}%
%
\subsection{\place{Starting Area} (Near \place{New
City})}\label{starting-area-near-new-city}%
%
\begin{twocolumnitemize}{Monsters encountered:}
\tightlist
\item
  \monster{Bambiphoots}
\item
  \monster{Dandiphoots}
\item
  \monster{Bitterbugs}
\item
  \monster{Stag Weevils}
\item
  \monster{Ravens}
\item
  \monster{Glow Moths}
\item
  \monster{Dane Initiates}
\item
  \monster{Alliphoots} (fixed)
\item
  \monster{Rattkin Rogues} (fixed)
\item
  \monster{Rattkin Bandits} (fixed)
\end{twocolumnitemize}

\faqentry{What do I do? I'm in the middle of nowhere!}

Well first off, equip your items. Then \textlcsc{save}. Get used to saving, as
you will be doing it constantly. Look around for a road and explore a bit.

\faqentry{What about the forest area? Can't I look there too?}

Of course you can. In fact, you should. There's a Journey Map kit that will
definitely help you hiding in a chest in the forest if you are starting a new
game. However, you should put that off until after you have cleared out the
\place{Starter Dungeon}, which is what you are looking for on the road. This is
because the monsters \textlcsc{guarding} the path to the chest are very tough,
and in your new condition you will most likely get creamed. Instead, find a
path on the road leading to a ladder going into the ground. That's the
starter dungeon.

\faqentry{I imported a saved game from Bane of the Cosmic Forge, but I'm not
  where you say I am!}

If you instead started out in a different area because you imported your
game, I really can't help you too much.

If you started near \place{Dionysceus} skip directly to the
\place{Dionysceus} section of this FAQ and see if you can get the
\qmwordqm{CRYSTAL} map in the \place{Temple of Deadly Coffers}\ldots{}if you
can afford the outrageous fees. You may need to use Legerdemain on \npc{Almagorte}
to pay your way through. At the very least try to get the Mind Control skill
before going to \place{New City} when you run out of cash, so that you have a
much easier time getting back to \place{Dionysceus}. Keep in mind that you're
on the same map as a newly created party, but on the other side of the poppy
fields.

If you started near \place{Nyctalinth}: go into the city, talk with
\npc{H'Jenn-Ra}, meet \npc{Shritis} (who you can say no to if you like
the Umpani more, but don't kill any T'rang NPCs until you can take on
\npc{Shritis} if you do!), and then exit into \place{New City} instead of
going back into the \place{Anthracax}. Be warned that most NPCs don't like
the T'Rang and will not trade with you, as you are considered to be their
ally (and enemies of their enemies).

If you started near \place{Ukpyr}: go in the city, join the Umpani I.U.F.,
decide whether or not you actually want to help K'borra T'Rang, and
proceed until you have to deliver a message to \npc{Rodan Lewarx}. (This is
the most profitable way to start, coincidentally.)

Incomplete Bane of the Cosmic Forge games will start where a new party
starts.

Regardless of where you start, you need to get to \place{New City}
eventually---the first quarter of the game is linear if you have not played
the game before (and therefore know what the maps say) and unless you started
in \place{Dionysceus}, the only map you've got a chance of getting is the
\qmwordqm{TEMPLE} map in \place{Orkogre Castle} unless you know exactly what
you are doing.

\pagebreak\subsection{Starter Dungeon}\label{starter-dungeon}\nopagebreak%
%
Monsters encountered:%
\begin{itemize}[WviiTwoColumn]
\item Encountered everywhere
\begin{itemize}
\tightlist
\item
  \monster{Mottle Cruds}
\item
  \monster{Bitterbugs}
\item
  \monster{Ravens}
\end{itemize}

{\nobreaklist{}\item Level 1
  \begin{itemize}
    \tightlist
  \item (No floor specific monsters)
  \end{itemize}}

\item Level 2

\begin{itemize}
\tightlist
\item
  \monster{Stag Weevils}
\item
  \monster{Boring Beetles}
\item
  \monster{Night Rooks} (fixed)
\item
  \monster{Ra-sep-re-tep} (fixed)
\end{itemize}
\end{itemize}


\faqentry{What's this fountain here?}

That fountain restores some health, stamina, and mana every time you sip
it. Don't forget that you can sip multiple times if once isn't enough To sip,
either click on the fountain once and then on the characters that you want to
drink from the fountain, or search and then click on the characters that you
want to drink from it. Don't forget that you can use the number keys (1--6) if
you consider the mouse to be too slow for this! However, be \textlcsc{VERY}
careful with fountains. Most are beneficial, but some have nasty effects on
you. Save before drinking from any fountain.

\faqentry{What the heck am I supposed to do here?}

Build levels. Clean out the entire dungeon. And to get to know the game
better of course. The most important tactic that you will learn over and over
here is to save before entering unexplored rooms and areas. If you have
someone with access to \class{Priest} spells, try to get \spell{Dispel
  Undead} if you can---you'll be happy to have it, I will guarantee that. If
something seems confusing, read the manual!  It is quite descriptive in how
to do things. If you're using a copied version of the game and didn't bother
to get the manual, look for a copy of Wizardry Gold---the manual is built in
in that version by using the F1 key, although Wizardry Gold tends to be more
buggy than Crusaders of the Dark Savant.

Despite what people may say about the game being non-linear, the first part
of the game is linear---you just can't handle most of the monsters in the
game at the level you are at. You'll see what I mean after deciding where to
go after \place{New City}--anywhere but \place{Orkogre Castle},
\place{Dionysceus} (via the poppy fields), and \place{Munkharama} is lethal.

\pagebreak\subsection{\place{New City}}\label{new-city}\nopagebreak%
%
\begin{twocolumnitemize}{Monsters encountered (varies by area of \place{New City}):}%
\tightlist
\item
  \monsterB{Gorn Spearmen}{Gorn Spearman}
\item
  \monster{Gorn Rangers}
\item
  \monster{Demented Munks}
\item
  \monster{Dark Forest Munks}
\item
  \monster{Rattkin Rogues}
\item
  \monster{Rattkin Bandits}
\item
  \monster{Dane Initiates}
\item
  \monster{Dane Disciples}
\item
  \monster{Savant Guards}
\item
  \monster{Savant Troopers} (fixed)
\item
  \monster{Savant Controllers} (fixed)
\item
  \monster{T'Rang Youngers}
\item
  \monster{T'Rang Wilders}
\item
  \monster{Umpani Ruffians}
\item
  \monster{Umpani Renegade}
\end{twocolumnitemize}

\faqentry{The Savant Trooper won't let me through! Why?}

You probably didn't state a place in \place{New City} for it to confirm.
Maybe there's an ad for a store in the \place{Starter Dungeon}
somewhere\ldots{} you must have found \textlcsc{something} resembling paper
in there.

\faqentry{What should I do here?}

Think of \place{New City} as a bigger version of the Starter Dungeon, except
with stores and healing that you have to pay for, for now. Definitely save
before entering a place, as there are a couple doors that hide very nasty
encounters.

\faqentry{Hey! I can't pick or force some of these doors! How do I get
  in?}

If you're referring to the \place{Umpani Detache} and the \place{T'Rshieches
  House}, you'll get in\ldots{}but from a different place. If you are
referring to a certain always-jammed door near the Marina, buy a Knock-Knock
scroll from \npc{Belcanzor} and have someone with at least 1 point in scribe
use it on the door\ldots{}but save first. Either that or use a level 6
Knock-Knock spell to open the door. Very useful treasure in the chest behind
it, considering how early in the game it is.

\faqentry{What's with the Wand Majestik in the \place{Curio Museum}? I stick
  my hand in but I can't grab it! All I get is this weird stone when I leave
  my hand in there and my leader gets diseased!}

\emph{Don't} stick your hand in there unless you have a way to cure your
first character of disease. You can use \npc{Father Rulae} in the Abbey in the
middle of \place{New City}, but you have to ``donate'' all your gold.  Better
to wait until later when you can buy \spell{Cure Disease} potions in
\place{Munkharama} or learn the spell yourself, as the Wand isn't going
anywhere anyway.

\faqentry{Can I use the boat \npc{Sogheim} has in the Marina?}

No, but you'll find a better one.

\faqentry{What the heck do I do at these Devil Faces in the \place{Curio
    Museum}?}

Read the \qmwordqm{BOAT} Map. Funny, some of those words match the
devils\ldots{}

\faqentry{What is that sign in the \place{Condemned Area}?}

It's a way into \place{Old City}. You'll figure out how to get in after you
help \npc{Barlone} in the \place{Rattkin Ruins}' Funhouse with a favor. Maybe
\npc{Professor Wunderland} was the friend he was speaking of---after all, he
\textlcsc{IS} a Rattkin, and he \textlcsc{IS} interested in \place{Old
  City}. Perhaps mentioning the place \npc{Barlone} mentioned will help jog his
memory.

\faqentry{What's the Book of Fables for? I managed to sneak past the Savant
  Android guarding it with my fastest character.}

It is for figuring out the witches names. Read it when you're in the
\place{Witch Mountains}, and especially pay attention to the last parts.

\faqentry{Who's the Gorn Officer that \npc{Paluke}s mentioned?}

You'll find out. Keep in mind when NPCs mention nouns like Officer or even a
place, there is a possibility that saying those words can open a flood of
useful info.

\faqentry{I found a Black Wafer. What does it do?}

It's for the \place{Constabulary}. Think of ETX as Entrance. Don't use it in
the \place{Forbidden Zone} however, unless you \textlcsc{want} to fight
\monster{Savant Troopers} and \monster{Savant Guards}.

\faqentry{How \emph{do} I get in the \place{Forbidden Zone}?}

You walk in. \verb|:)| You mean inside the inside area? Get the Control
Card from the \place{T'Rshieches House}\ldots{}but you'll need to have a chat
with \npc{Shritis} before you can get a chance at the chest.

\faqentry{How do I turn on the computers in the \place{Forbidden Zone}?}

Use the Comm-Link Device from the \place{Umpani Detache}.

\faqentry{What can I do with the computer? It wants me to login to some
  server.}

Nose around \place{Nyctalinth} and wake up a few Savant Androids lying around
Don't forget to nose around the \place{Observation Center} there for some
other answers. This is where the information you got in those two places is
used.

\faqentry{How do I open the Security Cell?}

Whichever server you didn't use (or used but it didn't work) the first time,
use it now.

\faqentry{Nooo! This isn't fair!!! Why isn't the \qmwordqm{LEGEND} map in New
  City?!?}

It tends to disappear \textlcsc{VERY} fast for some odd reason. Consequently,
return the Holy Work to \npc{Xen Xheng} as soon as possible and go looking
for \npc{Xen Xheng} after he leaves the dojo. (Locate Person scrolls
\textlcsc{will} help).  Ask \npc{Xen Xheng} about ``5 Flowers'' and see what
he says. Try to get the \qmwordqm{LEGEND} map now while it is early in the
game after following the instructions he gives you. Seeing as \npc{Ratsputin}
seems to come down here pretty early, I'd say that he's the one that normally
takes the \qmwordqm{LEGEND} map before it switches hands.

\pagebreak\subsection{\place{Orkogre Castle}}\label{orkogre-castle}\nopagebreak%
%
\begin{twocolumnitemize}{Monsters encountered:}
\item Encountered Everywhere (except below Prison):

\begin{itemize}
\tightlist
\item
  \monsterB{Gorn Spearmen}{Gorn Spearman}
\item
  \monster{Gorn Rangers}
\item
  \monster{Gorn Lancers}
\item
  \monster{Gorn Shamans}
\item
  \monster{Gorn Leaders}
\item
  \monster{Gorn Lords}
\item
  \monster{Gorn Ashigaru}
\item
  \monster{Demented Munks}
\item
  \monster{Dark Forest Munks}
\item
  \monster{Mad Warders}
\item
  \monster{Munk Ninjas}
\end{itemize}

\item Level 1

\begin{itemize}
\tightlist
\item
  \monsterB{Rattkin Thieves}{Rattkin Thief}
\item
  \monster{Rattkin Hunters}
\item
  \monster{Savant Guards}
\end{itemize}

\item Level 2

\item Level 3

\item Prison

\begin{itemize}
\tightlist
\item
  \monster{Dragonlizards} (fixed)
\item
  \monster{Fungus Oozes} (fixed)
\item
  \monster{Crawling Wastes} (fixed)
\item
  \monster{T'Rang Watchers} (fixed)
\item
  \monster{T'Rang Wilders} (fixed)
\item
  \monster{T'Rang Guarders} (fixed)
\item
  \monster{Rattkin Hunters} (fixed)
\item
  \monsterB{Rattkin Thieves}{Rattkin Thief} (fixed)
\item
  \monster{Rattkin Bandits} (fixed)
\item
  \monster{Dane Initiates} (fixed)
\item
  \monster{Dane Disciples} (fixed)
\item
  \monster{Boar Weevils} (fixed)
\item
  \monster{Venom Weevils} (fixed)
\item
  \monsterB{Rattkin Thieves}{Rattkin Thief}
\item
  \monster{Rattkin Hunters}
\end{itemize}

\item Below Prison

\begin{itemize}
\tightlist
\item
  \monster{Iguanadons}
\item
  \monster{Glow Mothras}
\item
  \monster{Spectral Moths}
\item
  \monster{Fungus Oozes}
\item
  \monster{Venom Weevils}
\item
  \monster{Stag Weevils}
\end{itemize}

\item Murkatos' \place{Inner Sanctum}

\begin{itemize}
\tightlist
\item
  \monster{Shadow Guardian} (fixed)
\item
  \monster{Spectral Ravens} (fixed)
\end{itemize}
\end{twocolumnitemize}

\faqentry{How the heck do I get into the Throne Room? I haven't found a key
  for the place!}

You'll have to get into the Ape's room first and flip a switch. I hear apes
like bananas\ldots{}but if you didn't bring one with you there are some in
the prison.

\faqentry{How do I open the doors in the Prison then? Is there a switch
  somewhere to throw?}

No, you have to get the Prison Keys from Murkatos' \place{Outer Sanctum}.

\faqentry{\emph{Glares} OK, and how do I get in the \place{Outer Sanctum}?}

You'll need a Polished Steelplate from below to reflect a beam of light into
the gate. Funny, why does my \spell{Detect Secret} eye glow when I pass by
that splot of blood in front?

\faqentry{I found these Boney Combs and Brushes. When is it used?}

Much, \textlcsc{much} later in the game. \place{Isle of Crypts} to be
exact. Look for a hideous Gorn woman there.

\faqentry{Hey! The chest in the Gorn King's room is empty! What gives?}

It looks like someone beat you to the map that was here. Whenever you come
across an empty chest or a chest that had waxy wrappings in it, that is where
a map was located. If the chest is empty, an NPC beat you to the map because
you were too slow. At this point in the game however, only \npc{Brother
  TShober} can beat you to this chest at your top speed---so don't talk to
\npc{Brother TShober} before coming here! This map was the \qmwordqm{TEMPLE}
map. Be very careful when you run into NPCs in an area that has a map---when
you see them around, you're probably too late to get the map! Of course, if
you take your time coming here almost anyone could have the map\ldots{}

\faqentry{That \monster{Shadow Guardian} is too hard! How do I beat it?}

Looks like you went way too fast. However, in the area below that you came
from, if you go back to where you fell down a fixed encounter (where you have
to fight some monsters) at a certain spot respawns---you will always walk into
the spot where the encounter is on your way back up. This must have been put
there for people that got stuck here because they were low level.

You wanted to know how to beat it though: You'll \textlcsc{need} the
\spell{Air Pocket} spell here.  Otherwise, when \spell{Asphyxiation} pops up
(and it will 90\% of the time), you will see most or all of your party die
unless you're high level Useful spells on the \monster{Shadow Guardian} in
addition to \spell{Air Pocket} include \spell{Purify Air} (because of
\spell{Poison Gas} spells it seems to love), \spell{Cure Poison},
\spell{Haste}, \spell{Magic Screen}, \spell{Armorplate}, and \spell{Enchanted
  Blade}.  The last three spells (if you have any of them) should be casted
out of battle before you engage the \monster{Shadow Guardian}.  If you can
summon help with Conjuration\fshyp{}Illusion\fshyp{}\spell{Create Life}, call
some up the first turn as well With those spells under your belt, the
\monster{Shadow Guardian} shouldn't be too much of a problem.

\faqentry{Aaack! All the sudden I'm silenced and 20--30 \monster{Spectral
    Ravens} fight me! What do I do?}%

This is an example where saving saves your butt. If you saved before entering
this area, you'll immediately realize that you're too weak for this
section. Come back later when your party can deal with all those blinking
birds without spells, and you'll have a better chance.

You will want to come back anyway---there's a key in the chest there that will
unlock the gate to a Magic-restoring only fountain here; other contents of
the chest includes the nifty Gem of Power, which can, when invoking it's
special power, give a character the Power Strike Personal Skill at the cost
of losing the Gem; you may want to use the Gem as a protective item instead
for a while however---it grants a −2~AC bonus to the wearer which is a lot
right now.

If you want to fight them now though, bring lots of \race{Faerie} Dust to put them
to sleep, and ready your \spell{Dispel Undead} and any bombs that you may
have found or bought.  You may want some \spell{Cure Lesser Condition}
potions in case a party member manages to run away from the fight because of
the “Terror” status the \monster{Ravens} inflict on you.  \spell{Cure Paralysis}
potions are nice too.

\pagebreak\subsection{\place{Munkharama}}\label{munkharama}\nopagebreak%
%
\begin{twocolumnitemize}{Monsters encountered:}
\item Encountered Everywhere

\begin{itemize}
\tightlist
\item
  \monster{Demented Munks}
\item
  \monster{Dark Forest Munks}
\item
  \monster{Mad Warders}
\item
  \monster{Munk Ninjas}
\end{itemize}

\item {\RaggedRight\ \place{Munkharama} and \place{Land of
      Dreams}\protect\footnote{\place{Land of Dreams}-only Monsters marked
      with an L}}

\begin{itemize}
\tightlist
\item
  \monsterB{Gorn Spearmen}{Gorn Spearman}
\item
  \monster{Gorn Rangers}
\item
  \monster{Gorn Lancers}
\item
  \monster{Gorn Ashigaru}
\item
  \monster{Umpani Ruffians}
\item
  \monster{Umpani Renegade}
\item
  \monster{Spectral Ravens}
\item
  \monster{Vampire Rooks}
\item
  \monster{Night Rooks}
\item
  \monster{Glow Mothras}
\item
  \monster{Rattkin Leaders}
\item
  \monsterB{Rattkin Thieves}{Rattkin Thief}
\item
  \monster{Rattkin Hunters}
\item
  \monster{T'Rang Wilders}
\item
  \monster{Nightmares} L
\item
  \monster{Dream Weavers} L
\item
  \monster{Nightmares} (fixed) L
\item
  \monster{Dream Weavers} (fixed) L
\item
  \monsterB{Furies}{Fury} (fixed) L
\end{itemize}

\item \place{Lost Temple} Level 1

\begin{itemize}
\tightlist
\item
  \monster{Gorn Rangers}
\item
  \monster{Gorn Ashigaru}
\item
  \monster{Umpani Ruffians}
\item
  \monster{Umpani Renegade}
\item
  \monster{Dragonlizards}
\item
  \monster{Komodo Dragons}
\item
  \monster{Skeletons}
\item
  \monster{Minoskell}
\item
  \monster{Vampire Rooks}
\item
  \monster{Spectral Ravens}
\item
  \monster{Dragonlizards}
\item
  \monster{Fungus Oozes}
\item
  \monster{Puxic Oozes}
\end{itemize}

\item \place{Lost Temple} Level 2

\begin{itemize}
\tightlist
\item
  \monster{Gorn Rangers}
\item
  \monster{Gorn Ashigaru}
\item
  \monster{Umpani Ruffians}
\item
  \monster{Umpani Renegade}
\item
  \monster{Umpani Scouts}
\item
  \monster{Spectral Ravens}
\item
  \monster{Spirits}
\item
  \monster{Ghosts}
\item
  \monster{Skeletons}
\item
  \monster{Minoskell}
\item
  \monster{Water Nymphs}
\item
  \monster{Jelly Stingers}
\item
  \monster{Frothing Munks} (fixed)
\item
  \monster{Leper Giants} (fixed)
\item
  \monster{Lord of Dark Forest} (fixed)
\end{itemize}

\item Crypt

\begin{itemize}
\tightlist
\item
  \monster{Skeleton Lords} (fixed)
\end{itemize}
\end{twocolumnitemize}

\faqentry{What's the answer to the well's riddle?}

What's another name for a quarter, a dime, a nickel, or a penny?

\faqentry{What is with that @\#@!@ Roulette? I put the beans in, but it
  always comes out \textlcsc{WHITE} \textlcsc{WHITE} \textlcsc{WHITE}
  \textlcsc{WHITE}!}

Well, you can avoid it by swimming past it to the Rubber Bear chest; you'll
just miss out on money and experience. If you want to solve it however, from
what I can tell the order the beans are placed in the receptacles depends on
which rooms you go first, and in which order you end up placing the
beans. Meaning it's random. \verb|:)| The listed colors on the roulette
only reflect how many beans you got right, and not which rooms they are
in. (Sorry about that.)  Try doing this as a process of elimination:

\begin{enumerate}[leftmargin=*]
\def\labelenumi{\arabic{enumi}.}
\item
  Grab all the beans. Try to have some system where you know which
  corner you found a bean. The easiest way is I can think of is to make
  a chart on paper with four boxes making up a larger square, kind of
  like this:
%
\begin{center}
\begin{tikzpicture}
    \matrix (m) [
  matrix of nodes,
  nodes in empty cells,
  every node/.style={anchor=base,text depth=.5ex,text height=2ex,text width=1em},
  nodes={outer sep=0pt}]
  {
 &  &  &  &  &  \\
 &  &  &  &  &  \\
 &  &  &  &  &  \\
 &  &  &  &  &  \\
  };
  % Horizontal lines.
  \draw[very thick, line cap=round] (m-1-1.north west) -- (m-1-6.north east);
  \draw[very thick, line cap=round] (m-3-1.north west) -- (m-3-6.north east);
  \draw[very thick, line cap=round] (m-4-1.south west) -- (m-4-6.south east);
  % Vertical lines.
  \draw[very thick, line cap=round] (m-1-1.north west) -- (m-4-1.south west);
  \draw[very thick, line cap=round] (m-1-3.north east) -- (m-4-3.south east);
  \draw[very thick, line cap=round] (m-1-6.north east) -- (m-4-6.south east);
\end{tikzpicture}
\end{center}

\item
  Clear out all the receptacle rooms if you haven't already.
\item
  \textlcsc{SAVE}.
\item
  Now, to mark things, NW, SW, NE, and SE are the four corners and the
  beans found in each corner as the lowercase version of the direction.
  If you use the grid above though, you can just mark an $\times$ in the
  appropriate corner.
\item
  Now try this to eliminate:
\end{enumerate}
%
\begin{itemize}
  \tightlist
\item
  nw → NW
\item
  sw → SW
\item
  ne → NE
\item
  se → SE
\end{itemize}

This places the beans in the receptacle closest to them. After you spin
the roulette, you will have several options:

\paragraph{Option 0: One White.} This will never show up. Don't even try to
tell me that you can get this!

\paragraph{Option 1: Two whites.} Pick two of the beans and switch them the
next time you try. For example, if the roulette came out Black Black White
White, try:%
%
\begin{itemize}
\tightlist
\item
  nw → NW
\item
  sw → SW
\item
  ne → SE
\item
  se → NE
\end{itemize}

If this worsens things to four whites, then you know the two that you did
\textlcsc{NOT} switch don't belong where they are and can solve the puzzle by
switching them. If this worsens to three whites, then you know that one of
the beans you switched was right, and you'll have to go back to the order you
had before and pick one of the same beans you switched the previous time and
a bean you didn't switch and see what happens.

In my case, you would put that the NW and SW don't belong in the NW and SW
receptacles, respectively. If it turns out better, then you've solved the
puzzle. Whichever it is, it can't stay the same.

\paragraph{Option 2: Three whites.} Pick one of the white spots and try out
the other beans in there until you come up with that spot as black as
well. Then treat it like it was two white two black, only you'll know exactly
which two beans need to be switched

\paragraph{Option 3: Four whites.} Since you've now eliminated four places
where beans should be, your grid is probably marked like this:
%
\begin{center}
\begin{tikzpicture}
    \matrix (m) [
  matrix of nodes,
  nodes in empty cells,
  every node/.style={anchor=base,text depth=.5ex,text height=2ex,text width=1em},
  nodes={outer sep=0pt}]
  {
$\mathbf{\times}$ &  & & &  & $\mathbf{\times}$ \\
 &  &  &  &  &  \\
 &  &  &  &  &  \\
$\mathbf{\times}$ &  & & &  & $\mathbf{\times}$ \\
  };
  % Horizontal lines.
  \draw[very thick, line cap=round] (m-1-1.north west) -- (m-1-6.north east);
  \draw[very thick, line cap=round] (m-3-1.north west) -- (m-3-6.north east);
  \draw[very thick, line cap=round] (m-4-1.south west) -- (m-4-6.south east);
  % Vertical lines.
  \draw[very thick, line cap=round] (m-1-1.north west) -- (m-4-1.south west);
  \draw[very thick, line cap=round] (m-1-3.north east) -- (m-4-3.south east);
  \draw[very thick, line cap=round] (m-1-6.north east) -- (m-4-6.south east);
\end{tikzpicture}
\end{center}

Try:%
%
\begin{itemize}
\tightlist
\item
  nw → NE
\item
  sw → SE
\item
  ne → NW
\item
  se → SW
\end{itemize}

If that comes out all white, then you now have more places eliminated,
and your grid probably looks like this:%
\begin{center}
\begin{tikzpicture}
    \matrix (m) [
  matrix of nodes,
  nodes in empty cells,
  every node/.style={anchor=base,text depth=.5ex,text height=2ex,text width=1em},
  nodes={outer sep=0pt}]
  {
$\mathbf{\times}$ &  & $\mathbf{\times}$ & $\mathbf{\times}$ &  & $\mathbf{\times}$ \\
 &  &  &  &  &  \\
 &  &  &  &  &  \\
$\mathbf{\times}$ &  & $\mathbf{\times}$ & $\mathbf{\times}$ &  & $\mathbf{\times}$ \\
  };
  % Horizontal lines.
  \draw[very thick, line cap=round] (m-1-1.north west) -- (m-1-6.north east);
  \draw[very thick, line cap=round] (m-3-1.north west) -- (m-3-6.north east);
  \draw[very thick, line cap=round] (m-4-1.south west) -- (m-4-6.south east);
  % Vertical lines.
  \draw[very thick, line cap=round] (m-1-1.north west) -- (m-4-1.south west);
  \draw[very thick, line cap=round] (m-1-3.north east) -- (m-4-3.south east);
  \draw[very thick, line cap=round] (m-1-6.north east) -- (m-4-6.south east);
\end{tikzpicture}
\end{center}

So try:%
%
\begin{itemize}
\tightlist
\item
  nw → SE
\item
  sw → NE
\item
  ne → SW
\item
  se → NW
\end{itemize}

and use the process of elimination to figure out where the beans go If you
still can't solve it there's nothing I can say that will help you.  It's hard
to keep track of the beans though, and you may have switched one pair
accidentally.

I have only managed to get \textlcsc{BLACK} \textlcsc{BLACK} \textlcsc{BLACK}
\textlcsc{BLACK} on the first try \textlcsc{ONCE}. Usually I end up getting it
right by the \nth{3} or \nth{4} try. As a reward however you get free cash,
experience, and a secret door opens up in the SE room.

\faqentry{What do I do at the \place{Land of Dreams}? The Palace Munk doesn't
  like my answers!}

I hope you remembered to ask \npc{Brother Moser} about rumors if you've met
him already. Ask him about the other name you saw with ``\place{Land of
  Dreams}'' on the front door for answers the Munk will like.

\faqentry{OK, I'm in the \place{Land of Dreams}, I stepped through the black
  door, but now I can't get out. What do I do?}

Remember what those Munks you saw were doing? Maybe you can too with the junk
they left behind. You have to go through the entire scenario until it repeats
before the activity will have some effect though.

\faqentry{What's a good item to pick?}

Totally up to you. The weapons are cursed but really strong (for this point
in the game) and the protective items, although cursed, grant a regeneration
bonus on the wearer (at least two of them do) and \textlcsc{really} good
protection.

\faqentry{What the heck is \npc{Xen Xheng} saying?}

You should have talked to \npc{Father Rulae} way back in \place{New City} and
said you wanted guidance, and followed his directions and the subsequent
directions from \npc{Brother TShober}. Otherwise, this is a dead end for you.

\faqentry{Is there something special about that middle area in
  \place{Munkharama} with Phoonzang's statue?}

Read the \qmwordqm{TEMPLE} map for hints. I hope you have decent swimming (at
least 20) to at least get to the middle area!  Hope your Scouting is decently
high, or that you use \spell{Detect Secret} quite often. If your swimming
isn't up to par, you may have to practice in the Polar Munk wading pool.

\faqentry{OK, I'm in this big underground dungeon. What do I do here?}

Locate an exit first. A\textlcsc{lways} locate an exit if you get dumped into
the middle of nowhere---that way when you're half-dead you have a way out When
you come out of the place, \textlcsc{save}, and clear a way back to
\place{Munkharama}--\place{Munkharama} is going west on the path you will end
up on After you have a way out that is cleared of fixed encounters (i.e.\
encounters that aren't random), go in back in and explore\ldots{}after you
get healed.

\faqentry{Um\ldots{}why did I just fall down a pit?}

Uh oh\ldots{}I hope you didn't save. This area is annoying. However, it
\emph{is} a good source of XP if you have \spell{Asphyxiation} and
\spell{Cure Disease}, at least for your first time through. And you can get
out eventually.

See, you're going to be fighting \monster{Frothing Munks} and a
\monster{Leper Giant} (two if on Expert mode) here. The \monster{Leper Giant}
is annoying because he's a giant---they tend to squash you like a bug if they
hit you. The annoying thing about the \monster{Frothing Munks} though is that
they throw up on you---and you get diseased!

However, the \class{Monk}s are at least weak against \spell{Asphyxiation} (or
a mass target damage spell like \spell{Nuclear Blast}), so if you have that
and \spell{Cure Disease}, they'll be an easy source of XP for you. Keep in
mind that they don't have a lot of hit points if you try using \spell{Nuclear
  Blast} or another similar spell. A level 2 or 3 one should kill them. The
\monster{Leper Giant} just takes a bit of perseverance (and lots of luck with
status ailment spells).

\faqentry{\texorpdfstring{But I don't have those
    spells!\protect\footnotemark{} What can I do about it?}{But I don't have
    those spells! What can I do about it?}}\footnotetext{See previous
  question.}

Not much. Don't go there is the only advice I can give---getting diseased at
this point in the game is \emph{very} nasty, and it's not really a necessary
area.

\faqentry{Yikes! That \monster{Lord of the Dark Forest} is hard! How do I
  kill him?}

With persistence. If you are \textlcsc{very} lucky, you can silence
him---he's a monk, unlike the rest of the Munks who are ninjas, so shutting
him up is a very effective way of stopping the nastier spells like
\spell{Lifesteal}. Deal with anyone he has with him first though. He might be
dangerous, but so is 6 \monster{Dark Forest Munks} that get a chance to cast
stuff on you as well, leaving you blind, irritated (both literally and
figuratively), and possibly insane, making it near impossible for you to beat
the \monster{Lord of the Dark Forest}.

If you have \spell{Fireball}\fshyp{}\spell{Iceball} or \spell{Nuclear Blast}
by this point (\spell{Nuclear Blast} not very likely at this point unless you
had a mage that changed profession to a \class{Bishop}), the others shouldn't
be too much of a problem if you are faster. This is why you needed to find
the exit first, so that you can come here in better health. However, if you
cannot defeat the \monster{Lord of the Dark Forest} at this point, you have
no hope of getting the \qmwordqm{CRYPT} map before someone else, as it means
that you are too low level to effectively kill him---by the time you're not,
the map will be gone.

\faqentry{What do I do with the Notched Shaft I found?}

Look for what looks to be like a pressed button and use it there while facing
it. It closes that pit that dropped you into the \monster{Frothing Munks} and
\monster{Leper Giant}. Keep in mind that the pressed button is located in the
same region (although not the same passageway) as the pit and ladder.

\faqentry{I can't find the Holy Work!}

Remember what \npc{Brother TShober} said about the ``golden face.''

\faqentry{What is with these waxy wrappings with the Holy Work?}

The chest also contained the \qmwordqm{CRYPT} map. Looks like someone beat
you to it. Try hunting down some Gorn NPCs or \npc{Brother TShober} (strangely
enough). They seem to be the first ones with it if you don't get it in all
the games I've played.

\faqentry{Aaah! Those \monster{Skeleton Lords} are creaming me!! And why am I paralyzed?}

Oh dear. Yet another example of ``Come back later.'' If you're playing on
expert mode you will \textlcsc{need} either \spell{Word of Death} on two
characters (because 1 level 7 \spell{Word of Death} OR \spell{Nuclear Blast}
won't send them back to the grave most of the time) or at least 3 characters
with access to Silence or the Silent Lyre, or (for the power hungry)
characters high level enough to shrug off Death and \spell{Fireball} spells.
Then you have to hope that the characters with Silence don't get
paralyzed. \spell{Nuclear Blast} tends to not work as well as \spell{Word of
  Death} on undead. There's no way you can take more than 2 turns of spells
from the \monster{Skeleton Lords} without getting someone clobbered to death, even with
\spell{Magic Screen} up. If you can, you're lucky---Death and 6
\spell{Fireball}s have a nasty effect on my party at this point.

\pagebreak\subsection{\place{Dionysceus}}\label{dionysceus}\nopagebreak
%
\begin{twocolumnitemize}{Monsters encountered:}
  \item Encountered Everywhere

    \begin{itemize}
      \tightlist
    \item \monster{Dane Initiates} (level 1 up)
    \item \monster{Dane Disciples} (level 2 up)
    \item \monster{Dane Canons} (level 3 up)
    \item \monster{Dane Priests} (level 4 up)
    \item \monster{Dane Apostles} (level 5 up)
    \item \monster{High Fathers} (level 6 up)
    \end{itemize}

  \item First level\fshyp{}\place{Temple of the Initiate}
    \begin{itemize}
      \tightlist
    \item \monster{Cachre Sludges}
    \item \monster{Glow Moths}
    \item \monster{Spectral Moths}
    \item \monster{Bitterbugs}
    \item \monster{Stag Weevils}
    \item \monster{Night Rooks}
    \end{itemize}

  \item Second level\fshyp{}\place{Temple of Divine Order}
    \begin{itemize}
      \tightlist
    \item \monster{Night Rooks}
    \item \monster{Vampire Rooks}
    \item \monster{Iguanadons}
    \item \monster{Skeletons}
    \item \monster{Venom Weevils}
    \item \monster{Stag Weevils}
    \end{itemize}

    \item Third level\fshyp{}\place{Temple of Eternal Night}
      \begin{itemize}
        \tightlist
      \item \monster{Bantari}
      \item \monster{Vampire Rooks}
      \item \monster{Dragonlizards}
      \item \monster{Fungus Oozes}
      \item \monster{Minoskells}
      \item \monster{The Beast} (fixed)
      \item \monster{Boar Weevils}
      \item \monster{Venom Weevils}
      \item \monster{Stag Weevils}
      \end{itemize}

    \item Fourth level\fshyp{}\place{Temple of Aerial Whimsey}
      \begin{itemize}
        \tightlist
      \item \monster{Bantari}
      \item \monster{Vampire Rooks}
      \item \monster{Spectral Ravens}
      \item \monster{Fire Crows}
      \item \monster{Dragonlizards}
      \item \monster{Fungus Oozes}
      \item \monster{Minoskells}
      \item \monster{Bear Weevils}
      \item \monster{Boar Weevils}
      \item \monster{Venom Weevils}
      \item \monster{Stag Weevils}
      \end{itemize}

    \item Fifth level\fshyp{}\place{Temple of Deadly Coffers}
      \begin{itemize}
        \tightlist
      \item \monster{Dragon Rooks}
      \item \monster{Fire Crows}
      \item \monster{Hog Beetles}
      \item \monster{Wraiths Spirits}
      \item \monster{Ghosts}
      \item \monster{Water Nymphs}
      \item \monster{Jelly Stingers}
      \item \monster{Gelimaga}
      \end{itemize}

    \item Sixth level\fshyp{}\place{Temple of Wanderers}
      \begin{itemize}
        \tightlist
      \item \monster{Shadow Crusts}
      \item \monster{Puxic Oozes}
      \item \monster{Fungus Oozes}
      \item \monster{Skeleton Lords}
      \item \monster{Fetid Corpses}
      \item \monster{Zombie Skells}
      \item \monster{Savant Troopers}
      \item \monster{Savant Guards}
      \item \monster{Fire Crows}
      \item \monster{Dragon Rooks}
      \item \monster{Vampire Vultures}
      \item \monster{Komodo Dragons}
      \item \monster{Bear Weevils}
      \item \monster{Boar Weevils}
      \item \monster{Venom Weevils}
      \item \npc{Magna Dane} (fixed)
      \end{itemize}
      
    \item \place{Tower Pit Exit}
      \begin{itemize}
        \tightlist
      \item See \place{Temple of Wanderers} and \place{Tower's Top} (except
        any fixed encounters)
      \end{itemize}

    \item \place{Tower's Top}
      \begin{itemize}
        \tightlist
      \item \monster{Luna Mothras}
      \item \monster{Glow Mothras}
      \item \WviiSPAWN{} (fixed)
      \item \monster{Vampire Vultures}
      \item \monster{Wraiths Spirits}
      \end{itemize}
\end{twocolumnitemize}

\faqentry{I can't make it through the Poppy Field to get here! Are you nuts?}

Do the Save\fshyp{}Restore method. Save each step \textlcsc{only} if at most
1 person falls asleep. You only need to make it through the poppy field once
doing this tedious method. Then turn left at the intersection you will come
across after the field. Now if you started near \place{Dionysceus}, you don't
have to go through this. Don't forget to bring \spell{Watchbells},
\spell{Cure Lesser Condition} spells and \spell{Cure Lesser Condition}
potions to stay awake if necessary, and if you have the Deadman's Hair, equip
it on someone (an elf character preferably, as elves already have hypnosis
resistance and this will boost it).

Alternately, you can spin your party until they wake up.  Spinning counts as
taking a step without actually taking a step, so you can move, spin the party
awake, and repeat.  However this is very time consuming.

\faqentry{What is \npc{Almagorte} talking about?}

These guys are greedy little misers. What do misers love most?

\faqentry{Aaack! I'm getting clobbered by traps!}

Buy some Jonga Powder from \npc{Almagorte} if you can't take the traps. (I
can't unless \place{Dionysceus} is the last area for me to tackle before
trying to go to the \place{Dragon Mountains}.) On most of the levels there
will be an urn like the one in that little 1$\times$1 arched area you saw
right after entering the Temple of the Initiate. Use some Jonga Powder
there. You'll need to buy Jonga Powder twice from \npc{Almagorte}. Save a use
of Jonga Powder for after this Tower by the way. You'll need it.

\faqentry{What are these little Golden Idols for?}

For opening the gate to the next level of course. Each level has one, and you
have to find the altar for it. Have fun looking.

\faqentry{(Pick a level) of the Dane Tower is too hard!}

Then come back later when you're stronger. However, it is possible on expert
mode for a party created from level one characters to get to the Temple of
Wanderers after clearing out the Starter Dungeon and \place{New City}, so
don't worry \textlcsc{too} much if a fight seems too hard.

\faqentry{What do I do on the \place{Temple of Divine Order}?}

Move around. There are little 1$\times$1 alcoves here. Stepping in them moves
the pits around. Try to fudge your way through. Burning powder in the urn
here (in an alcove before entering the Temple) acts as a \spell{Levitate} so you
don't have to worry about accidentally falling, and turns off a trap that
Silences your characters.

\faqentry{There's a doorway that opens onto darkness. Should I go in?}

Yes. That's the \place{Temple of Eternal Night}. I hope your mapping skills are
adequate.\footnote{At least 10 to be able to map walls--30 if you want to see
  alcoves.} Turn left immediately upon entering the Temple to look for the
urn---don't be afraid to try walking into what seems to be a wall if you have
horrible mapping skills---it might be the alcove with the urn.  After that,
explore the Temple at your leisure. There's a fountain on this floor in the
western half of the temple that restores HP, Stamina and Mana, so have fun
looking.

\faqentry{Bah. I hate that purple haze. Any way to turn it off?}

Yeah, if you went left in the beginning of the Temple like I said to and
looked for the urn. That automatic fight after the haze will go away as well.

\faqentry{Should I try to learn ``the word''?}

Yes. Do as the Dane ask and then come back here. Meditate on the word and
you'll find that everyone who doesn't fall asleep learns 1 point in the Mind
Control Personal skill. Repeat until everyone has a minimum of 1 point in the
skill. Now you'll never worry about the poppy field again, and as a bonus you
get better resistance to mental stuff.

\faqentry{Where is my money going? It seems to be disappearing!}

Just so you know, when Tollen Dane comes out to congratulate your rank,
you're also paying fees. The most expensive up at the top is 10,000 gold!
Meaning, if you can't pay up, come back later when you can.

\faqentry{What is with the \place{Temple of Aerial Whimsey}?}

It's a teleporter maze. Happy mapping---those teleporters aren't going to show
up. After a while the places you get sent to will look familiar Also, turn on
the \spell{Direction} spell (or look at your Journey Map Kit) as there is a
spinner teleporter as well. There are also two teleporters here that will
send you back to the \place{Temple of Eternal Night}, which can be a blessing
or a pain depending on whether you need a couple of drinks at the fountain or
not.

\faqentry{Who's the girl in the vision?}

That's \npc{Vi Domina}. Remember the girl from the introduction? (assuming
you watched it) Maybe you should find more about her when you can.

\faqentry{Aack! Treasure Chests are exploding!}

Sounds like you're in the \place{Temple of Deadly Coffers}. You're just going
to have to take the punishment---try to get it so that booby traps that hit
you are only benign ones like Stunner or minorly hurting ones like
Dagger---getting stoned or killed means you reload for sure.

\faqentry{There's a locked gate here in the far right area with the little
islands of chests, but none of the keys from the chests here work!
What's in the chest behind it?}

You need a key from the first area of the \place{Temple of Deadly Coffers}
where the urn in the middle is. The chest behind it contains the
\qmwordqm{CRYSTAL} map, but if you're playing on Expert mode, it's long gone
by now unless you started in the area---\npc{Kymas Turan} or \npc{Ratsputin}
probably took it first.

\faqentry{Where is everything in the \place{Temple of Wanderers}? It just
  looks like a bunch of dead ends.}

In this area, you have to see what new areas open up when you step in an
alcove. That means you're going to have to step in \emph{every} alcove and
check your Journey Map Kit to see if new areas opened up, and if your mapping
skill is below 30 here, you're going to be in big trouble. Keep in mind that
when I say alcove, I mean a 1$\times$1 room that only has one way in and out.

\faqentry{What's the buzzing sensation I'm getting?}

In this case, it means that you've reset the level to its original
configuration with everything closed.

\faqentry{Where do I put the Idol on this floor? I can't find an altar
  anywhere!}

Look in the far NW area of the Temple on this floor. There should be an
alcove you circle around while walking that is opened up by a secret button
on the wall.

\faqentry{What the\ldots{}I'm getting hit with fireballs! What's going on?}

You're in a nasty area of the Tower. The area behind the grates to your east
(where the fireballs are coming from) is the \place{Temple of the \npc{Magna
    Dane}}, which you will be able to get to after passing the \place{Temple
  of Wanderers} and paying for your membership as a Lord of Dane. Be warned
that going down this corridor means you will be taking heavy unpreventable
(except by luck and your fire resistances) fire damage and get into a fight
with \monster{High Fathers} and other Danes after the gate at the very
end. The fire damage goes away after you kill \WviiSPAWN{} or talk (and take
the Pit challenge) with the \npc{Magna Dane} though, as the \npc{Magna Dane}
will have left his Temple. I've gotten stuck here before though, as the Pit
exits here after a bit of winding and my party just can't take 80--100+ HP
fireballs at this point.

\faqentry{What is with this lever? I pull it and it resets?}

Yeah. It opens (and closes) all the passageways that are opened by alcoves on
the level. So if you walked in here when the fireballs are still going off
and yanked the lever, do it twice. Otherwise you'll be in a world of hurt as
you leave, find the passageway closed, and then have to go back to the lever
to reopen it again.

\faqentry{What do I do at the Pit?}

Talk to \npc{Magna Dane} to find out. You get to him by taking the stairs
down after going to the top of the tower, not by yanking a lever.

\faqentry{\npc{Magna Dane} is \textlcsc{hard}. Every time I start the fight,
  half my party is dead before I can do anything!}

Come back later. It is easier when everyone has at least a Dexterity of 16
After the fight, don't forget to search the area for any extra
booty. \emph{hint} \emph{hint}

\faqentry{Ack! The Ring of Demons is killing me when I put it on!}

\emph{Don't} put it on anyone that doesn't already have a Regen~+1 or a
Regen~+2 item equipped. That being said, if you have a \race{Faerie} and got
the \race{Faerie} Cap from the chest in the \place{Temple of Deadly Coffers},
or the one with the Chrome Key in the \place{Temple of Wanderers}, it is safe
to put it on him or her---the \race{Faerie} will just lose the HP regen bonus
because it cancels out. As a note about negative Regen items, the game
appears to take the highest regeneration bonus out of all the items you have,
and add the lowest negative regeneration penalty of the items you have to it,
to get an end result. That means that if you have a Ring of Demons and both
Gowns of Divine Mail on the same character, that character ends up with no
regeneration bonus, \textlcsc{NOT} a Regen~+2 bonus. So that Regen~+1 item
allows you to have as many −1 regen items equipped as you want.

\faqentry{How do I open the demon-head gate?}

What did you chop off the last demon you killed?

\faqentry{\texorpdfstring{But it didn't work!\protect\footnotemark{}}{But it
    didn't work!}}\footnotetext{See previous question.}

Try it from the \npc{Magna Dane}'s chamber. You \textlcsc{did} see that
secret button right?

\faqentry{What do I do with what's inside?}%

Whatever you want. You need the Coil for later though.\goodbreak

\pagebreak\subsection{\place{Ukpyr}}\label{ukpyr}\nopagebreak%
%
\WviiNeedSpace{}%
Monsters encountered:%
\begin{itemize}[WviiTwoColumn]
    \tightlist
  \item \monster{Umpani Ruffians}
  \item \monster{Umpani Renegades}
  \item \monster{Umpani Scouts}
  \item \monsterB{Umpani Armsmen}{Umpani Armsman}
  \item \monster{Umpani Troopers}
  \item \npc{General Yamo} (fixed)
  \item \npc{Sgt.~Balbrak} (fixed)
  \item \npc{Lt.~Gromo} (fixed)
  \item \monster{Spirits}
  \item \monster{Rattkin Leaders}
  \item \monster{Rattkin Hunters}
  \item \monsterB{Rattkin Thieves}{Rattkin Thief}
  \item \monster{Rattkin Bandits}
  \item \monster{Rattkin Rogues}
  \item \monster{T'Rang Wilders}
\end{itemize}

\faqentry{\texorpdfstring{Why do the Umpani immediately attack me when I
    enter \place{Ukpyr}?\protect\footnotemark{}}{Why do the
    Umpani immediately attack me when I enter
    \place{Ukpyr}?}}\footnotetext{From Bernice Carter.}

One of the following might have happened:

\begin{itemize}
\tightlist
\item
  You imported a game and started near \place{Nyctalinth}.
\item
  You killed too many Umpani NPCs and didn't bother to rectify the
  situation by Trucing with other Umpani NPCs and making them happy.
\end{itemize}

Either way, find an Umpani NPC (\npc{Rossarian} in the Arms of Argus at
\place{New City} is ideal) and Truce until the NPC is willing to trade with
you.  If you're using Wizardry Gold however, you're out of luck due to the
bug with diplomacy (bug being that it just simply doesn't work\ldots{}at
all).

\faqentry{Should I join the Umpani?}

Definitely---even if you hate them. You can betray them for the T'Rang by
helping K'Borra and lying about his presence near \place{Ukpyr} to the Umpani
if you hate the Umpani. Keep in mind that here is a great way to make
money---especially if you need it to pay off the Danes for going up in their
tower. You need at least 8000 gold for an outfitting however or you won't get
the returns on your investment.

\faqentry{What's the Humpawhammer?}

It's a teleportation device. It will transport you to the \place{Umpani
  Detache} in \place{New City} and is used for your (more profitable)
mission.

\faqentry{Should I help K'borra and the T'Rangs?}

Again, it is totally up to you. You may want to know that I've yet to get any
compensation from K'borra, although it may involve me finding him
again. Supposedly though, you need to talk to \npc{Shritis} to get your
reward.

\faqentry{Where exactly is information K'borra wants?}

It's in the \place{Spaceport Authority}. If you're going to help K'borra,
don't tell \npc{Sgt.~Balbrak} that you found T'Rangs! After you talk to
\npc{General Yamo}, you'll be able to go into the \place{Spaceport Authority}
without problems.

\faqentry{I'm giving K'borra the information he wants, but he's calling me a
  traitor! Why?}

Don't put a space between the ‘+’ and the rest of the coordinates. The only
space in there should be after the \nth{2} number and before the `D'.

\faqentry{I can't get that promotion \npc{Lt.~Gromo} said I'd get, even
  though I hit the practice target 3 times!}

He said to get 3 BULL'S EYEs, not 3 hits. That being said, even though it is
possible to get a BULL'S EYE at 0 Firearms skill (My \class{Valkyrie} once
got BULL'S EYE HIT HIT at a Firearms skill of 0), chances are you won't get 3
BULL'S EYEs at a high enough rate until you have a skill of at least 40 in
Firearms. After that it is very possible, but sometimes it takes a
while. (Once I got 3 BULL'S EYEs at 43 Firearms skill, another time it was at
64, and yet another at 53.) You will get a IUFSTFTUFS badge to bring to the
Supply Depot, and then be offered 14250 for a Blunderbuss and Flak Jacket. No
discounts on that unfortunately.  If you want the items though, get them
before meeting \npc{General Yamo}.


\pagebreak\subsection{\place{Rattkin Ruins}}\label{rattkin-ruins}\nopagebreak
%
\begin{twocolumnitemize}{Monsters encountered:}
  \item Encountered everywhere
    \begin{itemize}
      \tightlist
    \item \monster{Rattkin Ronin}
    \item \monster{Rattkin Leaders}
    \item \monster{Rattkin Hunters}
    \item \monsterB{Rattkin Thieves}{Rattkin Thief}
    \item \monster{Rattkin Bandits}
    \item \monster{Rattkin Rogues}
    \item \monster{T'Rang Wisers}
    \item \monster{T'Rang Watchers}
    \item \monster{T'Rang Tecniks}
    \item \monster{T'Rang Keepers}
    \item \monster{T'Rang Guarders}
    \item \monster{T'Rang Youngers}
    \item \monster{Dane Canons}
    \item \monster{Dane Disciples}
    \item \monster{Dane Initiates}
    \item \monster{Puxic Oozes}
    \item \monster{Fungus Oozes}
    \item \monster{Crawling Wastes}
    \item \monster{Fire Crows}
    \end{itemize}

  \item \place{Rattkin Ruins}
    \begin{itemize}
      \tightlist
    \item \monster{Grimal} (fixed)
    \item \monster{Rattkin Razuka} (fixed)
    \item \monster{Savant Controllers}
    \item \monster{Savant Troopers}
    \item \monster{Savant Guards}
    \end{itemize}

  \item \place {Rattkin Funhouse}
    \begin{itemize}
      \tightlist
    \item Upstairs
      \begin{itemize}
        \tightlist
      \item \monster{Zombie Skells}
      \end{itemize}

      \item Underground
        \begin{itemize}
          \tightlist
        \item \monster{Gelimaga}s
        \item \monster{Jelly Stingers}
        \item \monster{Zombie Skells}
        \item \monster{Fetid Corpses}
        \end{itemize}
    \end{itemize}
\end{twocolumnitemize}

\faqentry{How do I get in? The entrance is blocked off!}

Go back to where the path split into East (to the Ruins) and West (to
\place{Nyctalinth}) and explore around. You'll find something interesting
(not to mention the fights). This is an example of why it pays to explore
around no matter where you go.

\faqentry{What do I do after all the \monster{Man~O'~Grove} fights?}

You see that little square that isn't grass in the middle of all the
1$\times$1 pillars? Use the Bonsai Tree you got from Murkatos' \place{Inner
  Sanctum} on it It doesn't matter what you answer when you get asked a
question by the tree.

\faqentry{How exactly did that help me get in?}

Look around north of the blocked entrance for a tree with a gnarled face It
should be in an area next to city walls.

\faqentry{Why couldn't I have gone here in the first place?}

Because if you did go here before using the Bonsai Tree in that Grove, the
tree wouldn't let you up.

\faqentry{I can't get in Ratskells!}

Steal from a blind rat NPC outside. Check your inventory before and after you
meet him---you might get something stolen along the way! I've had key items
stolen this way before, so remember to save\fshyp{}restore until either
nothing is stolen or an item you don't care about gets stolen.

\faqentry{How do I get in the Funhouse?}

Ask \npc{Blienmeis} about ``Funhouse''. Oh, and have 1000 gold before you do.

\faqentry{What do I do in the Funhouse?}

You're going to be setting up a lot of things. You \textlcsc{need} a Detect
Secret or a good amount of scouting skill in here---especially after going
downstairs---or you won't be able to find the necessary items needed to set up
the place. You need various items found throughout the Funhouse and a Feather
Weight potion from Birdie's store. Don't forget the Black Pyramid and the
Wooden Dowel from upstairs! After you've set everything up, save, use the
Feather Weight potion (but on the main screen!) and set things off.

\faqentry{What does \npc{Barlone} want?}

Give him a time of the next T'Rang ship to leave\fshyp{}come. You should find
the information in the \place{Landing Port} (in the form of the T'Rang
Logbook) and the \place{Observation Control Center} (in the form of the
TX-Coder) in \place{Nyctalinth} Use the Coder on the Logbook while you're in
the full-screen view of your inventory to read it. You're advised to go to
\place{Nyctalinth} at this point---the information \npc{Barlone} gives you
will give you access to the \qmwordqm{DRAGON} map in \place{Old City}, but
the \qmwordqm{DRAGON} map is one of the maps that tends to get taken away
real fast---hence the need for haste. The \qmwordqm{BOAT} map in the Funhouse
however doesn't go away for a while in comparison to the \qmwordqm{DRAGON}
map but it \textlcsc{DOES} go away if you dawdle too long.

\faqentry{What's the ``another favor'' \npc{Barlone} mentions?}

Well, if \npc{Mick the Pick} or \npc{Ratsputin} is alive, you'll find out
when you meet them and then come back here to \npc{Barlone}.
Otherwise\ldots{}there's no way to find out. One of CODS' great
mysteries. \verb|:)| Think of it as an incentive to get here fast.

\faqentry{Aack! There's \textlcsc{more}? When does this ``Funhouse'' end?!?}

Not for a while. Have fun scrambling and trying to piece things out.  If
you're \textlcsc{REALLY} stuck on things, the Walkthrough at GameFAQs written
by Tom Needham will be able to get you through. It is a very good resource
for figuring out how to get through an area, although I disagree with some
things he states. You \textlcsc{WILL}, however, need to find a Bar \& Rope
and a Black Ball and used them in the appropriate areas to get through. You
will also need to make sure you've found a place to use the Wooden Dowel. You
will also find it completely necessary to fall down pits and such to find
everything you need. Hope you have \spell{Levitate} handy.

\faqentry{What do I do at the target?}

Throw the Speckled White Ball at it of course. Look for some ``cascading
troughs'' as the game calls it and jump in to find said ball.

\faqentry{What's with this rack of spears?}

It guards the \qmwordqm{BOAT} Map. Talk to \npc{H'Jenn-Ra} T'Rang in
\place{Nyctalinth} to figure out the combination, if you haven't talked to
him already.

\faqentry{I'm in wilderness after climbing out! Where am I?}

The \place{Witch Mountains}. Have fun looking around. Watch out for
\monster{Q'ua-tari}.

\pagebreak\subsection{\place{Old City}}\label{old-city}\nopagebreak
%
\begin{twocolumnitemize}{Monsters encountered:}
  \tightlist
\item \monster{Spirits}
\item \monster{Ghosts}
\item \monster{Puxic Oozes} (fixed)
\item \monster{Fungus Oozes}
\item \monster{Water Nymphs}
\item \monster{Dragonlizards}
\item \monster{Iguanadons}
\item \monster{Minoskells}
\item \monster{Skeletons}
\end{twocolumnitemize}

\faqentry{Why are there footprints in the dust?}

Why? Because someone got here before you and nabbed the \qmwordqm{DRAGON} map!

\faqentry{\texorpdfstring{Does that mean I should turn back
    then?\protect\footnotemark{}}{Does that mean I should turn back
    then?}}\footnotetext{Continuation of previous question.}

Not really. There's some useful random loot in here too---especially what is
probably your first Ankh. Most Ankhs can raise either Strength, Vitality,
Dexterity, Piety, or Speed, depending on what type of Ankh it is. There are
other Ankhs that do other things, but most of the ones that you run across
will be the stat-raising kind. If you already knew what to do to get in
however, you can go in here as early as after \place{Orkogre Castle}.

\pagebreak\subsection{\place{Nyctalinth}}\label{nyctalinth}\nopagebreak
%
\begin{twocolumnitemize}{Monsters encountered:}
\item \place{Nyctalinth}
  \begin{itemize}
    \tightlist
  \item \monster{T'Rang Youngers}
  \item \monster{T'Rang Guarders}
  \item \monster{T'Rang Wilders}
  \item \monster{T'Rang Keepers}
  \item \monster{T'Rang Watchers}
  \item \monster{T'Rang Assassins}
  \item \monster{T'Rang Wisers}
  \item \monster{T'Rang Elders}
  \item \npc{H'Jenn-Ra} (fixed)
  \item \monster{Rattkin Leaders}
  \item \monster{Rattkin Hunters}
  \item \monsterB{Rattkin Thieves}{Rattkin Thief}
  \item \monster{Savant Controllers}
  \item \monster{Savant Troopers}
  \item \monster{Savant Guards}
  \item \monster{Spirits}
  \item \monster{Fetid Corpses}
  \item \monster{Zombie Skells}
  \item \monster{Ymmu} (fixed)
  \item \monster{Vilet Kanebe} (fixed)
  \item \monster{Dragon Rooks}
  \item \monster{Hog Beetles}
  \end{itemize}

\item Underground area
  \begin{itemize}
    \tightlist
  \item \monsterB{Cave Thraxes}{Cave Thraxe}
  \item \monsterB{Crust Thraxes}{Crust Thraxe}
  \item \monster{Cave Slimes}
  \item \monster{Rock Lizards}
  \item \monster{Puxic Oozes}
  \item \monster{Fungus Oozes}
  \item \monster{Bear Weevils}
  \item \monster{Boar Weevils}
  \item \monster{Venom Weevils}
  \end{itemize}
\end{twocolumnitemize}

\faqentry{Where's \npc{H'Jenn-Ra}?}

Come on, it's not \textlcsc{that} hard to walk around and explore! He's in
the Northwest area in the \place{High Chamber} though, since you asked.

\faqentry{Should I say yes to \npc{Shritis}?}

It depends on if you hate the Umpani or not. Be warned that if you say yes,
\npc{Shritis} will constantly ask you (not to mention hunt you down) if
you've killed Rodan or not until you kill Rodan and keep his Golden Medallion
to present to \npc{Shritis}, or until he gets immensely ticked off at you and
attacks you every time he sees you.

If you really want to kill \npc{Rodan Lewarx} however, he's not \emph{too}
hard, for an NPC---use \spell{Missile Shield} to stop the bullets he shoots
at you most of the time.  It's those irritating \class{Alchemist}-type spells
he and his reinforcements may have that will get you.  Also, as a word of
warning, they get an increased chance of a critical hit with \textlcsc{all}
missile weapons, including Blunderbusses and Muskets.  Of course, if you say
no to \npc{Shritis} you're going to have to tread lightly every time you see
him---he'll be cranky when he sees you again, but at least he won't be trying
to hunt you down unless you kill T'Rang NPCs (including K'borra).

\faqentry{Should I go back to \place{Nyctalinth} after talking to
  \npc{Shritis} instead of going into \place{New City}?}

Yes, unless you got the information from \npc{Barlone} already---if you did you
should head for Professor Wunderland and the entrance to \place{Old City} to
get the \qmwordqm{DRAGON} map quickly before someone steals it.

\faqentry{Why don't these Finger Rods work on the \place{Breeding Grounds}
  door?}

Because they're for the Tactical Depot. The one for the \place{Breeding
  Grounds} can be found in the \place{High Chamber} on a T'Rang who is
cooking gruel.

\faqentry{What is the Mystery Ray for?}

You will use it on any Savant Androids you may find---especially the sleeping
ones in the T'Rang larder in the \place{High Chamber}. Use it as an item when
it prompts you with the Use, Talk, and Leave buttons. You can \emph{also} use
the Mystery Ray to hurt \emph{any} Savant Androids you run across---including
those nasty \monster{Savant Kui'sa-ka} near the end of the game and those Savant
Berserkers you wake up.

\faqentry{Is the information that the Savant Berserker spoke and the stuff
  that flashed on the computer screen in the \place{Observation Center}
  important?}

You betcha. Take a walk into the \place{Forbidden Zone} after you're bored of
\place{Nyctalinth}. Make sure you have the Comm-Link Device from the Umpani
Detache first though. BTW, if you helping \npc{Barlone}, I hope you remembered to
look for the TX-Coder in the \place{Observation Center}. Otherwise you won't be able
to read the logbook. Told you to build up scouting!

\faqentry{What do I do in the Graveyard? All I see are lots of graves, a
  building, and two alcoves with a psycho ghost who steals a jeweled staff
  before I can grab it.}

Well, you \textlcsc{have} to do three things: You have to grab the Longstem
Spade in the chest in the building here---it's going to be your shovel and
you're going grave robbing! \verb|:)| You also have to dig out the grave of
Notera Furmi (Not Firm) to get out of here. Lastly, you have to get the
\qmwordqm{SERPENT} map from the Tomb of Vilet Kanebe behind the energy
barrier, and that means getting that Crux of Crossing away from the crazy
ghost and equipping the cursed (literally) staff on someone while going
through the barrier---both in and out. That also means that you're going to be
in the graveyard at least twice---you need a little idol called the ``Tydnab
Emyt''\footnote{Time Bandit backwards with y's for i's} which can only be
found in the middle area of \place{Nyctalinth}--if you took out your Journey
Map kit and looked at that big black spot in the middle of the map, you'd
find it somewhere in the middle of that. Unfortunately you have to go through
the graveyard to get there.

\faqentry{Help! I've fallen and can't get up!}

I take it you dug up Notera Furmi's grave and fell in. Now, before you go
exploring, you have to watch out for gas pockets---some of your characters
will get poisoned from these. When you first bump into one, you should try to
step in every square in the room that had the Gas Pocket---there are more than
one, and they block all the exits. After that, deal with the poison however
you feel like; you stepped in every square so that you don't have to
\spell{Cure Poison} as many times, saving on mana.

The Gas Pocket Rooms also signify one of the two ways out---the Eastern Gas
Pocket room has a path that leads directly to the middle area of
\place{Nyctalinth}, and the Western Gas Pocket room leads to another exit
that is blocked by T'Rang eggs---you need a Thermal Pineapple to get out this
way, and if you have that you should know that one heck of a fight is on your
way out if you choose to go out that way\ldots{}and you can't avoid it if you
insist on using this route as the way out.

\faqentry{I'm in the middle area now. So where exactly is the Tydnab Emyt?}

It's beneath a statue of Phoonzang. (What a surprise\ldots{}) Use the Spade
to dig it up after you deal with the undead waiting in ambush where the
statue is.

\faqentry{So how exactly do I get the Crux of Crossing?}

Go into either alcove. It doesn't matter if the staff is in it or
not, or even if you already saw the ghost take it. Then just use the
Tydnab Emyt and watch.

\faqentry{Hey! I thought you said the \qmwordqm{SERPENT} map is here! I'm
  just getting waxy wrappings again!}

It is\ldots{}unless you took too long getting here and an NPC nabbed it
However, the \qmwordqm{SERPENT} map can usually be gotten even if you take
your time on expert mode---it's not as irritating to get as the
\qmwordqm{CRYPT} map or the \qmwordqm{CRYSTAL} map. Since I can usually get
it however, I can't take a guess as to who took it first.

\faqentry{\npc{H'Jenn-Ra} is killing my party! How do I deal with him?}

I guess you had a Pineapple after all. I won't lie to you---if your party is
not faster than \npc{H'Jenn-Ra}'s party, you're screwed unless you're on
easy mode, where there's a chance she'll appear by herself, or you're much
higher level, where most of their stuff has a reduced chance of working on
you. You will need the full gamut of protective and enhancement spells to
kill her otherwise, including \spell{Air Pocket} but not Fire or \spell{Ice
  Shield} unless she has company. On Expert mode, you need at least 2
characters that can cast mass target spells like \spell{Nuclear Blast} or
\spell{Word of Death} to kill off all the reinforcement T'Rang or the
Assassins and Elders will lay waste to you. Fire spells tend to be more
effective on the T'Rang. Good luck!

\paragraph{Note:} If you travel from \place{Nyctalinth} to
\place{Dionysceus}, you will run into T'Rangs ambushing the Helazoid
\npc{Jan-Ette}. Help her if you want the Eagle Eye skill later and to be on
friendly terms with the Helazoid.

\pagebreak\subsection{\place{Witch Mountains}}\label{witch-mountains}\nopagebreak
%
\begin{twocolumnitemize}{Monsters encountered:}
  \tightlist
\item \monster{Bantari}
\item \monster{Q'ua-tari}
\item \monster{Vampire Vultures}
\item \monster{Fire Crows}
\item \monster{Dragon Rooks}
\item \monster{Vampire Rooks}
\item \monster{Vultures}
\item \monster{Wood Dryads}
\item \monsterB{Faerie Witches}{Faerie Witch}
\item \monster{Forest Giants}
\item \monster{Boulder Giants}
\item \monster{Frost Giants}
\item \monster{Dinkle Wisps}
\end{twocolumnitemize}

\faqentry{What do I do here?}

Well, if you've returned the Holy Work to \npc{Xen Xheng} and joined his school of
5 Flowers, you can start collecting the flowers for one. If you haven't read
the Book of Fables from \place{New City} yet, take it out and read it now,
and remind yourself about when the witches meet. This next part doesn't seem
to be hinted anywhere in the game---you have to high-tail it to the
\place{Giant Cave}. That means you're going to be looking for a path
\textlcsc{DOWN}, so you're going to look for a cliff where if you take
another step, you fall into the air. Make a mental note of a place where you
can climb up if you find it first---you will go there later. If you end up
near a river, you're in the wrong direction---you're on the \place{Eryn River}
right now, which you've been to before.

\faqentry{So what's up the path that you told me not to go up right now?}

Well after the \place{Giant Cave}, you should climb up and explore. Besides
flower hunting (and always save before trying to pick any flower now!)  you
are also looking for the entrance to the \place{Witch Cave}, a very high
cliff that doesn't look like it can be climbed (but someone tried before),
and four 250~ft.\ vines to merge together into one big 1000 vine to be used
at the cliff If you find the cliff first, drop any vines you've found already
there to relieve encumbrance problems---they're very heavy.  You need to climb
down into that area before going into the \place{Witch Cave}.

\faqentry{And what exactly do I do after that?}

Why go into the \place{Witch Cave} with what you learned of course!

\pagebreak\subsection{\place{Giant Cave}}\label{giant-cave}\nopagebreak
%
\begin{twocolumnitemize}{Monsters encountered:}
  \tightlist
\item \WviiSPOT{} (fixed)
\item \monster{Gruengard} (fixed)
\item \monster{Bonehead} (fixed)
\item \monster{Brunatz} (fixed)
\item \monster{Munstachio} (fixed)
\item \monster{Boulder Giants}
\item \monster{Forest Giants}
\item \monster{Frost Giants}
\item \monster{Water Nymphs}
\item \monster{Red Pirannhas}
\item \monster{Glow Mothras}
\item \monster{Rock Lizards}
\item \monster{Dinkle Wisps}
\item \monster{Jelly Stingers}
\end{twocolumnitemize}

\faqentry{These giants keep killing me! How can I kill them?}

You kill them like anything else. I don't advise using attacking spells on
them, although cloud-type spells like \spell{Poison Gas} and
\spell{Firestorm} can help deal the damage.

\faqentry{\WviiSPOT{} is insane! Any pointers on how to stop him?}

Treat him as a more\ldots{}dangerous version of the giants you've been
killing. Consequently, anyone that can't dish out (and take) damage should be
hidden away via hide or invisibility potions bombing \WviiSPOT{} with
cloud-type spells and boosting spells on your party. \spell{Create Life} is a
\textlcsc{great} help here, but not the other two summon spells--\WviiSPOT{}
can flatten those silly demons quickly, but the \monster{Myxlmynx} might be
able to take a couple more hits.

\faqentry{Anything special here?}

Yes. \WviiSPOT{} has the Necromantic Helm somewhere in his lair.  Look for
it; it's the witching glass mentioned in the Book of Fables.  Also, the Red
Rosis is in the underground river. If you can swim four or more steps (and
have a mass target spell) go to the nearest access to the underground river
east of the entrance to the giant caves and swim south to that little
chokepoint of water (where you fight a lot of fishes---hence the need for the
mass target spell)---the Red Rosis should be on that little piece of dry land
right next to you to your west. The alternate (and slower way) is to take the
far eastern entrance to the water and swim to another region of the caves
that will wrap around the bottom area and stick you near the Red
Rosis\ldots{}after a couple fights.

There's also a chest in the extreme south area\ldots{}but you'll need either
Stamina Potions or lots of Water mana for Stamina and Restfull to get to it.


\pagebreak\subsection{\place{Witch Grove}}\label{witch-grove}\nopagebreak
%
\begin{twocolumnitemize}{Monsters encountered:}
  \tightlist
\item \monster{Bantari}
\item \monster{Q'ua-tari}
\item \monster{Vampire Vultures}
\item \monster{Fire Crows}
\item \monster{Dragon Rooks}
\item \monster{Vampire Rooks}
\item \monster{Vultures}
\item \monster{Wood Dryads}
\item \monsterB{Faerie Witches}{Faerie Witch}
\end{twocolumnitemize}

\faqentry{Huh? I've never heard of this place!}

It's actually my name for the area of the \place{Ukpyr} Mountains you climb
down into from the tall cliff.

\faqentry{So what exactly is so special about this area that you gave it a
  made-up name?}

You'll find out the names of the four witches here. Remember the witching
hour comment from the Book of Fables? That's right---you need to explore
around at night. If you find the area here that gives your characters the
chill, stay out of the clearing, rest until night, and then go back in with
the Necromantic Helm equipped. Too bad nothing seems to really hint that the
Necromantic Helm is the ``witching glass''. You can also find the White
Dahlia in this area.

\pagebreak\subsection{\place{Witch Cave}}\label{witch-cave}\nopagebreak
%
\begin{twocolumnitemize}{Monsters encountered:}
  \tightlist
\item \qmwordqm{Alethedies} (fixed)
\item \qmwordqm{Dk.~Savant} (fixed)
\item \qmwordqm{Vi~Domina} (fixed)
\item \qmwordqm{Statue} (fixed)
\item \monster{Witch's Lights} (fixed)
\item \monster{Spirits}
\item \monster{Ghosts}
\item \monster{Nightmares}
\item \monster{Dream Weavers}
\item \monster{Minoskells}
\item \monster{Puxic Oozes}
\item \monster{Cave Slimes}
\item \monster{Rock Lizards}
\item \monster{Luna Mothras}
\item \monster{Glow Mothras}
\item \monster{Spectral Ravens}
\item \monsterB{Faerie Witches}{Faerie Witch}
\end{twocolumnitemize}

\faqentry{Do I \textlcsc{HAVE} to pay the witches 1000 gold?}

Yes. Sorry. That means you need 4000 gold here\ldots{}not that you're in the
poorhouse at this point in the game anyway.

\faqentry{What exactly is this Elysiad of Divinity?}

It is for the Final Boss fight. I'll mention which fight it is when you get
there. Basically, stick it on your toughest character---when it is used it
immediately resurrects \textlcsc{ALL} your characters and restores them to
full hit points, stamina, and Mana even if they weren't resurrected.  At the
end of battle it does this \textlcsc{AGAIN}, making sure that everyone get's
the massive amount of experience points for it (hey, 800,000 so XP per
character for six characters on Expert mode is \textlcsc{NICE}, although
you'll get less on Normal and Easy.) Best of all, it doesn't add to your
``life'' number, so it treats you like you never actually died. It
\textlcsc{ONLY} works for the Final Boss battle however. Then there's the
tiny detail that your character has to be \textlcsc{ALIVE} to use it.

\faqentry{Wow! The \qmwordqm{SPHINX} Map! No one beat me to it?}

I guess having it locked up means there's no way in hell someone can beat you
to it. This is usually one of the last maps I get---because it never seems to
disappear no matter how long I take!

\pagebreak\subsection{\place{Ukpyr Mountains}}\label{ukpyr-mountains}\nopagebreak
%
\begin{twocolumnitemize}{Monsters encountered:}
  \tightlist
\item \monster{Bantari}
\item \monster{Q'ua-tari}
\item \monster{Rock Lizards}
\item \monster{Vampire Vultures}
\item \monster{Fire Crows}
\item \monster{Dragon Rooks}
\item \monster{Vampire Rooks}
\item \monster{Vultures}
\item \monster{Wood Dryads}
\item \monsterB{Faerie Witches}{Faerie Witch}
\item \monsterB{Mtn. Thraxes}{Mtn. Thraxe} (fixed)
\end{twocolumnitemize}

\faqentry{I take it the other two flowers are here?}

Yep. Save before picking either of them. Don't forget to look for the
entrance to the \place{Sphinx Cave}. You'll get here by the way from climbing
down the vine on the tall cliff to the area where the spirits were partying,
and then exploring further, or by exiting \place{Ukpyr} to the north and
snooping around all the sidepaths and the main path.

\pagebreak\subsection{\place{Sphinx Cave}}\label{sphinx-cave}\nopagebreak
%
\begin{twocolumnitemize}{Monsters encountered:}
  \tightlist
\item \monster{Red Pirannhas}
\item \monster{Vampire Vultures}
\item \monster{Fire Crows}
\item \monster{Dragon Rooks}
\item \monster{Vampire Rooks}
\item \monster{Vultures}
\item \monster{Wood Dryads}
\item \monsterB{Faerie Witches}{Faerie Witch}
\item \monster{Water Nymphs}
\item \monster{Rock Lizards}
\end{twocolumnitemize}

\faqentry{What do I do here?}

Read that handy \qmwordqm{SPHINX} map you just got. The Rebus Egge that you
got from attempting to get the Wand Majestik in \place{New City} is used
here---you \textlcsc{DID} assay\fshyp{}identify it eventually to find out what
it was to tell it apart from other stones right? It's the ``seed'' the map
refers to. You're going to need to swim at most 3 to 4 spaces by the way.

\faqentry{What do I do at the slugs in the floor?}

Isn't your \spell{Detect Secret} going off right now?

\pagebreak\subsection{\place{Isle of Crypts} \& \place{Mandolian Isles}}\label{isle-of-crypts-and-mandolian-isles}\nopagebreak
%
\begin{twocolumnitemize}{Monsters encountered:}
\item Above ground
  \begin{itemize}
    \tightlist
  \item \monster{Night Rooks}
  \item \monster{Dragon Rooks}
  \item \monster{Vampire Vultures}
  \item \monster{Glow Mothras}
  \item \monster{Luna Mothras}
  \item \monster{Red Pirannhas}
  \end{itemize}

\item Below ground

  \begin{itemize}
    \tightlist
  \item \monsterB{Ghosts of Gorn}{Ghost of Gorn} (fixed)
  \item \monster{Ungorn Daimyo} (fixed)
  \item \WviiDOOM{} (fixed)
  \item \monster{Wraiths}
  \item \monster{Spirits}
  \item \monster{Kolidras}
  \item \monster{Necromani}
  \item \monster{Fieros}
  \item \monster{Phantasmagoras}
  \item \monster{Skeletons}
  \item \monster{Zombie Skells}
  \item \monster{Fetid Corpses}
  \item \monster{Skeleton Lords}
  \item \monster{Q'ua-tari}
  \item \monster{Bantari}
  \item \monster{Dragon Rooks}
  \item \monster{Fire Crows}
  \item \monster{Witch's Lights}
  \end{itemize}
\end{twocolumnitemize}

\faqentry{Um\ldots{}which island do I go to?}

Which island has more than a 1$\times$1 pillar?

\faqentry{What exactly is on the islands with
  \texorpdfstring{1$\times$1}{1x1} pillars?}

You can walk through them to go somewhere special. But you're going to need
something in the \place{Hall of Gorrors} before being able to do anything
special with them\ldots{}and yes, you have to go in the pillars eventually.

\faqentry{I found this picture of the Sphinx on one of the walls. Anything
  special?}

Use the Wand Majestik on it. If you're a Greek Mythology buff (or at least
have been paying attention when the Sphinx was talking before!)  you'll know
the question she refers to, but you don't have to get the question right.

\faqentry{Waaa! That's a lot of booby traps!}

Yep. Save often. You're looking for secret buttons and levers. You're also
looking for a gold urn to burn that extra use of Jonga Powder.  Aren't those
traps nasty? If you want an (easier) time getting to the urn, after you get
to the area with fountains go east (turn left that is) and hug the left wall
until you get to all the little alcoves, then hug the right wall, ignoring
any alcoves for the moment. You'll only run into a few fixed encounters for
the first half of the way and get smacked by Evilspeak once or twice along
the second half.

\faqentry{What the hell? Why are my characters stoned (or various other nasty
  effects) after drinking? I thought fountains were beneficial?}

Not all of them. S\textlcsc{ave} before trying one. Heck, most of them here are
booby traps---I can only think of two fountains on this level that aren't
completely bad.

\faqentry{OK, so I've burned the Jonga Powder and turned off the nasty traps.
  What now?}

Explore of course. There's at least 1 nice treasure chest in here that you
can get no matter what keys you have. The best item you can get in said chest
is a Dragon Kite---a shield for \class{Valkyrie}s, \class{Fighter}s, and
\class{Lord}s that reduces their armor class by 4. That's if you have them
and need one however. Otherwise, a Faery Cap if you missed it before (or
Chamail items) is the \nth{2} best. Most of the other stuff is overshadowed by
better loot elsewhere. Bt the way, search all the Gorn Noblemen you can find,
but save before doing so!

\pagebreak\subsection{\place{Dragon Mountains}}\label{dragon-mountains}\nopagebreak%
%
\begin{twocolumnitemize}{Monsters encountered:}
  \tightlist
\item \monster{Emerald Dragons} (fixed)
\item \monster{Dragonessas}
\item \monster{Dragorras}
\item \monster{Dragon Pups}
\item \monster{Fetid Corpses}
\item \monster{Vampire Vultures}
\item \monster{Dragon Rooks}
\item \monster{Fire Crows}
\item \monster{Komodo Dragons}
\item \monster{Rock Lizards}
\item \monster{Dragonlizards}
\item \monster{Dinkle Wisps}
\item \monster{Jelly Stingers}
\end{twocolumnitemize}

\faqentry{OK, how do I get here?}

Well, according to the map that came with the game, it's waaay south
and a little west. Maybe you should follow the coast that way until you
hit something unusual\ldots{} (Minor note: You're looking for fog!)

\faqentry{I found the fog, but when I go through there's just a blank cliff
  here!}

Come here at night, then take a peek at the \qmwordqm{SERPENT} map. You're
going to need something from the Dane Tower here you know\ldots{}

\faqentry{Uh oh. Brombadeg's tough!}

No kidding. \verb|:)| Well OK, he's not \textlcsc{that} hard. Silence seems
to stop the ``cast enchantment'' thing he does, but I'm not going to
guarantee it Treat him like any other unique monster though and you'll be
fine.

\faqentry{Geez---this place is huge! What do I do?}

Read the \qmwordqm{DRAGON} map for hints of course. And watch out for
cave-ins Have fun exploring and watch out for the tightwad! (Well, you have
to find the tightwad anyway.)

\faqentry{A cave-in just occurred! How do I get out?}

You will never get completely trapped by a cave-in---all the places that have
cave-ins have an alternate way out, although at least one of those ways
requires you to have lots of swimming if you want to make it back to the
boat.

\faqentry{Where do I use the Key of the Dragon?}

Remember the gate with the black dragon on it in the \place{Isle of Crypts}?

\faqentry{Where does that ladder go?}

Climb up and find out. It's (relatively) safe.

\pagebreak\subsection{\place{City of Sky}}\label{city-of-sky}\nopagebreak
%
\begin{twocolumnitemize}{Monsters encountered:}
  \tightlist
\item \monster{Savant Kui'sa-ka}
\item \monster{Savant Controllers}
\item \monster{Savant Guards}
\item \monster{Helazoid Aces} (fixed)
\item \monster{Xenozoid Runners}
\item \monster{Xenozoid Flyers}
\item \monster{T'Rang Elders}
\item \monster{T'Rang Assassins}
\item \monster{T'Rang Watchers}
\item \monster{Meta-Droid} (fixed)
\item \monster{Ravens}
\end{twocolumnitemize}

\faqentry{Hey! What's with the invisible walls?}

Beats me. You're going to have to figure out the path through them.  Good
luck and happy mapping. \emph{evil grin} It's not that hard. Make a saved
game at the ladder and then try to find your way around. If the fights here
are too hard, I suggest you go gain experience elsewhere, gain perhaps 4 or 5
levels, and then come back and see if they are easier for you. The reason is
because by using a FAQ (or the clue book) to get answers, there is the
possibility that you are too low level for the area, and levels are very
important in this game as it affects many things, including your resistances
and whether enemy monsters will resist your attacks or not.

If you're too lazy though, look at John Needham's FAQ at GameFAQs (if the
site you're using for FAQs lacks it) for directions on how to get to
\npc{Dame Ke-Li}---if you don't have a Credit Card you have to go there
first---and then you probably want to buy some of her fun supply of items as
well.

\faqentry{Insert one credit? Where do I get a credit?}

It's referring to credit cards. Surely the Helazoids you've bumped
into dropped one or two? If not, look around for \npc{Dame Ke-Li} here---she
sells Credit Cards for 100 gold.

\faqentry{How do I get into the Storage Facility?}

The Storage Key for the Storage Facility is in the \place{Hall of
  Preservation}.

\faqentry{What about that gazebo with the glowing key? There are invisible
  walls around it.}

You need to get into the Storage Facility first. Then go into an alcove on
your way out (through the other side---there are two exits in the Storage
Facility).

\faqentry{What is the Key of Light used for?}

Remember those \textlcsc{really} nice looking weapons in the \place{Hall of
  Preservation}?

\faqentry{So which one should I pick then?}

Like all choices, it is up to you. The Frontier Phasers and the Cobaltine
Power Glove require Powerpaks for use however. If you've imported your party
from Bane of the Cosmic Forge, you probably have the Diamond Ring which can
open another one. The Light Sword and the Power Glove get bonuses when
fighting against robots like the Battle Droid (but not against Savant
Androids). Be warned that when you use the Diamond Ring to grab an item, the
Ring gets sacrificed for the item.

\faqentry{Anything special I should do with \npc{Dame Ke-Li}?}

You mean besides buy all those nice, wonderful things she's selling?
Yeah---give her the Helazoid Pennant you got from \npc{Jan-Ette} if you helped
\npc{Jan-Ette}. Other than that, she's the perfect person to have fun using the
Five Fingered Discount. Just don't get caught.

\faqentry{What exactly does that PK Crystal do?}

−4~AC bonus, Regen~+1, Psionic Resistance +50\%, usable only by
\class{Psionic}s.

\faqentry{What are the answers to the tests in the Hall of Crusaders?}

The first one is very obvious if you've been paying attention during the
game! The second one requires help from the \qmwordqm{STAR} Map and the
\place{Gaelin Stone} in the \place{Hall of Gorrors}. The third one\ldots{}well,
you'll see.

\faqentry{How do I get in the spaceship?}

Speak the answer to the first test.

\faqentry{What do I do inside?}

Search the place, like any other group of nosy adventurers would.  Don't stop
searching until the game keeps you from searching!

\faqentry{Where can I get more Powerpaks, other than killing lots of Helazoids?}

Buy those Credit Cards from \npc{Dame Ke-Li} and use them in a machine in the
Storage Facility---the machine gives out Powerpaks for Credits. If you really
have to make money at this point, you \emph{can} keep getting Powerpaks and
then selling them to \npc{Paluke}---or anyone other than \npc{Dame Ke-Li}.

\pagebreak\subsection{\place{Hall of Gorrors} (in the \place{Isle of
    Crypts})}\label{hall-of-gorrors-in-the-isle-of-crypts}\nopagebreak
%
\begin{twocolumnitemize}{Monsters encountered:}
  \tightlist
\item \monster{Ra-sep-re-tep} (fixed)
\item \monster{Thing from Hell} (fixed)
\item \monster{Horragoth} (fixed)
\item \monster{Myxlmynx} (fixed) / \monster{Fieros} (fixed)
\item \monster{D'Arboleth} (fixed)
\item \monster{Wraiths} (fixed)
\item \monster{Fiend of 9 Worlds} (fixed)
\item \monster{Beast of 1000 Eyes} (fixed)
\item \monster{Kolidra Necromani}
\item \monster{Red Pirannhas}
\end{twocolumnitemize}

\faqentry{Do I have to kill any of these Gorrors?}

Not really. They're supposed to be challenges, with the exception of the
lonely little \monster{Ra-sep-re-tep} in the SE area.

\faqentry{You said I needed something in here?}

Yeah. Whip out the \qmwordqm{CRYPT} map and read it. Then follow the
directions You'll nab something that needs to be used in the \place{Mandolian
  Isles}.

\faqentry{What do I do with it?}

Remember seeing an indentation with a similar design on the Isles?  Stick it
in that pillar and then go in the other one.

\faqentry{How do I open the gate to the \place{Hall of the Past}?}

Do you have the \qmwordqm{LEGEND} map? It's also called the Gaelin Legend.
Funny, there's a stone in the same area called the \place{Gaelin Stone}. I
wonder if there's a connection\ldots{}maybe you can use the map on the stone
somewhere.

\faqentry{These !@\#@! chests here hurt! Is there any reason to open them?
  The loot I'm getting sucks!}

Well, yes there is a reason to open them: They will occasionally spew out the
best items, weapons, and armor in the game. You're going to be here for
\textlcsc{HOURS} finding nice stuff to keep. \verb|:)| If you
\textlcsc{really} just want to know what's in there, go to
\url{http://www.softwarespecialties.com/GorrorsRevealed.html}. While you're
at it, visit \url{http://www.softwarespecialties.com/} and thank the guy
(Llevram that is) if you did look for his hard work. \verb|:)|

\faqentry{I want to kill the Gorrors! Can't you suggest anything?}

Well, let's see\ldots{}besides always cast \spell{Enchanted Blade},
\spell{Armorplate}, and \spell{Magic Screen} before the fight (except for
\monster{Ra-sep-re-tep})\ldots{}

\monster{Ra-sep-re-tep}: If you actually need help on him, that's not a good
sign.  \verb|:)|

\paragraph{\monster{Thing from Hell}} Quick and Dirty Way---Silence him, then
\spell{Deadly Poison} him. He'll go down fast.

Slow and Boring Way---Silence him and whack away. Use \spell{Armormelt} to
speed things up.

\paragraph{The Demon \monster{Horragoth}} Get high enough level that \spell{Death
  Cloud} and \place{Death Wish} have no effect on you. Then treat him as a
monster that does nothing except a rare \spell{Word of Death} or two and
start hacking away. Kill his companion(s) first though.

\paragraph{The Spirit \monster{D'Arboleth}} Watch out for
\spell{Lifesteal}. Otherwise treat him the same way as \monster{Horragoth},
minus the \spell{Word of Death}.

\paragraph{The \monster{Fiend of 9 Worlds}} Immediately Hide\fshyp{}Drink
Invisibility Potion.  Pummel him with as many different gas spells as
possible, i.e.\ \spell{Draining Cloud}, \spell{Poison Gas}, \spell{Toxic
  Vapors}, \spell{Firestorm}, \spell{Acid Bomb}. Especially \spell{Toxic
  Vapors}. Cast \spell{Create Life} for a distraction. Send out either
someone with Reflextion or with \spell{Armor Shield} and other protective and
hit-improving spells (who can hide again) to attack\fshyp{}hide once he gets
nauseous from the \spell{Toxic Vapors} and renew that spell and the non-air
based gas spells with a vengeance. Casting \spell{Armormelt}, \spell{Haste},
(and \spell{Missile Shield} if you're on expert mode) and casting
\spell{Armor Shield} on the attacker should help. Keep the \spell{Cure
  Poison} and \spell{Heal Wounds} prepared.

\paragraph{\monster{Beast of 1000 Eyes}} He's tough. Very tough. The way I
beat him was with a level 50th so party on easy mode: Everyone casted
beneficial spells the first turn (Especially \spell{Fire Shield} and
\spell{Missile Shield}), then hid\fshyp{}turned invisible the next. If no one
is visible, he won't attack (usually). Then, I casted \spell{Create Life} to
bring out a punching bag for the Beast and pumped the Beast with all the
Cloud-type spells I had, the same ones as the ones used on the Fiend. After
that, everyone who could hide went in waves of Attack\fshyp{}Hide whenever it
seemed like the Beast was overdue for a coughing fit or was otherwise
distracted by my summoned help, and refreshed the Cloud spells
constantly. Eventually (after 4 hours) he croaked.

Another suggestion I saw on the Software Specialties board is to equip
someone who can't hide with the Necromantic Helm, Bat Necklace or another
Light-resisting item like the Helazoid Pendant, and have him take all the
\spell{Dazzling Lights} hits; \spell{Dazzling Lights} will go straight for
the character if he's the only one visible, but with 100\% so Light
resistance + \spell{Magic Screen}, you should be able to resist it and shrug
it off.  This is because if everyone is invisible and the Beast casts
\spell{Dazzling Lights} at you (because you had someone visible in the
beginning of the round), it is more likely to hit everyone with the Lights.
Trying with an average level party of around 30th so level is extremely
difficult; once you get your party to around (or over) 100th level however,
the Beast stops becoming so invulnerable.


\pagebreak\subsection{\place{Hall of the Past}}\label{hall-of-the-past}\nopagebreak
%
\begin{twocolumnitemize}{Monsters encountered:}
  \tightlist
\item \monster{Rexx}
\item \monster{Myxlmynx}
\item \monster{Wraiths}
\item \monster{Spirits}
\item \monster{Bloodwyrms}
\item \monster{Yreguoths}
\item \monster{Necromanis}
\item \monster{Fieros}
\item \monster{Phantasmagoras}
\item \monster{Kolidras}
\item \monster{Skeleton Lords}
\end{twocolumnitemize}

\faqentry{Um\ldots{}I need a map. Bad.}

Sorry, I don't have one for you. You're going to have to do it yourself,
although having a 70+ Mapping skill here \textlcsc{really} helps.

\faqentry{Gee, thanks for the help.\texttt{\{/sarcasm\}}}

Heheh \texttt{\^{}\^{};;;}\ldots{}well, before you go in, you'll probably
want \spell{Dispel Undead}, \spell{Astral Gate}, \spell{Fire
  Shield}\ldots{}basically all the commonly used spells I listed waaay
earlier. If you didn't use them before, you're definitely going to be using
them here eventually. The only exception is probably
\spell{Asphyxiation}--most things in here aren't affected by it. Bring lots
of Moser's Mojo Tea your first time through, as well as Heavy
Heal\fshyp{}Stamina potions. Don't settle for the weaker stuff by this point.

\faqentry{There's a fountain there that I keep getting teleported away from!}

You'll need to find the Key of Waters and open a gate elsewhere that opens
into the same room. As long as you avoid the teleporter spot, you'll be able
to drink from the fountain. It restores mana only.

\faqentry{What's with these crystals?}

Read the \qmwordqm{CRYSTAL} map. It applies to here. You have to hit all
eight crystals, and one of them won't figure into the descriptions on the map
Just so you know: if you get it wrong, when you step in the pentagram it will
open up a passageway to the left. When you're right it will open up a passage
to the right instead.

\faqentry{Great, I've found the Star map. How exactly do I use it?}

\begin{quoting}
  Look first at a man, and if thee looks rightly, then soon shall ye come
  full circle. Then look beneath him, and if thee looks rightly, then soon
  shall thee once again come full circle. Thus may thee divine the puzzle
  from the pieces, and from it derive thy solutions.
\end{quoting}

That's the actual lines in the map that you're interesting in. If you've
copied down the buttons from the \nth{2} test, great. If you haven't, they're
Serpent, Gate, Wand, Pyramid, Star, Dragon, Cross, Skull, Map. Start at the
side with a Man symbol, look for one of the buttons in that group of symbols
(the Temple Map) and write it down. Move to the next face on the right side
of the stone (in other words, circle the stone counterclockwise) repeating
the same thing but using the top section only. Once you're back on the side
with the Man, do the same thing for the bottom row. The buttons will fall
into place.

\pagebreak\subsection{\place{Tomb of the Astral Dominae}}\label{tomb-of-the-astral-dominae}\nopagebreak
%
\begin{twocolumnitemize}{Monsters encountered:}
\tightlist
\item \npc{Dk.~Savant} (fixed)
\item \monster{Savant Kui'sa-ka} (fixed)
\item \monster{Cosmo-Bots} (fixed)
\item \monster{Mega-Bots} (fixed)
\item \monster{Meta-Droids} (fixed)
\item \monster{Battle Droids} (fixed)
\item \monster{Mega-Bots}
\item \monster{Meta-Droids}
\item \monster{Battle Droids}
\end{twocolumnitemize}


\faqentry{Any suggestions on surviving these robots?}

Liberal use of \spell{Magic Screen}, \spell{Fire Shield}, and \spell{Ice
  Shield} will dampen the damage from the cannons. Can't do squat about the
lasers other than kill the darn things as fast as possible, although
\spell{Missile Shield} may deflect them. The Light Sword and Power Glove will
do double damage to these guys though.

\faqentry{How do I open the tomb?}

Look for various places in the tomb that have something written on
the floor and step on them. This will take you all over the map. After
you have stepped in all those marked spaces, the gate will automatically
open.

\faqentry{Hey! Is the Astral Dominae gone?}

No. Just signal Vitalia to come on down and give her the Ring of the Globe
and the Locket of the Tomb.

\faqentry{Huh? Where did I get the Locket from?}

You \emph{did} remember to go back up to those 1$\times$1 pillars after
solving the \qmwordqm{CRYPT} map\ldots{}right?

\faqentry{What about the Ring?}

Did you finish the Great Test in \place{City of Sky} and completely search the
shuttle?

\faqentry{And while we're baffled, how the heck do we signal Vitalia?}

You \emph{did} visit the \place{Forbidden Zone} in \place{New City} after
getting that strange information from the Savant Berserker and the
\place{Observation Center} in \place{Nyctalinth} and the Comm-Link
Device\ldots{}didn't you?

\faqentry{Why does the Dark Savant keep wiping the floor with my party?}

You're just unlucky. \texttt{:(} Still, you can use the Elysiad in this
fight, and this fight only. Personally, I didn't really need it. The Mystery
Ray was helpful too, at least on the Kui'sa-ka. Hope you have 100 Mind
Control by the way: they \textlcsc{LOVE} \class{Psionic}s.

\faqentry{Why does Vitalia ask me about a spaceship?}

This decides what endings you can get. Yes means that you have to go to
Phoonzang's shuttle to grab an ending. No means you go back to the
\place{Forbidden Zone} for the endings, although I didn't have problems
getting an ending from Phoonzang's shuttle as well by saying no.

Various other possible questions:

\faqentry{What should I bring with me for the ending?}

That is (sort of) covered in the very last part of the Spoilers section.

\section{Spoilers}\label{spoilers}
%
This is a (minor) spoiler section If you don't like spoilers, ignore this
last section. There's a reason why the spoiler section is
\textlcsc{last}--although some of the questions above are semi-spoilish in
nature, this section assumes that you know most of the things in the game
Besides any spontaneous blurbs I may have, if you've got a question about the
game that I feel is spoilish I will add it here.

\subsection{Items}\label{items}
%
As I'm a rather avid item collector, I thought it would be nice to spill
some information just for the heck of it

\faqentry{Where are all the bard instruments? What do they do?}

\WviiNeedSpace{}%
Here are all I know of:%
\begin{longtable}{@{}
  P{0.25\linewidth-\tabcolsep}
  p{0.2\linewidth-\tabcolsep}
  P{0.55\linewidth-\tabcolsep}@{}} \toprule
Poet's Lute & Sleep & Most bards start with it. \\[1.2ex]
Angel's Tongue & \spell{Bless} & Buy/steal from \npc{Brother TShober} when you first meet him. \\[1.2ex]
Chromatic Lyre & Itching Skin & \place{Land of Dreams}. \\[1.2ex]
Lute of Sloth & Slow & Search the area where you summoned Maa-Gogg the \monster{Man~O'~Grove} for a chest. \\[1.2ex]
Silent Lyre & Silence & \place{Dionysceus}, in the Book of Immortals chest. \\[1.2ex]
Pipes of Doom & Terror & Buy\fshyp{}steal from \npc{Blienmeis} at Ratskells in \place{Rattkin Ruins} when he is in Ratskells. \\[1.2ex]
Cornu of Demonspawn & \spell{Astral Gate} & Kill \WviiSPAWN{} \\[1.2ex]
Siren's Wail & Confuse & \place{Dragon Mountains}, in the pirate chest or a Gorrors chest. \\[1.2ex]
Horn of Prometheus & \spell{Fireball} & Gorrors chest or a certain \place{Hall of Past} chest. \\[1.2ex]
Lyre of Cakes & Healthfull\footnotemark{} & A certain \place{Hall of Past} chest---you choose between this and the Horn of Prometheus. \\ \bottomrule
\end{longtable}

\footnotetext{Lyre of Cakes has been proven to not give back your health. People
  have theorized that it instead acts like a mass version Cure Lesser
  Condition.}

\faqentry{Where are the items that give me personal skills?}

Everywhere. Not all skills require items by the way:

\paragraph{Firearms} Have a character practice on the firing range; can be
given to all, but each character must practice individually; you can't have
everyone learn the skill after one person uses it---only the characters
practicing will learn the skill.

\paragraph{Snakespeed} Get the 5 flowers as \npc{Xen Xheng} asked and merge them
together Use a Holy Water to merge with the White Dahlia before merging the
Dahlia with the other 4 merged flowers. Only one person can get this.

\paragraph{Reflextion} Ask \npc{Blienmeis} about ``Reflextion'' when he is in
Ratskells. He will offer to sell you a Ring of Reflextion for 12000 gold.
Use the Ring's special ability to give the character using it
Reflextion. Only one character\fshyp{}game can get this.

\paragraph{Power Strike} Use the Gem of Power's special ability to give the
character using it Power Strike. The Gem of Power is located in the secret
room in Murkato's \place{Inner Sanctum}. Only one character\fshyp{}game can
get this.

\paragraph{Eagle Eye} Rescue \npc{Jan-Ette} from the \monster{T'Rang
  Assassins}. Give the Helazoid Pennant she gives you to \npc{Dame Ke-Li} at
\place{City of Sky} for the Helazoid Pendant. Use the Pendant's special
ability to give the character using it Eagle Eye. Only one
character\fshyp{}game can get this.

\paragraph{Mind Control} Kill the Psi-Beast(s) on the \place{Temple of
  Eternal Night} level of \place{Dionysceus} and return to the room the Danes
were in on the same level to meditate. Anyone who doesn't fall asleep gains a
point in the skill.

\faqentry{Can I get the other items in the Hall of Preservation?}

As far as I know, the answer is no if you didn't import a party from Bane of
the Cosmic Forge. The Diamond Ring you get in Bane of the Cosmic Forge can
open up the doors as well, although like the Key of Light it goes away when
you use it---which means you have to decide whether the item inside is worth
losing a −5~AC accessory that females can use. In my opinion, the Cobaltine
Power Glove, the Light Shield, and the Light Sword are all worth it, but not
the Frontier Phasers.

\faqentry{What do the wandering NPCs sell?}

\begin{twocolumnitemize}{This is from what I can remember, so there may be some mistakes:}
\item \npc{Lord Galiere} and \npc{King Ulgar}
  \begin{itemize}
    \tightlist
    \item Spear
    \item Awl Pike
    \item Spear+2 (rarely or only in the beginning of the game)
    \item Quilt Tunic (sometimes)
    \item Quilt Leggings (sometimes)
    \item Fur Leggings (sometimes)
    \item Cuir Gauntlets (sometimes)
    \item Leather Cuirass (sometimes)
    \item Studded Hauberk (sometimes)
  \end{itemize}

\item \npc{Capt.~Boerigard}
  \begin{itemize}
    \tightlist
    \item Spear+2
    \item Studded Hauberk
  \end{itemize}

\item \npc{Brother TShober}
  \begin{itemize}
    \tightlist
    \item Random potion (most of the times)
    \item Random potion (sometimes)
    \item Random potion (rarely)
  \end{itemize}

\item \npc{Xen Xheng}
  \begin{itemize}
    \tightlist
    \item Sai (sometimes)
    \item Nunchukas (sometimes)
    \item Ninjato (sometimes)
    \item Shuriken (15)
    \item Bushido Blade (higher levels)
    \item Do-Maru (U)
    \item Do-Maru (L)
    \item Tosei-do (U) (higher levels)
    \item Tosei-do (L) (higher levels)
    \item Ninja Cowl\protect\tikzmark{xen-xheng-topbrace}
    \item Ninja Garb (U)
    \item Ninja Garb (L)
    \item Tabi Boots\tikzmark{xen-xheng-bottombrace}
      \begin{tikzpicture}[overlay, remember picture]
        \draw [decoration={brace,amplitude=0.5em},decorate,ultra thick,black]
        let \p1=(xen-xheng-topbrace), \p2=(xen-xheng-bottombrace) in
        ({max(\x1,\x2)+2em}, {\y1+0.8em}) -- node[right=0.6em] {\parbox{5em}{may not have all at once}} ({max(\x1,\x2)+2em}, {\y2});
      \end{tikzpicture}
  \end{itemize}

\item \npc{Kymas Turan}
  \begin{itemize}
    \tightlist
    \item Mitre
    \item Brimstone Nuggets
    \item Ju-ju Stones (sometimes)
    \item Dragon's Teeth (sometimes)
    \item Random scroll
    \item Random scroll
    \item Random scroll (sometimes)
    \item Random scroll (rarely)
    \item Random scroll (rarely)
  \end{itemize}

\item All Umpani NPCs
  \begin{itemize}
    \tightlist
    \item Powder and Shot (25) (sometimes)
    \item Walriblade (sometimes)
    \item Flak Vest (sometimes)
  \end{itemize}

\item All T'Rang NPCs
  \begin{itemize}
    \tightlist
    \item ??? (I think nothing, but possibly Shock Rod)
  \end{itemize}

\item All Rattkin NPCs
  \begin{itemize}
    \tightlist
    \item Various keys
    \item (Don't remember what else---they die too fast)
  \end{itemize}

\item \npc{Jan-Ette}
  \begin{itemize}
    \tightlist
    \item Credit Card (most of the time)
    \item Powerpak (sometimes)
  \end{itemize}
\end{twocolumnitemize}

Any maps the NPCs may have are always displayed first when you go to the
``Buy'' screen.

\faqentry{What are some good equipment that I can get?}

Depends on your point in the game. Here's some treasure troves where you can
get items better than what you should normally be getting if you go
\textlcsc{ASAP}:

\paragraph{\place{Orkogre Castle}} Try to get into the \place{Armory}
early. (You have to get in anyway.) Leather and Studded Hauberks may be
helpful as well as the bows and the occasional naginata or halberd.

\paragraph{\place{Eryn River}} After meeting \npc{Brother TShober}, take the
first path you see on the left, and follow it to the first place with water
that you see Save, and swim north---you'll need 30 swimming however. Follow
the path to the end; the giant battle at the end is a doozy for a weak party.
When you open the chest there though, you will get a Crusaders's 2H Axe~+1,
Bracers of Defense, Chain Mail~+1 (U), Chain Mail~+1 (L), and a Crusader Helm
as well as gold.

\paragraph{\place{Giant Cave}} You can get here early by exploring the
\place{Eryn River} map (The same map that you meet \npc{Brother TShober} the
first time) and locating the path that leads to the \place{Witch Mountains}
area. You should max out your swimming however. When you enter, crack open
the Journey Map Kit and as west and south as possible (you may have to
explore the entire west area) until you both hit water and can't go west
anymore. Now, if you have some area effect spells like \spell{Fireball} handy
and lots of Water points for Stamina spells (or lots Stamina potions), look
at the area this way.

You should be facing south, with a wall to your right, part of the passageway
to your left, and water in front of you and to the left in front of the other
part of the passageway.  The water should have walls that are visible from
this point.  Now, go left until you're line up with the wall on the left
side, so that if you took two or three steps forward you'd be lined up with
the wall.  Now for the first annoying part: Take two steps forward and into
the \monster{Glow Mothra} fight.  Deal with them.  (Level 4
\spell{Asphyxiation} works wonders here.)

Go back to shore and rest, then get ready to use some spells\fshyp{}items and
head back into the water facing south all the way until you get to an opening
on your left.  Go in that watery opening, but before you go into the more
open room after the opening, get some stamina back and (if you want to, save)
walk into the room to fight \monster{Pirannha Sharks}, which are tougher
\monster{Red Pirannhas}.

After you've (hopefully) successfully dealt with them, go forward until you
hit the wall (you should be facing east now), turn right, take a step forward
in the water, turn left (facing east again) and you'll see a treasure chest.
Go in and get your Plate Mail~+2 (both upper and lower) and some random loot.
Reload the loot until you see some weapon you like---I like the Cat~O'~9
Tails, but some may like the Vulcan Hammer or Sword of Fire.

\paragraph{\place{Nyctalinth}} In the underground area, if you explore past
the Western Gas Pocket Area, you will eventually find a path that leads to an
encounter with \monsterB{Cave Thraxes}{Cave Thraxe}, but starts with some
text. After you fry those bugs, if you go to the northeast corner of the room
and search, you'll find a Broadsword~+1 and random loot. Reload until you get
something you like, which may include a Dragon Kite.

\paragraph{\place{Rattkin Funhouse}} After you do \npc{Barlone} a favor, if
\npc{Ratsputin} is still alive you can run into him and he will demand that
you pay dues that are 20\% of your current amount of gold. Pay at least once,
then return to \npc{Barlone}---he will offer you his treasure trove and a
Displacer Cloak (you better have inventory room before talking!) for 40,000
gold. The loot is decent and can include a Dragon Kite (again).

\paragraph{\place{Isle of Crypts}} West area after the area with all the
fountains contains a secret button in one of the alcoves. Press it, go in the
passage, deal with the demon fight, and pick the chest open for a last chance
at Chamail items, a Faerie Cap, or a Dragon Kite without wasting a spot in a
Gorrors chest (although you might get the Dragon Kite again in the
\place{Dragon Cave}).

Or, if you feel you are strong enough, go to the extreme southwest area of
the map (where the Ugly Gorn Woman corpse is buried) and enter the alcove
with the Gorn Nobleman that is directly west of her.  Save, cast preparatory
spells, then search to fight \monster{Ungorn Daimyo} and possibly a bunch of
\monster{Skeleton Lords}.  Defeat them for the Hi-Kane-Do (U and L), a
Kabuto, a Bushido Blade, and a Wakizashi~+1

\faqentry{Where can I get the Cane of Corpus?}

Kill \npc{Blienmeis}. Easier said than done of course.

\faqentry{Where are some monster generators?}

Several places:

\paragraph{\place{New City}} In the \place{Control Center}, stick the Black
Wafer instead of the Control Card in the slot to summon \monster{Savant
  Guards} and Troopers.

\paragraph{\place{Orkogre Castle}} In the area below the prison; you must
step in the square you fall in to respawn.

\paragraph{\place{Greater Wilds}} You will find an area that has a path
circling a small area. Step in the middle of that area to set off an
encounter. Step on the square you had to go through to get in this area to
reset the encounter.

\paragraph{\place{Isle of Crypts}} The intersection in front of the fountain
in the \place{Hall of the Past}. Encounters reset automatically.

\subsection{Gameplay}\label{gameplay}%
%
\faqentry{What's the best party to get through the game?}

Almost anything works, really. If you want to be able to use the most items
in the game, you will need an \race{Elf} fighter-type, a \race{Faerie}
\class{Ninja}, (the following non-\race{Faerie}) \class{Samurai},
\class{Ranger}, \class{Valkyrie}, \class{Psionic}, \class{Bishop}, and one of
the character profession changed from a (non-\race{Faerie}) bard.  You still
can't use the Medicine Bag, but you're not missing much from it. If you want
to just play for fun, anything can work. If you want to get through the first
areas quickly, you may opt for a simple party of two \class{Fighter}s, a
\class{Mage}, a \class{Priest}, a \class{Thief}, and a \class{Bard} for
relative speed through the initial parts. Heck, even a party of all
\class{Ninja}s can get through the game. If all else fails, use the change
profession button to get a class that you need at the moment once you have
the minimum stat requirements to change into the class.

\faqentry{Should I use missile weapons?}

You can if you want to. A lot of them are pretty nice in damage and
effects. However, the really nice missiles are in limited quantities; unless
you cheat, you may end up finding yourself never using them. If you are the
economic type, you will hate any missile weapons, including the Musket and
Blunderbuss, which must be reloaded every other turn in order for you to use
it effectively. Keep in mind however that missile weapons are long range, and
can target monsters no matter where they are.

\faqentry{Should I put the difficulty on easy? There are some places in the
  game where I just want to get past a fight to go on with the game, not
  spend extra time gaining levels, but if I do I feel like I'm
  cheating\ldots{}}

If you're just trying to get through the game, there's nothing wrong with
it. Expert mode is there when you think that the game has become too easy; I
wrote this playing on Expert mode, and I can tell you that some fights that
seem easy on Easy are royal pains on Expert. The \npc{Magna Dane} fight is a
good example.

\faqentry{What are some battle strategies I can use?}

Well for starters, you can try to keep \spell{Enchanted Blade},
\spell{Armorplate}, and \spell{Magic Screen} on all the time. Although they
will be less effective than if you had just casted them should you get into a
nasty fight, at least you will be somewhat prepared for it. In-battle spells
you may wish to cast to beef up your characters are \spell{Bless},
\spell{Armor Shield}, \spell{Haste}, and \spell{Superman} on someone who will
be attacking a lot. Most of the other helping spells should only be used if
the situation calls for it---for example, casting \spell{Ice Shield} and
\spell{Fire Shield} when you're up against enemies that can use the
appropriate type of spells or attacks. If you have a bard, take advantage of
his or her instruments and start hammering the enemy away with a status
ailment (or just Blessing everyone constantly with the Angel's Tongue).

\faqentry{I keep missing the \qmwordqm{LEGEND} map. Is there any way I can
  get it faster?}

Actually, yes. You can say the trigger ``Holy Sacrament'' to \npc{Father
  Rulae} at any point in the game, and he will let you in. Be warned---do
\textlcsc{NOT} say Holy Sacrament if you just sacrificed all your money, as
the Holy Sacrament pathway will open up, closing the path to the other
fountain he has.

\section{Transferring to Wizardry~8}\label{transferring-to-wizardry-8}%
%
\faqentry{What items transfer?}

Quite a bit actually. Your starting equipment and low level equipment for
sure. As for other, more powerful things, it depends on the type of item.

\paragraph{Armor} Anything you can buy in \place{New City} should transfer.
Also, Chamail and Chain Mail+1 will transfer.  Nothing else that you find in
the last several areas of the game appear to transfer though, so far,
although the Wizard's Cone should transfer.

\paragraph{Weapons} All Bows, Arrows, Crossbows, Quarrels, Slings, and Sling
Bullets appear to transfer, no matter what kind, with exception to the Siege
Arbalest.  Any weapon you can buy in the \place{New City} area when you
initially enter the area also appears to transfer, but not later equipment
like Vorpal Blades.  Also, Stun Rods transfer and are \textlcsc{HEAVILY}
suggested as they do rather obscene amounts of damage for a low level party
Although I don't have one on a transferring party, the Spear+2 should most
likely transfer as well.  The Mystery Ray transfers and goes from a weapon
that only hurts Savant Androids to a weapon that can hurt anything.
Strangely enough, Powder and Shots will transfer, but not Muskets or the
Blunderbuss.  Any Shuriken will transfer, so if you have Death Stars, you
keep them.

\paragraph{Miscellaneous} All reagents transfer.  The Book of Relics
transfers The Valentine Necklace and Storage Key are supposed to transfer
according to “myrddin” on the Software Specialties Wizardry~8 forum.  Bat
Necklaces transfer.  Scrolls, Wands, and Potions transfer (even Moser's Mojo
Tea!).  Mana Stones transfer.  Cross of Protection apparently transfers,
according to Scronan of the Ironworks Wizardry~8 board.

\paragraph{Bardic Instruments} Poet's Lute, Angel's Tongue, Chromatic Lyre,
Siren's Wail, Cornu of Demonspawn, and Lute of Sloth will transfer.

\paragraph{What definitely does not transfer} Any of the good Gorrors
treasure.

\faqentry{Do all items transfer automatically?}

Yes, they all transfer automatically.

\faqentry{Will my skills transfer?}

Yes and no. Don't expect your personal skills to, except Firearms. All your
scores will be reduced to approximately \fractionsOn{1/5} or less of what
they were originally, so a 100 in Swords turns into a 20 in Swords. Also,
skills that aren't supposed to be raised by the class you transfer in are
there, but unavailable. Your spells are stripped to 2 level 1 spells of your
own class.

\faqentry{Where do I end up starting in Wizardry~8?}

You start with who you left with. \verb|:)| Failing that, you start in a
\place{Monastery}. I do not know what you start with if you decide to take
the Globe for an ending however.

\faqentry{I want to ask you something about Wizardry~8. Can I?}

That's not a good idea, as I do not have a FAQ for Wizardry~8
specifically. This section is here only because it also pertains to Wizardry~7.

\chapter{Miscellaneous}\label{ch:misc}

\section{Version Updates}\label{version-updates}%
%
\begin{description}[style=nextline, labelwidth=4.5em, leftmargin=!, labelindent=0em]
\item[v1.95b (06/05/16)] \LaTeX{} conversion for this PDF.

\item[v1.95a (11/17/01)] Quick minor correction.

\item[v1.95 (11/16/01)] Added a section about transferring to Wizardry~8 and
  made a few changes. Trimmed down the list of places where this FAQ is
  displayed due to various sites going down and changing over time.

\item[v1.94 (3/3/01)] Made a few changes to some areas of the FAQ and credited
Bernice Carter.

\item[v1.93 (10/20/00)] Added Neoseeker to the list of places that can display
this FAQ.

\item[v1.92 (10/13/00)] Added VGMasters to the list of places that can display
this FAQ. Did some minor editing here and there, and added another two
possible questions so that this wouldn't feel like a totally pointless
update. \verb|;)|

\item[v1.91 (10/2/00)] Added Cheat Empire to the list of places that can display
this FAQ. Also did some minor editing, and added a paragraph's worth of
words in the closing paragraph.

\item[v1.9 (8/15/00)] Forgot to trim down another section in the beginning.
Minor editing. Expecting to update with a full Walkthrough next update,
but don't hope too much.

\item[v1.8 (8/7/00)] Trimmed down some areas and spaced out others as per
Dallas' recommendations, and minor editing. Added a question in \place{Ukpyr}.

\item[v1.7 (7/9/00)] Fixed some errors thanks to the people who volunteered on
the Software Specialties board. Thanks guys.

\item[v1.6 (7/5/00)] Added a new section in Spellcasting. Editing changes.
Added \url{www.rpgclassics.com} to the list where this FAQ is displayed.

\item[v1.5 (6/22/00)] Added www.the-spoiler.com to the list where this FAQ is
displayed. Very slight editing changes.

\item[v1.4 (6/20/00)] Added www.psxcodez.com to the list where this FAQ is
displayed. Very slight editing changes.

\item[v1.3 (6/13/00)] Minor editing corrections. Added Spoilers section.

\item[v1.2 (6/8/00)] Minor editing corrections. Sources added.

\item[v1.1 (6/7/00)] Added Version updates section and two sites to the list
where this FAQ is displayed.

\item[v1.0 (6/6/00)] None.
\end{description}

\section{Sources}\label{sources}%
%
\begin{itemize}
\item Software Specialties Wizardry~7 board Crusaders of the Dark Savant Clue
Book (comes with Wizardry Gold, although it has some errors that you
don't know about until you play through the game)
\item My own playing experience
\end{itemize}

\section{Future plans}\label{future-plans}%
%
The bottom item is what will most likely come up next.  Don't expect anything
on the top to come anytime soon.

\begin{itemize}
\item
  HTML version of the Dark Savant clue book. I won't be copying it, but
  it will try to be in a similar format. It will also use the maps below
  to help navigate around. Don't expect this in the near future at all,
  as my HTML is horrible.
\item
  Monster list, with the type of weapon the monster wields, what spells
  it casts, and any special abilities. Don't expect this to occur before
  September of 2001, and if any other CRPGs come out within that time,
  don't expect it at all. Still, it will get done eventually, although
  now I don't know if I'll finish it before Wizardry~8 comes out.
  However, Llevram has kindly put together a program that does all this
  already in the meantime. \verb|:)| You should be able to get it at
  www.softwarespecialties.com. Just scroll down the left menu and click
  on the last link there.
\item
  Full color maps saved from the game (and slightly modified in cases
  like the Funhouse where the waterslide isn't mapped). Some of these
  are actually done, but I don't have time to go through each and every
  single area and map everything the way it appears on the Journey Map
  Kit, \textlcsc{AND} pass my classes I'll get too distracted.
\item
  A spoiler walkthrough. This FAQ is only a semi-spoiler and skips some
  things. The walkthrough will try to guide you on the quickest way
  around things. I can probably get this done before the Monster List
  and is next on my to do list for this FAQ unless someone has a
  suggestion for an existing section of the FAQ. This has now been
  combined with the Full color maps. I will most likely table this and
  everything above due to lack of time, but let's see how my time turns
  out.
\item
  A beginner's section that is designed to explain how some things work
  that will not spoil the game in anyway.
\item
  Any minor corrections and suggestions to the FAQ. I'll try to do what
  I can, but remember: I'm only human
\end{itemize}

\section{Afterword}\label{afterword}
%
Thanks for reading this guide. As I've said before, any comments and such
non-spam and non-flame in nature are appreciated. If you have a question not
answered here, I'll be happy to add it after you ask me.  Right now, I'm in
the middle of mapping all of the game and doing the ``online Walkthrough''
train of thought to kill two birds with one stone. This is going to take a
\textlcsc{LOT} longer than I expected, due to the need to do assignments for
class and the relocation (and swapping) of the computer I was using to map
the game, so it is on hiatus for the moment while I get my life organized. I
can still answer questions however as well as update this FAQ if more
questions get asked pertaining to the game (but not technical stuff)

\clearpage
\phantomsection\addcontentsline{toc}{chapter}{Index}%
\printindex

\phantomsection\addcontentsline{toc}{chapter}{Monsters index}%
\printindex[monsters]

\phantomsection\addcontentsline{toc}{chapter}{NPCs index}%
\printindex[npcs]

\phantomsection\addcontentsline{toc}{chapter}{Places index}%
\printindex[places]

\phantomsection\addcontentsline{toc}{chapter}{Spells index}%
\printindex[spells]

\end{document}
